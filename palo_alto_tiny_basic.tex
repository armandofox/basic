
\section{PALO ALTO, 1970 - COMPUTING FOR EVERYONE}

\makequotation{Let us stand on each others' shoulders, not on each
  others' toes.}{Dennis Allison, Stanford University, 1975}

%  what did 'universal access' mean in the culture of 1970 CA?


\milestone
The microprocessor and the first personal computer, MITS Altair

Fast forward to 1970s California, when the first personal computers (Altair,
Mark-8) were becoming available.  By this time, BASIC's popularity as a
beginner language had spread, and most computer manufacturers offered
some version of BASIC on their computers.
Bob Albrecht, a refugee from computer
maker Control
Data, had become uneasy with the industry's emphasis on computers for
corporations rather than people.  He had
started the People's Computer Company in Menlo Park, a nonprofit walk-in
computer center where anyone could walk in and pay per hour to use
terminals connected to a \w{PDP-8}, a popular inexpensive computer made by
\w{Digital Equipment Corporation}, and (later) a remote Hewlett-Packard
mainframe computer on which 
HP had donated time.  People wrote and ran small-business programs, played games
(mostly written in BASIC), and so on.
A frequent visitor was Ted Nelson, an early proponent and visionary of
the idea of linked hypertext.  Like Kurtz, Nelson believed computers
would be important and everyone should learn about them; the rallying
cry of his book \emph{Computer Lib/Dream Machines} was ``You can and
you must understand computers now!''


\milestone
Now, with ``personal'' computers on the horizon, there was an
opportunity to refocus on democratizing computing power for individuals.
In possibly the earliest example of \w{open source} software,
Albrecht convinced colleague Dennis Allison, an idealistic Stanford
lecturer, to create a ``\w{Tiny BASIC}'' that could be
given away for free and would be ``lite'' enough to run
on the emerging PCs, most with less than 4KB RAM, so that programming
might be accessible to all~\cite{allison_tiny_basic}.  
(Albrecht would later help co-author the introductory programming manual
that came with the breakthrough \w{Commodore VIC-20}
computer~\cite{commodore}.) 
Allison released Tiny BASIC in 1975, and with Albrecht, started the
newsletter \emph{Dr. Dobb's Journal of Tiny BASIC 
Calisthenics and Orthodontia} (``Dobb'' being a contraction of their
first names)
%% as a sister publication to the People's Computer Company newsletter 
to publish information about it.
By overwhelming reader demand, the quirky newsletter was eventually
turned into a full-fledged monthly publication for computer enthusiasts
named simply \w{Dr. Dobb's Journal}, which continued in print
until 2009 and still circulates online.  In 1985, Dr. Dobb's Journal
would publish the 
\w{GNU Manifesto}, programmer-activist Richard Stallman's call to action
to support the cause of free software.

\begin{geeknote}{}
  Tiny BASIC defined a subset of BASIC (no string support, integer
  arithmetic only) in terms of a virtual machine.  By mid-1976,
  interpreters for the ``Tiny BASIC intermediate representation''
  existed for the Intel 8080, Motorola 6800, and MOS Technology 6502
  microprocessors. 
\end{geeknote}

Palo Alto Tiny BASIC and the TRS-80 Model I

democratization hippie style

cite ``What the Dormouse Said''
