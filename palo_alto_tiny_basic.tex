
\section{MENLO PARK, 1970 - PEACE, LOVE, AND COMPUTERS}

\makequotation{Let us stand on each others' shoulders, not on each
  others' toes.}{Dennis Allison, Stanford University, 1975}

%  what did 'universal access' mean in the culture of 1970 CA?

The People's Computer Company in Menlo Park, California, 
might well be the forerunner of the Internet caf\'{e}.
Bob Albrecht, a former employee of computer maker Control Data,
was disturbed by
the computer industry's emphasis on selling computers to corporations and
the military.
He thought computers should empower ordinary people to do new things,
and in order to make computer usage affordable, in
1970 he
started PCC as a ``drop-in'' facility 
where anyone could pay by the hour to use a shared computer.
%% As John Markoff recounts in \emph{What the Dormouse Said}, the
%% ``share alike'' mentality of counterculture 1970s California was
%% deeply ingrained in the PCC's mission.
%% and (later)
%% a remote Hewlett-Packard
%% mainframe computer on which HP had donated time.
%% By this time BASIC's popularity had spread, so many PCC users wrote
%% (and used) small-business programs and games written in BASIC on the
It helped that the maverick startup company Digital Equipment
Corporation (DEC) had recently released the disruptively-priced and
phenomenally successful
\href{https://www.computerhistory.org/revolution/minicomputers/11/331}{PDP-8}
minicomputer, which cost one-fifth as much as the least expensive IBM
computer (\$18,500 when introduced in 1964m about \$147,000 in 2018).

%% A frequent visitor was Ted Nelson, an early proponent and visionary
 %% the idea of linked hypertext.
%% Like Kurtz, Nelson believed computers
%% would be important and everyone should learn about them; the rallying
%% cry of his book \emph{Computer Lib/Dream Machines} was ``You can and
%% you must understand computers now!''

But a much more significant portent of democratization was just around
the corner---literally.
In 1971, the young electronics company Intel (for ``Integrated
Electronics''), whose Santa Clara headquarters were only about 15
miles from Menlo Park, had won a contract to design and manufacture chips for
a line of calculators made by a Japanese company.
Engineer Ted Hoff came up with the idea of creating a single
``generic'' chip that could be programmed to work in different
calculator models.
That chip, the Intel~4004, was not powerful enough to be the center of
a true computer; but from the point of view of design, it was the
world's first microprocessor.

\begin{tangent}
  The invention of the Intel~4004 had some parallels to
  DEC's development of the PDP-8.  In both cases, the respective
  companies had actually been asked to design custom hardware for a
  specific task: the PDP-8 was to have been a control
  system for a nuclear reactor, just as the 4004 was to have been the
  heart of a set of chips for making
  calculators.  In both cases, the engineers had the insight that a
  general-purpose computer with \emph{software} tailored for the
  custom task might be more cost-effective in the long run because the
  same hardware could be used for other applications.  This pattern
  would be echoed again starting in the 1990s: general-purpose PCs
  plus sophisticated software have replaced special-purpose hardware
  in virtually every application of computing, including video games, mobile
  phones, medical equipment, and special-purpose computers for working
  with graphics and sound.
\end{tangent}

Seeing the possibilities, Intel rapidly improved the design to allow
more powerful operations and take advantage of advances in chip
manufacturing.
In 1974 Intel introduced the 8080, the first inexpensive
microprocessor powerful enough to build a credible small computer
around.
That computer was the MITS~Altair 8800, announced in 1975 on the
cover of \emph{Popular Electronics}.

\smallpicfigure{figs/PopularElectronics.jpg}{fig:altair}{The Altair 8800
  on the cover of the January
  1975 \emph{Popular Electronics} is actually an empty case, since the
  computer itself was lost in the mail.
}

For \$397 (\$1,800 in 2018), you received a bag of parts, a
circuit board, and a schematic.  If you could solder and assemble it properly, you 
could build a ``microcomputer'' for your own use that ran nearly half as fast as the shared
``minicomputers'' at PCC.  (DEC itself had coined
``minicomputer'' to distinguish its computers from the much
larger and more expensive ones from IBM.)
For \$100 more, you could
order the computer pre-assembled, though the waiting time was nearly a year since
there were few MITS employees available to do the soldering by hand.

Non-hobbyists may have had to squint hard to see the Altair 8800 as an
actual computer.
The only way to enter information was via front-panel switches;
the only way it could display information was a row of LEDs.
The circuit board's connectors, similar to the ``expansion slots'' PCs
would have later, accommodated expensive adapters (not included) that
let you plug in a keyboard or tape drive (also not included).
But Bob Albrecht immediately saw the potential of the Altair: as the
first computer based on a low-cost mass-produced microprocessor rather
than on a custom design, it could radically democratize computing.
In possibly the earliest example of open source software, Albrecht convinced
colleague Dennis Allison, an idealistic Stanford computer science
lecturer, to propose a specification for a ``lite'' BASIC that could
be implemented on emerging personal computers like the Altair.
In 1975, Allison published a design proposal in the People's Computer
Company newsletter for ``Tiny BASIC''~\cite{allison_tiny_basic}, which
defined
a subset of BASIC compact enough to be implemented on emerging microcomputers.
Allison invited readers to submit implementations of his design, 
emphasizing that the design was not proprietary and a major goal was to
share successful implementations with others.  As we will see shortly,
Allison was motivated in part by the fact that Bill Gates's new
company had already created a BASIC that would work on microcomputers, but was charging
\$150 for it (about \$680 in 2018).

\smallpicfigure{figs/drdobbs_logo.png}{fig:drdobbs_logo}{
Typesetter and paste-up artist Eric
Balakinsky was told to create a logo for a magazine about Tiny BASIC,
which was described to him as ``an exercise in computer
programming\ldots{}that doesn't use very many bytes of memory.''
Balakinsky whimsically captured this as ``running light without
overbyte,'' and combining Dennis and Bob into ``Dobb,'' coined
the title \emph{Dr.~Dobb's Journal of Tiny BASIC Calisthenics and
Orthodontia}~\cite[p.~265]{dormouse}.
}

The response from readers was so overwhelming that Albrecht and
Allison decided to spin off a new magazine devoted to free and
low-cost software for the home computer, even though at the time there
was no home computer industry to speak of.
The quirky \w{Dr.\ Dobb's Journal} eventually became a full-fledged
monthly magazine for PC enthusiasts that would continue in print until
2009 and still circulates online.
By mid-1976, Dr. Dobb's had published Tiny BASIC implementations
for the Motorola 6800, MOS Technology 6502, and Intel 8080
microprocessors.
The 8080 version, called \w{Palo Alto Tiny BASIC} and created by computer engineer
and Homebrew Computer Club member Li-Chen Wang, would become one of the
most influential.  The wildly successful TRS-80 Model~I from Radio Shack---among the first 
fully-assembled, ready-to-run
personal computers under \$1000---came with a direct descendant of
Palo Alto Tiny BASIC built in, and the 
``BASIC Programming'' cartridge for the Bally Astrocade
home video game console, the first time programming
was conceptualized as a video game,
would also use a version of Palo Alto Tiny
BASIC.
Tiny BASIC may also be the first software listing to include
the phrase ``Copyleft, All Wrongs Reserved,''
to emphasize that the
implementation was intended to be freely shared.
As John Markoff recounts in \emph{What the Dormouse Said}~\cite{dormouse},
the ``share-alike'' ethos of the California counterculture strongly
influenced the early development of the technical community, which
consisted of many of the same people; that ethos is arguably the
basis of Open Source today, a phenomenon that has yet to be replicated
in fields outside of computing.
Fittingly, in 1985 Dr.\ Dobb's would publish the \w{GNU Manifesto},
programmer-activist Richard Stallman's call to action to support the
cause of free software and institutionalize the term ``Copyleft.''

But the ``share-alike'' counterculture exemplified by Albrecht and
Allison was about to collide with a much more business-centric culture
exemplified by two ambitious hackers from Seattle.




