
\section{MENLO PARK, 1970 - PEACE, LOVE, AND COMPUTERS}

\makequotation{Let us stand on each others' shoulders, not on each
  others' toes.}{Dennis Allison, Stanford University, 1975}

%  what did 'universal access' mean in the culture of 1970 CA?

The People's Computer Company in Menlo Park, California, 
might well be the forerunner of today's
modern Internet caf\'{e}s.
Bob Albrecht, a former employee of computer maker Control Data, started PCC
is 1970
as a nonprofit computer center where anyone could walk in and pay
by the hour to use a computer.
Albrecht was disturbed by
the industry's emphasis on computers for corporations and
the military rather than for people.
His idea was to make computers available to ordinary people---in the
case of PCC, those computers were just
terminals connected to a DEC~PDP-8, and (later)
a remote Hewlett-Packard
mainframe computer on which HP had donated time.
By the early 1970s, BASIC's popularity as an easy-to-learn beginner
language had spread, 
and most mainframe computers, including those available through PCC, 
could run BASIC programs.
So most PCC users wrote (and used) small-business programs
and games in BASIC.
%% As John Markoff has recounted in \emph{What the Dormouse Said}, the
%% ``share alike'' mentality of counterculture 1970s California was
%% deeply ingrained in the PCC's mission.

%% A frequent visitor was Ted Nelson, an early proponent and visionary of
%% the idea of linked hypertext.  Like Kurtz, Nelson believed computers
%% would be important and everyone should learn about them; the rallying
%% cry of his book \emph{Computer Lib/Dream Machines} was ``You can and
%% you must understand computers now!''

In 1971, Intel had developed the world's first microprocessor, the
4004.  Intel had won a contract to make a set of calculator chips for a
Japanese company, and engineer Ted Hoff came up with the idea of
creating a single ``generic'' chip that could be programmed to work in
different calculator models.  While the 4004 was not powerful enough to
be the center of a true microcomputer, Intel improved the design and in 
1974 introduced the 8080, the first inexpensive microprocessor powerful
enough to build a credible microcomputer around.  
That microcomputer was the MITS~Altair 8800, announced in 1975 
on the cover of \emph{Popular Electronics}.

For \$397 (\$1,800 in 2018), you got a bag of parts, a
circuit board, and a schematic.  If you could solder and assemble it properly, you 
could build a computer whose performance was nearly half that of a typical
``mainframe'' computer such as those at People's Computer Company.
%% for \$100 more (\$482 more in 2015) you could
%% order it pre-assembled, though the waiting time was nearly a year since
%% there were few MITS employees available to do the soldering and
%% debugging.  
Input was via
front-panel switches for painfully entering a
single machine language instruction at a time; output was a row of  LEDs for
displaying the binary value of a single memory location.  The circuit board's
edge connectors, similar to the later PC's ISA slots, accommodated 
peripheral adapters if you wanted to plug in a keyboard or tape drive.

\smallpicfigure{figs/PopularElectronics.jpg}{fig:altair}{The Altair 8800
  on the cover of the January
  1975 \emph{Popular Electronics} is actually an empty case, since the
  computer itself was lost in the mail.
}

Non-hobbyists may have had to squint
hard to see this as an actual personal computer,
but Albrecht
immediately saw that the Altair could  further democratize computing.
In possibly the earliest example of open source software, he
convinced colleague Dennis Allison, an idealistic Stanford computer
science lecturer, to propose a specification for a ``lite'' BASIC that could
be implemented on emerging personal
computers like the Altair.
The idea was that the hobbyist community would be invited to contribute
implementations of this spec for different computers, and the
implementations could be given away for free to encourage others to
learn programming.
In 1975, Allison published a design proposal in the People's Computer
Company newsletter for ``Tiny
BASIC''~\cite{allison_tiny_basic}, which defined 
% the Backus-Naur grammar for 
a subset of BASIC with no string support and integer arithmetic only.
Allison invited readers to submit implementations of his design and
emphasized that the design was not proprietary and a major goal was to
share successful implementations with others.

The overall response was so overwhelming that Albrecht and Allison
decided to spin off a new magazine devoted to free and low-cost software
for the home computer, even though at the time there was no home
computer industry to speak of.
They explained to resident typesetter
and paste-up artist Eric Balakinsky that Tiny BASIC was ``an exercise in
computer programming\ldots{}that doesn't use very many bytes of
memory.''
Balakinsky whimsically captured ``exercise'' as
``calisthenics'' and ``doesn't use very many bytes'' as ``orthodontia''
(what you get to avoid overbite), and combining Dennis and Bob into
``Dobb,'' coined the title \emph{Dr.~Dobb's Journal of Tiny BASIC
Calisthenics and Orthodontia}~\cite[p.~265]{dormouse}.
The quirky \w{Dr. Dobb's Journal} eventually became a
full-fledged monthly magazine for PC enthusiasts that
would continue in print until 2009 and still circulates online.
By mid-1976, Dr. Dobb's had published Tiny BASIC interpreter listings
for the Motorola 6800, MOS Technology 6502, and Intel 8080.
The 8080
version, called Palo Alto Tiny BASIC and created by computer engineer
and Homebrew Computer Club member Li-Chen Wang, would become one of the
most influential.

\begin{tangent}
The source code for Palo Alto Tiny BASIC may be the first time
the phrase ``Copyleft, All Wrongs Reserved'' appeared in print (in the
comments at the top of the listing).
Fittingly, in 1985 Dr. Dobb's would publish the \w{GNU Manifesto},
programmer-activist Richard Stallman's call to action to support the
cause of free software and institutionalize the term ``Copyleft.''
\end{tangent}

Palo Alto Tiny BASIC was destined to become one of the most influential
implementations. 
The wildly successful TRS-80 Model~I from Radio Shack---one of the first 
fully-assembled, works-out-of-the-box  
personal computers under \$1000---would carry a direct descendant of
Palo Alto Tiny BASIC in its ROM.
The ``BASIC Programming'' cartridge for the Bally Astrocade
home video game console, the first time programming
was conceptualized as a video ``game,''
would also use a version of Palo Alto Tiny
BASIC.

But home video game consoles---and even PCs---were still a few years
away, and when they arrived, two cultures would collide: the northern
California hippie ``share-alike'' counterculture exemplified by Albrecht
and Allison, and a much more business-centric culture exemplified by
two farsighted hackers from Seattle.



