
\section{MENLO PARK, 1970 - PEACE, LOVE, AND COMPUTERS}

\makequotation{Let us stand on each others' shoulders, not on each
  others' toes.}{Dennis Allison, Stanford University, 1975}

%  what did 'universal access' mean in the culture of 1970 CA?

The People's Computer Company in Menlo Park, California, 
might well be the forerunner of the Internet caf\'{e}.
Bob Albrecht, a former employee of computer maker Control Data, started PCC
in 1970
as a nonprofit computer center where anyone could walk in and pay
by the hour to use a computer.
Albrecht was disturbed by
the industry's emphasis on selling computers to corporations and
the military.
He wanted computers to be accessible to ordinary people and empower
them to do new things, so he started PCC as a ``drop-in'' facility
where anyone could pay by the hour to use a shared computer.
%% As John Markoff recounts in \emph{What the Dormouse Said}, the
%% ``share alike'' mentality of counterculture 1970s California was
%% deeply ingrained in the PCC's mission.
%% and (later)
%% a remote Hewlett-Packard
%% mainframe computer on which HP had donated time.
By this time BASIC's popularity had spread, so many PCC users wrote
(and used) small-business programs and games written in BASIC on the
disruptively-priced DEC~PDP-8 minicomputer available there.

%% A frequent visitor was Ted Nelson, an early proponent and visionary
 %% the idea of linked hypertext.
%% Like Kurtz, Nelson believed computers
%% would be important and everyone should learn about them; the rallying
%% cry of his book \emph{Computer Lib/Dream Machines} was ``You can and
%% you must understand computers now!''

But a much more significant portent of democratization was just around
the corner.
In 1971, the young electronics company Intel (for ``Integrated
Electronics'') had won a contract to design and manufacture chips for
a line of calculators made by a Japanese company.
Engineer Ted Hoff came up with the idea of creating a single
``generic'' chip that could be programmed to work in different
calculator models.
That chip, the Intel~4004, was not powerful enough to be the center of
a true computer; but from the point of view of design, it was the
world's first microprocessor.
Seeing the possibilities, Intel rapidly improved the design to allow
more powerful operations and take advantage of advances in chip
manufacturing.
In 1974 Intel introduced the 8080, the first inexpensive
microprocessor powerful enough to build a credible microcomputer
around.
That microcomputer was the MITS~Altair 8800, announced in 1975 on the
cover of \emph{Popular Electronics}.

\smallpicfigure{figs/PopularElectronics.jpg}{fig:altair}{The Altair 8800
  on the cover of the January
  1975 \emph{Popular Electronics} is actually an empty case, since the
  computer itself was lost in the mail.
}

For \$397 (\$1,800 in 2018), you received a bag of parts, a
circuit board, and a schematic.  If you could solder and assemble it properly, you 
could build a computer for your own use that ran nearly half as fast as the shared
computers at PCC.
(For \$100 more, you could
order the computer pre-assembled, though the waiting time was nearly a year since
there were few MITS employees available to do the soldering by hand.)


Non-hobbyists may have had to squint hard to see the Altair 8800 as an
actual personal computer.
Input was via front-panel switches for painfully entering a single
machine language instruction at a time; output was a row of LEDs for
displaying the binary value of a single memory location.
The circuit board's connectors, similar to the ``expansion slots'' PCs
would have later, accommodated expensive adapters (not included) that
let you plug in a keyboard or tape drive (also not included).
But Bob Albrecht immediately saw the potential of the Altair, the
first computer based on a low-cost mass-produced microprocessor rather
than on a custom design, to radically democratize computing.
In possibly the earliest example of open source software, he convinced
colleague Dennis Allison, an idealistic Stanford computer science
lecturer, to propose a specification for a ``lite'' BASIC that could
be implemented on emerging personal computers like the Altair.
In 1975, Allison published a design proposal in the People's Computer
Company newsletter for ``Tiny BASIC''~\cite{allison_tiny_basic}, which
defined
a subset of BASIC compact enough to be implemented on emerging PCs.
Allison invited readers to submit implementations of his design, 
emphasizing that the design was not proprietary and a major goal was to
share successful implementations with others.  As we will see shortly,
Allison was motivated in part by the fact that Bill Gates's new
company had created a BASIC that would work on PCs, but was charging
\$150 for it (about \$680 in 2018).

\smallpicfigure{figs/drdobbs_logo.png}{fig:drdobbs_logo}{
Typesetter and paste-up artist Eric
Balakinsky was told to create a logo for a magazine about Tiny BASIC,
which was described to him as ``an exercise in computer
programming\ldots{}that doesn't use very many bytes of memory.''
Balakinsky whimsically captured this as ``running light without
overbyte,'' and combining Dennis and Bob into ``Dobb,'' coined
the title \emph{Dr.~Dobb's Journal of Tiny BASIC Calisthenics and
Orthodontia}~\cite[p.~265]{dormouse}.
}

The response from readers was so overwhelming that Albrecht and
Allison decided to spin off a new magazine devoted to free and
low-cost software for the home computer, even though at the time there
was no home computer industry to speak of.
The quirky \w{Dr. Dobb's Journal} eventually became a full-fledged
monthly magazine for PC enthusiasts that would continue in print until
2009 and still circulates online.
By mid-1976, Dr. Dobb's had published Tiny BASIC implementations
for the Motorola 6800, MOS Technology 6502, and Intel 8080
microprocessors.
The 8080 version, called \w{Palo Alto Tiny BASIC} and created by computer engineer
and Homebrew Computer Club member Li-Chen Wang, would become one of the
most influential: the wildly successful TRS-80 Model~I from Radio Shack---one of the first 
fully-assembled, ready-to-run
personal computers under \$1000---came with a direct descendant of
Palo Alto Tiny BASIC built in, and the 
``BASIC Programming'' cartridge for the Bally Astrocade
home video game console, the first time programming
was conceptualized as a video game,
would also use a version of Palo Alto Tiny
BASIC.
It may also be the first time
the phrase ``Copyleft, All Wrongs Reserved'' appeared in print (in the
comments at the top of the listing), to emphasize that the
implementation was intended to be freely shared.
As John Markoff recounts in his book \emph{What the Dormouse Said}~\cite{dormouse},
the ``share-alike'' ethos of the California counterculture strongly
influenced the early development of the technical community, which
consisted of many of the same people; that ethos is arguably the
basis of Open Source today, a phenomenon that has yet to be replicated
in non-computing fields.
Fittingly, in 1985 Dr. Dobb's would publish the \w{GNU Manifesto},
programmer-activist Richard Stallman's call to action to support the
cause of free software and institutionalize the term ``Copyleft.''

But the ``share-alike'' counterculture exemplified by Albrecht and
Allison was about to collide with a much more business-centric culture
exemplified by two ambitious hackers from Seattle.




