
\section{PALO ALTO, 1970 - COMPUTING FOR EVERYONE}

%  what did 'universal access' mean in the culture of 1970 CA?

Fast forward to 1970 California, when the first personal computers (Altair,
Mark-8) were becoming available.  Bob Albrecht, a refugee from computer
maker Control
Data, had become uneasy with the industry's emphasis on computers for
corporations rather than people.  He 
started the People's Computer Company in Menlo Park, a nonprofit walk-in
computer center where anyone could walk in and pay per hour to use
terminals connected to a PDP-8 and (later) a remote Hewlett-Packard
mainframe computer on which 
HP had donated time.  People wrote and ran small-business programs, played games
(mostly written in BASIC), and so on.
A frequent visitor was Ted Nelson, an early proponent and visionary of
the idea of linked hypertext.  Like Kurtz, Nelson believed computers
would be important and everyone should learn about them; the rallying
cry of his book \emph{Computer Lib/Dream Machines} was ``You can and
you must understand computers now!''

Albrecht convinced colleague Dennis Allison, an idealistic Stanford
lecturer, to create an \w{open source} ``\w{Tiny BASIC}'' that would run
on the emerging PCs, most with less than 4KB RAM, so that programming
might be accessible to all.  (Albrecht would later help co-author the introductory programming manual
that came with the breakthrough \w{Commodore VIC-20}
computer~\cite{commodore}.) 
Dennis and Bob started the newsletter \emph{Dr. Dobb's Journal of Tiny BASIC
Calisthenics and Orthodontia} (``Dobb'' is a contraction of their names)
as a sister publication to the People's Computer Company newsletter to
publish information about Tiny BASIC.
By overwhelming reader demand, the quirky newsletter was eventually
turned into a full-fledged monthly publication for computer enthusiasts
named simply \w{Dr. Dobb's Journal}, which continued in print
until 2009 and still circulates online.

Palo Alto Tiny BASIC and the TRS-80 Model I

\begin{milestone}{Open source}

\end{milestone}
