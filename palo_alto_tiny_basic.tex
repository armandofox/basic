
\section{MENLO PARK, 1970 - PEACE, LOVE, AND COMPUTERS}

\makequotation{Let us stand on each others' shoulders, not on each
  others' toes.}{Dennis Allison, Stanford University, 1975}

%  what did 'universal access' mean in the culture of 1970 CA?



Fast forward to 1970s California, when the first personal computers
(Altair, Mark-8) were becoming available.
By this time, BASIC's popularity as a beginner language had spread,
and most computer manufacturers offered some version of BASIC on their
computers.
Bob Albrecht, a refugee from computer maker Control Data and from
computer companies that had worked on defense contracts, had become
uneasy with the industry's emphasis on computers for corporations and
the military rather than for people.
Albrecht started the People's Computer Company (PCC) in Menlo Park, a
nonprofit walk-in computer center where anyone could walk in and pay
a modest fee per hour to use terminals connected to a \w{PDP-8} computer donated by
\w{Digital Equipment Corporation} and (later) a remote Hewlett-Packard
mainframe computer on which HP had donated time.
People wrote small-business programs and games, mostly in BASIC, which
was easy to learn and widely available.
As John Markoff has recounted in \emph{What the Dormouse Said}, the
``share alike'' mentality of counterculture 1970s California was
deeply ingrained in the PCC's mission.
(Albrecht would later help co-author the introductory programming manual
that came with the breakthrough \w{Commodore VIC-20}
computer~\cite{commodore}.  At \$300, not only did it outsell all other
computers when introduced in 1982 (over 1 million units), but it also
stimulated the sales of the first under-\$100 modem, which came bundled
with free trials of various dialup services including CompuServe and in
1982 accounted for the largest traffic on that network.)

%% A frequent visitor was Ted Nelson, an early proponent and visionary of
%% the idea of linked hypertext.  Like Kurtz, Nelson believed computers
%% would be important and everyone should learn about them; the rallying
%% cry of his book \emph{Computer Lib/Dream Machines} was ``You can and
%% you must understand computers now!''

The announcement of the Altair 8800 in 1975 as a ``personal'' computer
seemed to Albrecht to present a new opportunity to democratize computing
power for individuals.
In possibly the earliest example of \w{open source} software, Albrecht
convinced colleague Dennis Allison, an idealistic Stanford computer
science lecturer, to come up with a spec for a ``lite'' BASIC that could
be implemented on emerging personal
computers like the Altair and given
away for free.
In 1975, Allison published a design proposal in the People's Computer
Company newsletter for ``Tiny
BASIC''~\cite{allison_tiny_basic}, which defined the Backus-Naur grammar
for a subset of BASIC with no string support and integer arithmetic only.
Allison invited readers to submit implementations of his design and
emphasized that the design was not proprietary and a major goal was to
share successful implementations with others.

The first version of Tiny BASIC was created by two Texas programmers,
but the overall response was so overwhelming that Albrecht and Allison
decided to spin off a new magazine devoted to free and low-cost software
for the home computer, even though at the time there was no home
computer industry to speak of.
Albrecht, Allison, and the PCC staff explained to resident typesetter
and paste-up artist Eric Balakinsky that Tiny BASIC was ``an exercise in
computer programming\ldots{}that doesn't use very many bytes of
memory.''
Balakinsky whimsically captured the ``exercise'' concept as
``calisthenics'' and ``doesn't use very many bytes'' as ``orthodontia''
(what you do to avoid overbite), and combining Dennis and Bob into
``Dobb,'' coined the title \emph{Dr.~Dobb's Journal of Tiny BASIC
Calisthenics and Orthodontia}~\cite[p.~265]{dormouse}.
While it began as a vehicle for discussing Tiny~BASIC, by overwhelming
reader demand the quirky \w{Dr. Dobb's Journal} eventually became a
full-fledged monthly magazine for personal computer enthusiasts that
would continue in print until 2009 and still circulates online.
By mid-1976, Dr. Dobb's had published Tiny BASIC interpreter listings
for the Motorola 6800, MOS Technology 6502, and Intel 8080.
The 8080
version, called Palo Alto Tiny BASIC and created by computer engineer
and Homebrew Computer Club member Li-Chen Wang, may be the first time
the phrase ``Copyleft, All Wrongs Reserved'' appeared in print (in the
comments at the top of the listing).

\begin{tangent}
Fittingly, in 1985 Dr. Dobb's would publish the \w{GNU Manifesto},
programmer-activist Richard Stallman's call to action to support the
cause of free software and institutionalize the term ``Copyleft.''
\end{tangent}

Palo Alto Tiny BASIC became one of the most influential implementations.
The TRS-80 Model~I from Radio Shack (1977)---one of the first
fully-functional personal computers under \$1000 that worked
out-of-the-box without assembly or tinkering, and which wildly outsold
Radio Shack's conservative estimates---shipped with a version
of BASIC that is a direct descendant of it.
The ``BASIC Programming'' cartridge available for the Bally Astrocade
home video game console (1977) also used a version of Palo Alto Tiny
BASIC, arguably the first time any programming language was featured on
a game console.

Although these ready-to-run microcomputers were still a couple of years in
the future, two hackers from Seattle were able to see around that
corner.
