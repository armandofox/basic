
\section{MENLO PARK, 1970 - PEACE, LOVE, AND COMPUTERS}

\makequotation{Let us stand on each others' shoulders, not on each
  others' toes.}{Dennis Allison, Stanford University, 1975}

%  what did 'universal access' mean in the culture of 1970 CA?

By the early 1970s, BASIC's popularity as a beginner language had spread,
and most mainframe computer manufacturers offered some version of BASIC on their
computers.
Bob Albrecht, a former employee of computer maker Control Data and other
computer companies that had worked on defense contracts, had started the
People's Computer Company (PCC), a 
nonprofit walk-in computer center in Menlo Park (now at the heart of
California's Silicon Valley).  PCC was a forerunner of today's cyber caf\'{e}s:
Anyone could walk in and pay
a modest hourly fee to use terminals connected to a DEC~PDP-8, and (later)
a remote Hewlett-Packard
mainframe computer on which HP had donated time.
People wrote (and used) small-business programs and games, mostly in
BASIC, which 
was easy to learn and widely available.
As John Markoff has recounted in \emph{What the Dormouse Said}, the
``share alike'' mentality of counterculture 1970s California was
deeply ingrained in the PCC's mission.

%% A frequent visitor was Ted Nelson, an early proponent and visionary of
%% the idea of linked hypertext.  Like Kurtz, Nelson believed computers
%% would be important and everyone should learn about them; the rallying
%% cry of his book \emph{Computer Lib/Dream Machines} was ``You can and
%% you must understand computers now!''

In 1974 Intel introduced the 8080, the first inexpensive microprocessor powerful
enough to build a credible microcomputer around.  The MITS~Altair 8800,
announced in 1975 on the cover of \emph{Popular Electronics}, was
that computer.
For \$397 (\$1,914 in 2015), you got a bag of parts, a
circuit board, and a schematic, and if you could assemble it properly, you 
had a computer whose performance was nearly half that of a
\$500,000 IBM minicomputer~\cite{need_citation}.
%% for \$100 more (\$482 more in 2015) you could
%% order it pre-assembled, though the waiting time was nearly a year since
%% there were few MITS employees available to do the soldering and
%% debugging.  
Input was via
front-panel switches for painfully entering a
single machine language instruction at a time; output was a row of  LEDs for
displaying the binary value of a single memory location.  The circuit board's
edge connectors, similar to the later PC's ISA slots, accommodated 
peripheral adapters if you wanted to plug in a keyboard or tape drive.

\smallpicfigure{figs/PopularElectronics.jpg}{fig:altair}{The Altair 8800
  on the cover of the January
  1975 \emph{Popular Electronics} is actually an empty case, since the
  computer itself was lost in the mail.
}

Non-hobbyists may have had to squint
hard to see this as an actual personal computer,
but Albrecht, who had started PCC as a reaction to
the industry's emphasis on computers for corporations and
the military rather than for people, 
immediately saw that the Altair could  further democratize computing.
In possibly the earliest example of open source software, he
convinced colleague Dennis Allison, an idealistic Stanford computer
science lecturer, to design a spec for a ``lite'' BASIC that could
be implemented on emerging personal
computers like the Altair and given
away for free.
In 1975, Allison published a design proposal in the People's Computer
Company newsletter for ``Tiny
BASIC''~\cite{allison_tiny_basic}, which defined the Backus-Naur grammar
for a subset of BASIC with no string support and integer arithmetic only.
Allison invited readers to submit implementations of his design and
emphasized that the design was not proprietary and a major goal was to
share successful implementations with others.

The overall response was so overwhelming that Albrecht and Allison
decided to spin off a new magazine devoted to free and low-cost software
for the home computer, even though at the time there was no home
computer industry to speak of.
They explained to resident typesetter
and paste-up artist Eric Balakinsky that Tiny BASIC was ``an exercise in
computer programming\ldots{}that doesn't use very many bytes of
memory.''
Balakinsky whimsically captured ``exercise'' as
``calisthenics'' and ``doesn't use very many bytes'' as ``orthodontia''
(what you get to avoid overbite), and combining Dennis and Bob into
``Dobb,'' coined the title \emph{Dr.~Dobb's Journal of Tiny BASIC
Calisthenics and Orthodontia}~\cite[p.~265]{dormouse}.
The quirky \w{Dr. Dobb's Journal} eventually became a
full-fledged monthly magazine for PC enthusiasts that
would continue in print until 2009 and still circulates online.
By mid-1976, Dr. Dobb's had published Tiny BASIC interpreter listings
for the Motorola 6800, MOS Technology 6502, and Intel 8080.
The 8080
version, called Palo Alto Tiny BASIC and created by computer engineer
and Homebrew Computer Club member Li-Chen Wang, may be the first time
the phrase ``Copyleft, All Wrongs Reserved'' appeared in print (in the
comments at the top of the listing).

\begin{tangent}
Fittingly, in 1985 Dr. Dobb's would publish the \w{GNU Manifesto},
programmer-activist Richard Stallman's call to action to support the
cause of free software and institutionalize the term ``Copyleft.''
\end{tangent}

Palo Alto Tiny BASIC was destined to become one of the most influential
implementations. 
When the wildly successful TRS-80 Model~I from Radio Shack was
introduced in 1977---one of the first 
fully-functional, works-out-of-the-box  
personal computers under \$1000---its ROM carried a direct descendant of
Palo Alto Tiny BASIC.
The ``BASIC Programming'' cartridge available for the Bally Astrocade
home video game console (1977), the first time that the act of programming
was sold as a video ``game,''
also used a version of Palo Alto Tiny
BASIC.

But in 1975, when the Altair was announced, ready-to-run PCs sold in
retail stores were still a couple of years in the future.
Fatefully, two hackers from Seattle were able to see around that corner.


