
\makequotation{It is practically impossible to teach good programming to
  students that have had a prior exposure to BASIC: as potential
  programmers they are mentally mutilated beyond hope of
  regeneration.}{Edsger W. Dijkstra, Turing Award winner~\cite{dijkstra-truths}}

The invention of the PC, which democratized computing, owes its success
to two important inventions, one in hardware and one in software.
The hardware invention arrived in 1971, when engineer Ted Hoff and others
at Intel invented the 4004 microprocessor, the distant ancestor of
the heart of virtually all modern PCs.  The microprocessor made it
economically practical to build a small computer at a consumer-friendly price
point, and the sales volume at that price point lowered the price even
further, resulting in an unprecedented positive feedback cycle in
which computers become faster yet cheaper year after year.
The software invention was the creation and rapid dissemination of the
high-level introductory programming language BASIC, which made that
computer power accessible to beginners and launched a software revolution.

Before universities had computers that were widely accessible to students, BASIC
provided a kinder, gentler introduction to computing suitable for
everyone, especially nontechnical majors.
Before there was such a thing as a PC software industry, 
BASIC allowed early hobbyists to quickly get started writing their own
programs.
Before computer games became a multibillion-dollar industry, beginning
programmers and hobbyists were writing their own games in BASIC, to
exploit the rapidly-developing capabilities of graphics and sound on
early PCs. 
Much of the other software that fueled the PC revolution---online
bulletin boards, simple utilities, small-business software---was written
in BASIC. 
As science fiction author David Brin has written~\cite{why_johnny_cant_code},
despite its flaws, BASIC
was sufficiently nonthreatening to introduce an entire generation of
newbies to the joy of programming.

Today's world-class universities boast that over 90\% of all
undergraduates are exposed to introductory programming, and movements
like ``CS for all'' and sites like \T{code.org} aim to expose everyone
to coding.
Dartmouth College, where BASIC was invented, had largely
achieved this goal by 1971 on its campus~\cite{man_and_computer} by implementing the
farsighted vision of BASIC's creators.
A large supporting cast of characters---idealistic professors,
ambitious entrepreneurs, garage hobbyists, socially-conscious
professionals, and the companies and institutions where they
worked---introduced  an entire generation
of hobbyist programmers to the beauty and joy of computing, all using BASIC.

Yet despite its pivotal role in the PC revolution, BASIC is probably
one of the most maligned programming languages to achieve widespread
use, as quotes like Dijkstra's suggest.
It's time the true story of BASIC's influence was told.

