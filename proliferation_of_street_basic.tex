
\section{"STREET BASIC" BECOMES A MUST-HAVE FEATURE}

\makequotation{I think it's fair to say that more persons in the world know how to
write simple programs in BASIC than in any other language. It is true
that most of them are probably still unable to vote or buy a drink.  And
if FORTRAN is the lingua franca, then certainly it must be true that
BASIC is the lingua playpen.}{Thomas E. Kurtz, co-creator of BASIC, in
  1981~\cite{hopl}} 

In the 1970s there was no third-party software industry as we now know
it; with a few exceptions such as office-management software,
PC owners would have to develop their own software, and BASIC was the
simplest path, as it had been designed to be easy to learn.  In a sense,
BASIC was to the first couple of generations of PCs what binary
compatibility became later: if your computer could run BASIC programs,
you had access to a growing library of programs, even if it required
manually addressing the various incompatibilities across interpreters.
Indeed, when Commodore developed the Commodore 64, which would become
the best-selling 
PC of all time, Commodore gave it a BASIC adapted from
their earlier VIC-20 so that people could take advantage of its BASIC program
library, especially educational software for in-classroom
use~\cite[p. 414]{commodore}. 

Vendors soon realized that a computer without BASIC was a
competitive liability.  Sphere Computer and Processor Technology
boasted low-cost computers with BASIC, but the BASIC was
vaporware at the time of introduction.  Processor Technology 
contracted their BASIC out to NorthStar, which also licensed it to
others and ultimately undersold PT with its own computer,
and Sphere's BASIC underperformed when finally
released~\cite[p. 114, 134]{veit}.  Both companies quickly went under.

Commodore Computer, whose ruthless chairman Jack Tramiel was obsessed
with low prices, bought MOS Technology outright in order to have a
guaranteed in-house source of 6502 chips for its PET, VIC-20, and Commodore
64 computers.
Peddle had long envisioned an inexpensive personal computer based on
the 6502, and when MOS
Technology was acquired by Commodore, he got his chance, helping to
design the Commodore PET.
Peddle knew that since people would have to create their own software,
the computer would have a competitive advantage if it shipped with a
programming language built right in.  At the time, evne floppy disk
drives were too expensive to ship with PCs, and portable hard disks
weren't even on the horizon, so the only practical way to include an operating
system and programming language along with the computer would be to burn
that software into read-only memory
(ROM) chips soldered onto the computer's circuit board.

Fortunately, MOS Technology, which had been recently acquired by
Commodore, was able to produce ROM chips at low cost, albeit with very
limited capacity.
Peddle and hard-driving Commodore founder Jack Tramiel negotiated a
deal with the nascent Micro-Soft:
for a one-time license fee of \$25,000,
Micro-Soft would adapt their BASIC to work with the 6502 and
fit into MOS Technology's limited-capacity ROM chips, 
and allow Commodore to package it with every computer.
The wording of the deal ultimately allowed Commodore to package the
BASIC with future models, including the Commodore 64, which sold more units than
any other PC~\cite{commodore}.
The Commodore
PET thereby established the pattern that all PC makers would follow:
BASIC is built right into the hardware, and when you turn on the
machine out of the box, even with no disk or cassette storage attached, you're
immediately dropped into BASIC and you can start programming.

This arrangement allowed Commodore to continue an arrangement that was a
cornerstone of spendthrift CEO Jack Tramiel's cost-cutting strategy:
vertical integration.
With the Microsoft one-time license in hand, Commodore could produce
computers without relying on external vendors for either hardware or
software (recall that Altair relied on Intel for the 8080
microprocessor, other companies for ROM and memory boards, and Microsoft
for BASIC), thereby streamlining their manufacturing and tightly
controlling costs.

BASIC soon became an important selling point for PCs.  Stan Veit,
longtime editor of \emph{Computer Shopper}, reports 
that when Steve Jobs demonstrated the prototype Apple~I at an ACM (New
York chapter) dinner meeting in 1976, his colleagues were
astonished---but the dealers' reaction was ``please call when BASIC is
available.''  Jobs said Woz was ``working hard'' on an
interpreter~\cite[pp. 92ff]{veit}.
On August 26, 1976, at the first national computer show in Atlantic City,
NJ, Woz was in the hotel room using the hotel room's TV to finish work
on BASIC; it worked on their floor models when the show opened on August
28.  It was the only 6502 computer on a floor full of 8080s

When the Apple ][ was introduced in 1977, its BASIC, created in a hurry
by Wozniak based on Hewlett-Packard BASIC (recall Woz was an ex-HP
employee), was inferior to the Micro-Soft BASIC used by the PET.
When Mike Markkula was hired as CEO of Apple, he quickly negotiated a
license with Micro-Soft [sic] to adapt their BASIC to the new Apple ][ Plus,
paying double what Commodore had paid~\cite[p. 114]{commodore}.
That same year, Radio Shack made a deal with Gates to provide
Micro-Soft BASIC as an upgrade feature to its TRS-80 Model I (which
had originally shipped with a variant of the free Palo Alto Tiny
BASIC), with Gates again doubling the price that Apple had paid.
Micro-Soft BASIC was now the dominant computer language on the first
three major PCs---the Commodore PET, TRS-80 Model I, and Apple
][, all released in 1977.

\begin{figure}
  \begin{tabular}{|l|l|l|l|l|}
  \hline
  Computer \& company 
  & Released 
  & Price (US) 
  & BASIC in ROM
  \\
  Apple II
  & June 10, 1977 
  & \$1298 
  & Wozniak's Integer BASIC 
  \\
  Apple II Plus \& later
  & June 10, 1977 
  & \$1298 
  & ``Applesoft'' based on Microsoft BASIC
  \\
  Radio Shack TRS-80 Model I
  & August 3, 1977 
  &   \$599 
  & ``Level I'' BASIC based on Palo Alto Tiny BASIC
  \\
  Radio Shack TRS-80 Model I with expansion unit/Model II
  & 
  & 
  & ``Level II'' BASIC by Microsoft
  \\
  Commodore PET
  & October 1977 
  & \$699? 
  & Commodore BASIC by Microsoft
  \\
  \end{tabular}
  \caption{\label{fig:timeline} The first crop of ``ready-to-run'' PCs
    all featured built-in BASIC.  All the BASICs in the rightmost column
    are ports/adaptations of Microsoft BASIC.}
\end{figure}

In the two years since the Altair's release in 1975, PCs had evolved 
features such as simple graphics
and sound and the ability to plug in devices like joysticks, but the
features worked differently on each system.  The incompatibility was not
the result of capriciousness: PC designers were trying to keep prices
low while providing more features to entice buyers, and the tension
between the two often required them to do technological headstands to
implement a feature cheaply enough to keep their prices competitive.

So BASIC had to be modified differently for each computer in order to
allow programmers to exploit those different features: the graphics
features added to Applesoft BASIC were different from those in TRS-80
BASIC.
  The
``dialects'' of BASIC proliferated and, like species gradually becoming
distinct and unable to mate, programs written in one dialect would soon
fail to run without modification on others.  Depending on the hardware
capabilities of the computers, the changes might be substantial or
impossible,
despite being authored in ``the same'' language.

\begin{tangent}
The best analogy today is probably libraries.  The C programming
language is fairly consistent across environments, but the libraries
available, especially for supporting special hardware features, are
quite different.  Consequently there is no expectation of portability
for a C program 
that relies on machine-specific specialized libraries.
\end{tangent}

BASIC's creators were dismayed at the proliferation of incompatible
dialects and the addition of \emph{ad~hoc} features that (in their
opinion) were
often implemented in poor technical taste and contrary to the original
spirit and goals of 
the language.  They began using the derogatory term ``street BASIC'' to
distinguish these mass-market dialects from the original language they
had been curating and using at Dartmouth.

Yet for all the criticism of ``street BASIC,'' its user
experience was quite close to the ``appliance'' view that BASIC's
creators originally intended:
each time a user turned on his PC, the first thing he'd see was the
familiar BASIC prompt, allowing him to jump right in and start
programming rather than navigating an operating system or other details.
Just as the creators intended, many of these users came to assume that
BASIC \emph{was} the ``operating 
system'' of their computer because it was the first thing they saw
each time they turned on the machine.
