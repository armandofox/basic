
\section{"STREET BASIC" BECOMES A MUST-HAVE FEATURE}

In the 1970s there was no third-party software industry as we now know
it; PC users would have to develop their own software, and BASIC was the
simplest path, as it had been designed to be easy to learn.  In a sense,
BASIC was to the first couple of generations of PCs what binary
compatibility became later: if your computer could run BASIC programs,
there was access to a growing library of programs, even if it required
manually addressing the various incompatibilities across interpreters.
Indeed, when Commodore came out with the Commodore 64, the best-selling
PC of all time, part of their strategy was to use the same ROM BASIC as
the VIC-20 so that people could take advantage of its BASIC program
library, especially educational software for in-classroom
use~\cite[p. 414]{commodore}. 

Vendors soon realized that whereas the presence  BASIC was previously a
differentiator, its absence was rapidly becoming a 
competitive liability.  Sphere Computer and Processor Technology
boasted low-cost computers with BASIC, but the BASIC was
vaporware at the time of introduction.  Processor Technology 
contracted their BASIC out to NorthStar, which also licensed it to
others and ultimately undersold PT with its own computer,
and Sphere's BASIC underperformed when finally
released~\cite[p. 114, 134]{veit}.  Both companies quickly went under.


Chuck Peddle had long envisioned an inexpensive personal computer based on
the 6502, and when MOS
Technology was acquired by Commodore, he got his chance, helping to
design the Commodore PET.
Peddle knew that since people would have to create their own software,
the computer would have a competitive advantage if it shipped with a
programming language built right in.  At the time, evne floppy disk
drives were too expensive to ship with PCs, and portable hard disks
weren't even on the horizon, so the only practical way to include an operating
system and programming language along with the computer would be to burn
that software into read-only memory
(ROM) chips soldered onto the computer's circuit board.

Fortunately, MOS Technology, which had been recently acquired by
Commodore, was able to produce ROM chips at low cost but with very
limited capacity.
Peddle and hard-driving Commodore founder Jack Tramiel negotiated a
deal with the nascent company Micro-Soft [sic], which had produced a
compact BASIC interpreter for the Altair 8800,
to adapt their BASIC to work with the 6502 
and make it small enough that it could fit into ROM chips.

The 
PET therefore established the pattern that all PC makers would follow:
BASIC is built right into the hardware, and when you turn on the
machine out of the box, even with no disk or cassette storage attached, you're
immediately dropped into BASIC and you can start programming.

The PET was announced in 1977 and was the first PC for the mass market,
selling for \$700.
Commodore paid a one-time fee of \$25,000 for Micro-Soft's BASIC, and
their maneuvering ultimately allowed them to use it in every computer
they ever sold, including the Commodore 64, which sold more units than
any other PC~\cite{commodore}.

BASIC soon became an important selling point for PCs.  Stan Veit,
longtime editor of \emph{Computer Shopper}, reports 
that when Steve Jobs demonstrated the prototype Apple~I at an ACM (New
York chapter) dinner meeting in 1976, his colleagues were
astonished---but the dealers' reaction was ``please call when BASIC is
available.''  Jobs said Woz was ``working hard'' on an
interpreter~\cite[pp. 92ff]{veit}.
On August 26, 1976, at the first national computer show in Atlantic City,
NJ, Woz was in the hotel room using the hotel room's TV to finish work
on BASIC; it worked on their floor models when the show opened on August
28.  It was the only 6502 computer on a floor full of 8080s

When the Apple ][
was introduced later in 1977, its BASIC was inferior to the 
Micro-Soft BASIC used by the PET: it could only handle whole numbers and
was missing some important functions.  When Mike Markkula became
president of Apple, one of the first things he did was negotiate a
license with Micro-Soft to put their BASIC into the new Apple ][ Plus,
paying double what Commodore had paid~\cite[p. 114]{commodore}.  That
same year, Radio Shack introduced their PC, the TRS-80 Model~I.  Its
built-in BASIC was essentially the free but very limited
Tiny BASIC that Dennis Allison
had developed in California; Radio Shack made a deal with Gates to
provide Micro-Soft BASIC as an upgrade feature, giving Micro-Soft BASIC
placement on the first three major PCs, all released in 1977.


Ironically, for all the authors' critiques of ``street BASIC,'' its user
experience was quite close to the ``appliance'' view: starting with the
Apple II Plus (?check this; maybe TRS-80 model I?), most home computers
didn't have hard disks or at first even floppies to boot from, so they
``booted'' into a BASIC interpreter in ROM.  The first thing the user
would see was a BASIC prompt. 


What it was like
  - ROM BASIC vs "true" OS.  TO most of us who grew up on BASIC, it
  *was* the OS, shell, ...
  - BASIC and I/O - OS facilities: an uneasy marriage.  Compare GWBASIC,
  TRSBASIC level 2, Applesoft/Integer BASIC + DOS

What had changed by the time BASIC was ported to PCs?
-  Graphics (albeit all mutually incompatible)
-  Need for combining BASIC and machine language
-  needed to work with memory-mapped IO (joysticks, etc)

Not fast enough as performance language, but primitive 'scripting
language' if underlying runtime system allowed access to 'interesting'
behaviors (eg graphics); esp prevalent in late PCs (VIC, C64), Visual
BASIC (scripting MS Office), and finally VB.net


BASIC and graphics: a way to do entertainment and games
  - ZX80, TRS-80: graphic character glyphs, plus most BASICs provided "gotoXY"
  - Apple II: lo-res graphics mode repurposes text glyphs
  - Apple II: weirdly memory-mapped hires graphics library; shape tables
  - VIC-20, C64
  - Famous graphics demos in BASIC: Fire Organ, moire pattern, breakout
  (Woz added graphics and PDL() commands to IntBasic just to be able to
  recreate Breakout)

As popular press vehicle for computing
  - "BASIC games" books
  - type-in listings

