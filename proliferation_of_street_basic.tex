
\section{"STREET BASIC" BECOMES A MUST-HAVE FEATURE}

\makequotation{I think it's fair to say that more persons in the world know how to
write simple programs in BASIC than in any other language. It is true
that most of them are probably still unable to vote or buy a drink.  And
if FORTRAN is the lingua franca, then certainly it must be true that
BASIC is the lingua playpen.}{Thomas E. Kurtz, co-creator of BASIC, in
  1981~\cite{hopl}} 

In the 1970s there was no third-party software industry as we now know
it; with a few exceptions such as office-management software, PC owners
would have to develop their own software, and BASIC was the simplest
path, as it had been designed to be easy to learn.
In a sense, BASIC was to the first couple of generations of PCs what
binary compatibility became later: if your computer could run BASIC
programs, you had access to a growing library of programs, even if it
required manually addressing the various incompatibilities across
interpreters.

As Peddle had foreseen, PC vendors soon realized that a computer without
built-in BASIC was a competitive liability.
Sphere Computer and Processor Technology boasted low-cost computers
whose BASIC was vaporware at the time of introduction and underperformed
when finally released~\cite[p. 114, 134]{veit}; both companies quickly
went under.
When Steve Jobs demonstrated the prototype Apple~I at an ACM (New York
chapter) dinner meeting in 1976, his colleagues were astonished at the
hardware, but dealers' reactions amounted to ``please call when BASIC is
available.''
Jobs said Woz was ``working hard'' on BASIC~\cite[pp. 92ff]{veit}, which
he was---on August 26, 1976, at the first national computer show in
Atlantic City, NJ, Woz was using his hotel room's TV to finish work on
Apple Integer BASIC in time for the show's opening on August 28.
The trend was clear: users expected to turn on their PCs and
start programming in BASIC, courtesy of a ROM-based interpreter.

Microsoft was perfectly positioned to take advantage of this trend.
Woz's Apple BASIC was markedly inferior to Micro-Soft's, and when Mike
Markkula joined Apple in 1977 as an investor and employee, he
quickly negotiated a deal with Gates to port Micro-Soft BASIC to
the Apple ][, paying double what Commodore had
paid~\cite[p. 114]{commodore}.
That same year, Radio Shack made a deal with Gates to provide Micro-Soft
BASIC as an upgrade feature to its TRS-80 Model I, whose entry-level BASIC
was a direct descendant of the free Palo Alto Tiny BASIC, with Gates
again doubling the price that Apple had paid.
Micro-Soft BASIC was now the dominant computer language on the first
three major PCs---the Commodore PET, TRS-80 Model I, and Apple ][, all
released in 1977.

Later, when Commodore developed the popular VIC-20 (the first
full-featured PC for under \$300) and the phenomenally successful
Commodore~64, which would become the best-selling PC of all time and the
first to sell a million units, the wording of the deal between Commodore
and Microsoft allowed Commodore to tweak the original PET BASIC and
include it in every VIC-20 and Commodore 64 ROM without any further
payments to Microsoft.
This infuriated Gates, but it did allow VIC-20 and Commodore~64 buyers
to take advantage of the many BASIC programs that had been written for
the PET, especially educational and in-classroom
software~\cite[p. 414]{commodore}.

\tablefigure{figs/basics.tex}{fig:basics}{The first crop of ``ready-to-run'' PCs
    all featured BASIC in ROM.  Woz's Integer BASIC was written from
    scratch; TRS-80 Level~I BASIC was based on Palo Alto Tiny BASIC; the
    rest are ports of Microsoft BASIC.}

In the two years since the Altair's release in 1975, PCs had evolved 
features such as simple graphics
and sound and the ability to plug in devices like joysticks, but the
features worked differently on each system.  The incompatibility was not
the result of capriciousness: PC designers were trying to keep prices
low while providing more features to entice buyers, and the tension
between the two often required them to do technological headstands to
implement a feature cheaply enough to keep their prices competitive.

So BASIC had to be modified differently for each computer in order to
allow programmers to exploit those different features: the graphics
features added to Applesoft BASIC were different from those in TRS-80
BASIC.
  The
``dialects'' of BASIC proliferated and, like species gradually becoming
distinct and unable to mate, programs written in one dialect would soon
fail to run without modification on others.  Depending on the hardware
capabilities of the computers, the changes might be substantial or
impossible,
despite being authored in ``the same'' language.

\begin{tangent}
The best analogy today is probably libraries.  The C programming
language is fairly consistent across environments, but the libraries
available, especially for supporting special hardware features, are
quite different.  Consequently there is no expectation of portability
for a C program 
that relies on machine-specific specialized libraries.
\end{tangent}

BASIC's creators were dismayed at the proliferation of incompatible
dialects and the addition of \emph{ad~hoc} features that (in their
opinion) were
often implemented in poor technical taste and contrary to the original
spirit and goals of 
the language.  They began using the derogatory term ``street BASIC'' to
distinguish these mass-market dialects from the original language they
had been curating and using at Dartmouth.

Yet for all the criticism of ``street BASIC,'' its user
experience was quite close to the ``appliance'' view that BASIC's
creators originally intended:
each time a user turned on his PC, the first thing he'd see was the
familiar BASIC prompt, allowing him to jump right in and start
programming rather than navigating an operating system or other details.
Just as the creators intended, many of these users came to assume that
BASIC \emph{was} the ``operating 
system'' of their computer because it was the first thing they saw
each time they turned on the machine.
