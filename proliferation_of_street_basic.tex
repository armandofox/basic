
\section{STREET BASIC BECOMES A MUST-HAVE FEATURE}

\makequotation{I think it's fair to say that more persons in the world know how to
write simple programs in BASIC than in any other language. It is true
that most of them are probably still unable to vote or buy a drink.  
%% And
%% if FORTRAN is the lingua franca, then certainly it must be true that
%% BASIC is the lingua playpen.
}{Thomas E. Kurtz, co-creator of BASIC, in
  1981~\cite{hopl}} 

Since in the 1970s there was no third-party PC software industry as we
now know it, the target audience for PCs was assumed to be people who
would learn to code, and BASIC was by design the simplest path.
While the term ``open source'' hadn't been invented in 1964, and BASIC's
source code itself was not ``open'' as we use that term today, its
inventors were clear that anyone was free to implement the language
without requiring a license from Dartmouth.
Indeed, there wasn't even a requirement that implementers adhere to a
spec; ``Minimal BASIC'' wasn't even standardized by the American
National Standards Institute until 1978.

\begin{tangent}{}
This isn't true of some modern languages: for example, to certify that
your implementation of Java is compatible with the reference spec, you
are required to run it through an Oracle ``compatibility kit'' whose
licensing terms impose intellectual property constraints whose effect is
to restrict how you may redistribute your
implementation~\cite{apache-java-letter,apache-resigns-jcp}.
\end{tangent}

Because BASIC was unencumbered and easy to learn, it quickly became
influential not only in colleges 
and universities, but among PC makers. 
Indeed, as Chuck Peddle had foreseen, PC vendors soon realized that a
computer \emph{without} 
built-in BASIC was a competitive liability.
Sphere Computer and Processor Technology boasted low-cost computers
whose BASIC was vaporware at the time of introduction and underperformed
when finally released~\cite[p. 114, 134]{veit}; both companies quickly
went under.
When Steve Jobs demonstrated the prototype Apple~I at an ACM (New York
chapter) dinner meeting in 1976, the technical attendees were astonished at the
hardware, but dealers' reactions amounted to ``call us when BASIC is
available.''
Jobs said Woz was ``working hard'' on BASIC for the follow-on
Apple~II~\cite[pp. 92ff]{veit}, which 
was true---on August 26, 1976, two nights before the first national computer show in
Atlantic City, NJ, Woz was using his hotel room's TV to finish work on
Apple ``Integer BASIC'' in time for the show's opening.

\begin{tangent}
Unlike Gates, Woz knew little about interpreters or compilers, he
couldn't afford an assembler program, and his knowledge of BASIC was
based on a few hours at a high school terminal and his job at
Hewlett-Packard, which made BASIC available for its
mainframes.
Woz was primarily interested in making it easy to program games,
so Apple BASIC had
commands to draw color graphics and read the position of a joystick,
but no floating-point arithmetic or file I/O.
\end{tangent}

\smallpicfigure{figs/breakout.jpg}{fig:breakout}{The \emph{Breakout}
  game implemented in Apple Integer BASIC.  According to Woz
  himself~\cite{woz-basic}, being able to code games like this one,
  which he had designed for Atari a few years before, was his
  main goal for Apple BASIC.}

The trend was clear: users expected to turn on their PCs and
start programming in BASIC, courtesy of a ROM-based interpreter.
And Micro-Soft was perfectly positioned to take advantage of this trend.
Woz's Apple BASIC was markedly inferior to Micro-Soft's, and when
angel investor Mike
Markkula joined Apple in 1977 as the new CEO, he
quickly negotiated a deal with Gates to port Micro-Soft BASIC to
the Apple~II, paying double what Commodore had
paid~\cite[p. 114]{commodore}.

That same year, Radio Shack made a deal with Gates to provide Micro-Soft
BASIC as an upgrade feature to its TRS-80 Model I, whose entry-level BASIC
was a direct descendant of the free Palo Alto Tiny BASIC, with Gates
again doubling the price that Apple had paid.
Micro-Soft BASIC was now the dominant computer language on the first
three major PCs---the Commodore PET, TRS-80 Model I, and Apple II, all
released in 1977.

\tablefigure{figs/basics.tex}{fig:basics}{The first crop of ``ready-to-run'' PCs
    all featured BASIC in ROM.  Woz's Integer BASIC was written from
    scratch; TRS-80 Level~I BASIC was based on Palo Alto Tiny BASIC; the
    rest are ports of Microsoft BASIC.}

If your computer had BASIC, you had access to a growing library of
community-authored programs, even though it usually required laboriously
typing the programs in (since tape and disk formats were incompatible across
different computer systems).
But even though most early PC BASICs were direct descendants of the
original Micro-Soft BASIC, there were differences.
In the two years
since the Altair's release in 1975, PCs had evolved features such as
simple graphics and sound and the ability to plug in devices like
joysticks, but the features worked differently on each system.
The incompatibility was not the result of capriciousness: PC designers
were trying to keep prices low while providing more features to entice
buyers, and the tension between the two often required them to do
technological headstands to implement a feature cheaply enough to keep
their prices competitive.

So BASIC had to be modified differently for each computer in order to
allow programmers to exploit those different features: the graphics
features added to Applesoft BASIC were different from those in TRS-80
BASIC, just as today there's a big distinction between (say)
the C language, which is fairly consistent across environments, vs.\ the
availability of specific C libraries on different platforms.
As PC hardware features became more exotic and ``dialects'' of BASIC
evolved to exploit them, manually tweaking the source code in published program
listings was soon insufficient to allow BASIC programs to run on
different platforms, like species that
gradually become distinct and unable to mate.
In time, over 60
different PC models implemented BASIC in ROM alone, not counting later
disk implementations such as GWBASIC for MS-DOS.

BASIC's creators were dismayed at this proliferation of incompatible
dialects and the addition of \emph{ad~hoc} features that (in their
opinion) were often implemented in poor technical taste and contrary to
the original spirit and goals of the language.
They began using the derogatory term ``street BASIC'' to distinguish
these mass-market dialects from the original language they had been
curating and using at Dartmouth.
But as university educators, they may have been even more dismayed that
the ``killer app'' for BASIC turned out to be not formal computing
education, but games.
