
\section{STREET BASIC BECOMES A MUST-HAVE FEATURE}

\makequotation{I think it's fair to say that more persons in the world know how to
write simple programs in BASIC than in any other language. It is true
that most of them are probably still unable to vote or buy a drink.  
%% And
%% if FORTRAN is the lingua franca, then certainly it must be true that
%% BASIC is the lingua playpen.
}{Thomas E. Kurtz, co-creator of BASIC, in
  1981~\cite{hopl}} 

While the term ``open source'' hadn't been invented in 1964, and BASIC's
original source code itself was not ``open'' as we use that term today, its
inventors made clear that anyone was free to \emph{implement} the language
without a license from Dartmouth.
Perhaps unfortunately, this permissiveness also meant that there was
no standard set of requirements that any particular version of BASIC
had to meet in order to call itself BASIC.


\begin{tangent}{}
In contrast, modern languages are often tightly protected.
For example, Java implementations must pass an Oracle ``compatibility
suite'' whose licensing terms impose intellectual property constraints
that restrict how you may redistribute your
implementation~\cite{apache-java-letter,apache-resigns-jcp}.
 ``Minimal BASIC'' was standardized by the American
National Standards Institute in 1978, but by that time, many variants
of the language had already proliferated and the standard was largely ignored.
\end{tangent}

Nonetheless, precisely because BASIC was legally unencumbered and easy to learn, it quickly became
influential not only in colleges and universities, but among PC
makers.
Just as Chuck Peddle had foreseen, PC vendors soon realized that a
computer \emph{without} a built-in BASIC was a nonstarter.
Sphere Computer and Processor Technology boasted low-cost computers
whose BASIC was vaporware at the time of introduction and
underperformed when finally released~\cite[p. 114, 134]{veit}; both
companies quickly went under.
When Steve Jobs demonstrated the prototype Apple~I at a
professional conference\footnote{Actually a dinner meeting of the New York chapter
of the Association for Computing Machinery, the first and largest
professional society for computing professionals.} in
1976, the engineers in the audience were astonished at the hardware, but
dealers' reactions amounted to ``call us when BASIC is available.''
Jobs said Woz was ``working hard'' on BASIC for the follow-on
Apple~II~\cite[pp. 92ff]{veit}, which was true---on August 26, 1976,
two nights before the first national computer show in Atlantic City,
not far from New York, Woz was using the TV in  his hotel room as a
monitor,  finishing work on Apple
``Integer BASIC'' in time to demonstrate it at the show.

%% \begin{tangent}
%% Unlike Gates, Woz knew little about interpreters or compilers, he
%% couldn't afford an assembler program, and his knowledge of BASIC was
%% based on a few hours at a high school terminal and his job at
%% Hewlett-Packard, whose mainframe computers had BASIC on them.
%% Woz was primarily interested in writing games, so his Apple BASIC
%% included commands to draw color graphics and read the position of a
%% joystick, but no floating-point arithmetic or file I/O.
%% \end{tangent}

\smallpicfigure{figs/breakout.jpg}{fig:breakout}{The \emph{Breakout}
  game implemented in Apple Integer BASIC.  According to Woz
  himself~\cite{woz-basic}, being able to easily create games like this one,
  which he had designed for Atari a few years before, was his
  main goal for Apple BASIC.}

The trend was clear: users expected to turn on their PCs and start
programming immediately using BASIC built right into the ROM,
and Micro-Soft was perfectly positioned to exploit this
trend.
While Gates was a software wizard, Woz was really a hardware specialist, and
his Apple BASIC was inferior to Micro-Soft's.  (Woz later stated that
his main goal for Apple BASIC was to make it easy to write simple
graphics games, as in Figure~\ref{fig:breakout}.)
When investor Mike Markkula joined Apple in 1977
as the new CEO, he quickly negotiated a deal with Gates to adapt
Micro-Soft BASIC to the Apple~II, paying double what Commodore had
paid~\cite[p. 114]{commodore}.
That same year, Radio Shack made a deal to provide Micro-Soft
BASIC as an upgrade for its TRS-80 Model~I, which shipped with 
the free Palo Alto Tiny BASIC,
again doubling the price that Apple had paid.
Micro-Soft BASIC was now the dominant computer language on the first
three major PCs---the Commodore PET, TRS-80 Model~I, and Apple~II, all
released in 1977.

%% \tablefigure{figs/basics.tex}{fig:basics}{The first crop of ``ready-to-run'' PCs
%%     all featured BASIC in ROM.  Woz's Integer BASIC was written from
%%     scratch; TRS-80 Level~I BASIC was based on Palo Alto Tiny BASIC; the
%%     rest are ports of Microsoft BASIC.}

But a problem was starting to emerge: those three different
``versions'' of Micro-Soft BASIC weren't all identical.
The three computers had different hardware capabilities such as
graphics and sound, so Micro-Soft made some changes to tailor each version of
BASIC to its computer model.
As new computers entered the market and rapidly proliferated
innovative hardware features, each manufacturer further modified their
computer's version of BASIC to take advantage of the computers' newest
features.
Like species that gradually become distinct and unable to mate, these
different ``dialects'' of BASIC soon diverged enough that all but the
simplest BASIC programs became vendor-specific---programs written for
one computer's version of BASIC would require changes to work on
another computer's version, if it could be made to work at all.

Kemeny and Kurtz, BASIC's original creators, were dismayed at this
proliferation of incompatible dialects and the addition of
\emph{ad~hoc} features that (in their opinion) were often in poor
technical taste and contrary to the original spirit and goals of the
language.
They began using the derogatory term ``street BASIC'' to distinguish
these incompatible mass-market dialects from the original language
they had been curating and using at Dartmouth.
But as university educators, they may have been even more dismayed
that the ``killer app'' for BASIC turned out to be not formal
computing education as they had intended, but games and information exchange.


