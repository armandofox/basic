

\section{INFLUENCE AND LEGACY}

\makequotation{Whether we're still programming in it or not, the spirit
  of BASIC lives on in all of us.}{Jeff Atwood, founder of
  StackExchange and author of Coding Horror blog~\cite{codinghorror_basic}}

With no economical mass storage (floppy disks) coming until 1980, and no
consumer-facing Internet until the 1990s, BASIC programs were
distributed by a far more prosaic medium: you had to type in the source
code from a printed listing, usually in hobbyist magazines or club
newsletters.

\smallpicfigure{figs/creative_computing_1_1.jpg}{fig:creative_computing}{%
  The inaugural issue of \emph{Creative Computing} (1974--1985)
  autographed by publisher David Ahl, courtesy of
  \href{https://archive.org/details/creativecomputing}{The Internet Archive}. The magazine
  heavily featured ``type-in programs'' in each issue---games, personal finance,
astrology, and every other conceivable category that the community of
computer hobbyists might gravitate towards.  Many issues are available
for from the Internet Archive.}

The mother of all ``type-in'' BASIC games traces its origins to 1971,
when Mike Mayfield and some high school friends ``borrowed'' time on
an SDS Sigma~7 computer at UC~Irvine to write a game inspired by the
original \emph{Star Trek} TV series.
The show had just ended its 3-season run and was very popular among
computer hobbyists.
In the game, the player commands the USS~Enterprise and moves it
around a grid, seeking and destroying Klingon warships.
The grid was rendered as a set of text characters, and was redrawn
after each turn, so the game could be played on both printing
terminals like the ASR-33 and the video displays that were around
the corner.
Younger readers may be surprised to learn how many surprisingly
sophisticated games relied on line-at-a-time text output, but it was the only
output medium guaranteed to be available in all BASIC dialects (and in
most other programming languages as well).

Mayfield later made some changes to allow the program to run on
Hewlett-Packard computers using their ``dialect'' of BASIC, in
exchange for some time on an HP-2000C timesharing computer at a local
HP sales office.

\smallpicfigure{figs/basicgames.png}{fig:basicgames}{%
Ahl ported his original collection of games (including Super Star
Trek) to Microsoft BASIC, which had emerged as the \emph{de~facto}
standard for microcomputers, and published the resulting collection as
\emph{BASIC Computer Games}.
The book became a staple in high school computer labs and hobbyists'
homes.}

The Hewlett-Packard version was discovered by David Ahl, a programmer
then working in the education department of Digital Equipment
Corporation (DEC) and who had previously started the magazine
\emph{Creative Computing}.
In the share-alike mentality of the day, Ahl immediately created a
modified version that worked with DEC's dialect of BASIC (on the RSTS
system mentioned earlier) and published it in the DEC Education
Newsletter, encouraging readers to submit their own games.
That version, Super Star Trek, was
\href{https://archive.org/details/CreativeComputingv01n04MayJune1975/page/n41}{featured}
in the May/June 1975 issue of Creative Computing, a magazine Ahl had
started.
Games were the most popular kind of type-in program published in such
magazines, and Ahl eventually collected the ``greatest hits'' games
published in Creative Computing and compiled them into a book called
\emph{Basic Computer Games}~\cite{basic_computer_games}, the first
computer book to sell over a million copies.

Games in BASIC were major drivers of the success of PCs, despite the
text-only capabilities of the first models.
Scott Adams wrote his first interactive text adventure game in BASIC,
and started the first software company devoted exclusively to game
publishing, Adventure International.
An interactive text adventure is a turn-based game in which you, the
player, are exploring a fictional nworld trying to collect treasure,
battle evil pirates, or whatever.
The player would type something like ``Go North''; the computer types
back something like ``You are next to a running stream''.
Adams even published a complete game called \emph{Pirate's Adventure}
as a type-in program in a special issue of BYTE devoted to adventure
games~\cite{byte80:adventure}; the article helpfully notes which parts
of the BASIC program would have to be changed to run on other
Microsoft-derived BASIC dialects.
Adventure games were not new---the first one (called ADVENT, because
the computer on which it was written allowed file names to be only six
letters long) had been
\href{https://armandofox.blogspot.com/2007/08/the-original-original-adventure.html}{created
by Will Crowther} several years before---but it was written in
FORTRAN, the earlier language created for engineers and
mathematicians, so ADVENT's audience was limited to techies at
universities.
Now, with the wide availability of BASIC and a huge audience of
consumers, such games briefly enjoyed major commercial success, as
related in the fascinating documentary \emph{Get
Lamp}~\cite{get_lamp}.

But the dominance of text-based games was short-lived.  As PC hardware
became more sophisticated, games evolved to exploit new features.
Later versions of Star Trek displayed ships as graphic icons rather than
letters and symbols on a text grid.
Scott Adams enhanced his adventures' visual appeal with colorful---albeit
static---images.
\emph{Oregon Trail}, a turn-based simulation based on the US westward
migration and originally written in 1971 by US~history TA Don Rawitsch and colleagues,
received graphic enhancements and became
a bestseller on the Apple~II and other computers;
the millions of
North American elementary school students who played the game in school
in the 1980s--1990s
may be unaware that it was originally text-only.
Strategy game publisher
Automated Simulations published two turn-based simulation games about
space colonization,
% high-quality ``space opera'' games,
\emph{Starfleet Orion} and \emph{Invasion Orion}, that were not only
written in BASIC but could be modified by sophisticated users to
customize the gameplay: since BASIC programs were interpreted, the
companies had no choice but to provide the source code.
\w{Odyssey: The Compleat Apventure} [sic], a 1980 turn-based adventure game
written in Apple BASIC,
used both the  ``high resolution'' graphics mode ($280\times 192$, 6 colors)
on the Apple to display overview maps and
``low resolution'' mode ($40\times 40$, 15
colors) to display cave and dungeon interiors in some phases of game play.

\picfigure{figs/odyssey.png}{fig:odyssey}{Four screenshots from \emph{Odyssey: The Compleat
    Apventure}, written in Wozniak's original Integer BASIC.  Each ``phase'' of the game was
  a completely separate program, loaded \w[Overlay (programming)]{overlay}-style.}

Meanwhile, Commodore, the company under whose umbrella MOS Technology
had produced the wildly popular 6502 microprocessor, had been working on
an advanced video game chip they hoped to sell to home game console manufacturers
such as Atari.
When Commodore was unable to find any customers for the chip, two young
Commodore engineers, Bob Yannes and Al Charpentier, decided instead to
develop an inexpensive computer around it.
The VIC-20 had very impressive graphics
compared to its competitors, and at just \$300, it dropped the price
of an entry-level PC by more than half.
Bob Albrecht brought his informal and folksy writing style to the
introductory programming manual that came with the breakthrough
computer~\cite{commodore}.
Not only did the VIC-20 outsell all other computers when introduced in
1982 (over 1 million units), it also stimulated the sales of the first
under-\$100 modem, which came bundled with free trials of various dialup
services including CompuServe (a paid service that was a forerunner of
today's World Wide Web) and in 1982 accounted for the largest
traffic on that network.
The first bulletin board systems (BBSes), analogous to discussion
boards such as Reddit today, were written in BASIC.
A little over a year later, Commodore engineers outdid themselves with
the Commodore~64, the best-selling personal computer of all time (over
10 million sold), featuring spectacular graphics and sound and 64KiB
of RAM for an unheard-of \$595.

Commodore chairman Tramiel may be the last person to have outsmarted
Bill Gates.
The wording of the original contract with Microsoft allowed Commodore
to package BASIC with every \emph{computer} sold, not just every
\emph{PET} sold, as long as no substantive changes (``derivative
works'') were made to Microsoft's product.
Of course, this meant Commodore couldn't legally modify Microsoft
BASIC to allow access to the advanced graphics and sound capabilities
in the VIC-20 or Commodore~64, so programmers had to learn the much
more difficult 6502 machine language to make use of those
advanced capabilities.
But it allowed Commodore to sell over 1~million VIC-20s and over
10~million Commodore~64s without further payments to Microsoft.
This infuriated Bill Gates, but it did mean that VIC-20s and
Commodore~64s could generally run the many BASIC programs that had
been created for the PET, especially educational software, making the
computers an easy sell into homes and
schools~\cite[p. 414]{commodore}.

BASIC also provided a new avenue of empowerment for young programmers.
Game designer Richard Garriott, whose \emph{nom de plume} is Lord
British, started writing games in BASIC at the age of thirteen.
At eighteen he created his first game offered for sale,
\emph{Akalabeth,} the first game to provide a 3D first-person
perspective view of the dungeon maze (``first person shooter''),
albeit a crude one~\cite{akalabeth}.
The first version of his very
successful game \emph{Ultima} was also written in BASIC.
Garriott went on to become one of the most successful (and wealthiest)
computer game designers in the world.
In principle, any parent could now buy a computer for their child and
sign a distribution agreement to allow the young prodigy's work to
find its market.

As of this writing (2018), cloud-based programming environments such
as Codenvy and Cloud9 allow anyone to fire up a Web browser and
immediately start programming.
For all the criticism of ``street BASIC,'' its user experience was
quite close to this ``programmer appliance'' view, as BASIC's creators
originally intended: when PC users turned on their computers, the
first thing they'd see was a BASIC interpreter prompt, allowing them
to jump right in and start writing video games before there was Atari
and communications programs before we had the Web.
One of the world's largest and most
successful software companies (Microsoft) got its start by creating
BASIC for the PC, and three of the most successful computer companies
of the time (Commodore, Radio Shack, and Apple) owed their success to
availability and ease of use of that BASIC.
According to Wikipedia, there are today over 450 extant and extinct
dialects of BASIC.
Some would call that victory.

