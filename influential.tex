

\section{INFLUENCE AND LEGACY}

\makequotation{Whether we're still programming in it or not, the spirit
  of BASIC lives on in all of us.}{Jeff Atwood, founder of
  StackExchange.com, author of Coding Horror blog~\cite{codinghorror_basic}}

In 1971, Mike Mayfield and some high school friends used an SDS Sigma 7
computer to write a turn-based text game inspired by the original Star
Trek TV series, which had just ended its 3-season run and was very
popular.
In the game, the player commands the USS Enterprise and moves it around
a grid seeking and destroying Klingon warships.
Mayfield later ported the game to Hewlett-Packard BASIC in exchange for
some time on an HP-2000C timesharing computer at a local HP sales
office.
The HP version was found by David Ahl of DEC's education department, and
in the share-alike mentality of the day, he immediately ported it to DEC
BASIC and shared it freely via the DEC Education
Newsletter, encouraging readers to submit their own games.
Of course, at the time there was no user-friendly Internet to speak of,
so newsletters and magazines would
publish source code listings that users would laboriously type in.

In 1978 Ahl launched \emph{Creative Computing}, one of the first
magazines focused on emerging microcomputers and featuring lots of
so-called ``type-in programs.''
Ahl ported his original collection of games (including an improved
\emph{Super Star Trek}) to Microsoft BASIC, which had emerged as the
\emph{de~facto} standard microcomputer BASIC, and republished the result
as \emph{BASIC Computer Games}~\cite{basic_computer_games}, the first
computer book to sell over a million copies.

BASIC was easy to learn and becoming ubiquitous on microcomputers.
Younger readers may be surprised to learn how many pioneering games were
written in BASIC and relied on line-at-a-time text output, and not just
the (generally short and simple) programs in \emph{BASIC Computer
Games}.
For example, the earliest interactive text-only adventure game (you type
``Go North''; the computer types back something like ``You are next to a
running stream'') had been written by Will Crowther in FORTRAN, but now
that BASIC was clearly going to be the high-level language distributed
with PCs, new adventure game authors such as Scott Adams wrote in BASIC.
Adams even published a complete game called \emph{Pirate's Adventure} as
a type-in program in a special issue of BYTE devoted to adventure
games~\cite{byte80:adventure}; the article helpfully notes which parts
of the BASIC program would have to be changed to run on other
Microsoft-derived BASIC dialects.
\emph{Oregon Trail}, a turn-based simulation based on the US westward
migration, was written in 1971 by Don Rawitsch, a college senior who was
a student teacher for an American history course, with the help of
fellow student teachers Bill Heinemann and Paul Dillenberger.

But soon enough, emerging PCs with graphics opened
new possibilities.
Scott Adams enhanced his adventures' visual appeal with colorful---albeit
static---images.
Oregon Trail also received graphic enhancements and became
a bestseller on the Apple ][ and other computers;
the game was popular among
North American elementary school students in the mid 1980s to late
1990s.  
Automated Simulations published two turn-based simulation games about
space colonization,
% high-quality ``space opera'' games,
\emph{Starfleet Orion} and \emph{Invasion Orion}, that were not only
written in BASIC but could be modified by sophisticated users to
customize the gameplay.

\w{Odyssey: The Compleat Apventure} [sic] was a 1980 turn-based adventure game
written in Apple II Integer BASIC
that made extensive (for the time) use of
graphics, using the Apple's ``high resolution'' mode ($280\times 192$, 6 colors)
to display overall maps showing the player's
avatar and major landscape features (Figure~\ref{fig:odyssey_hires}), and its
``low resolution'' mode ($40\times 40$, 15
colors) mode to display ???.
%% The three main ``phases'' of the game were authored as separate programs
%% that shared data via disk files, using a primitive overlay-like mechanism
%% to transition between phases.  
%% The various programs, totaling 64~KiB of tokenized code, were written in Apple II Integer
%% BASIC, the 
%% original interpreter hand-coded in assembly by Steve Wozniak for
%% the Apple II before Microsoft adapted its floating-point BASIC for that
%% computer.  Between the game programs (64~KiB total), bitmaps of the ``hi-res''
%% maps (four at 8~KiB each), a few assembly routines to help with graphics
%% drawing (6.5~KiB), and the image of Integer BASIC, which had to be
%% included on any boot floppy that wanted to use it since Integer BASIC
%% was replaced by Applesoft BASIC in ROM early in the Apple II's lifetime,
%% the program nearly filled the 130~KiB floppy disk.



BASIC and graphics: a way to do entertainment and games
  - ZX80, TRS-80: graphic character glyphs, plus most BASICs provided "gotoXY"
  - Apple II: lo-res graphics mode repurposes text glyphs
  - Apple II: weirdly memory-mapped hires graphics library; shape tables
  - VIC-20, C64
  - Famous graphics demos in BASIC: Fire Organ, moire pattern, breakout
  (Woz added graphics and PDL() commands to IntBasic just to be able to
  recreate Breakout)

As popular press vehicle for computing
  - "BASIC games" books
  - type-in listings


According to Wikipedia,
there are \w[List of BASIC dialects]{over 450 dialects} of BASIC
(extant and extinct).



Ousterhout predicted interpreted scripting langauges would be ``glue''.
BASIC served this role to glue together bits of assembly.  Graphics,
comms (BBS), serial card, other stuff


Platform constraints as virtues:
 - ZX80 "Hampson's Plane" uses blackout during display generation 
 - ZX80 and other BASICs where "Save" saves BASIC memory image incl
 variable values (save game?)

% milestone: consumer-priced modems => BBSs, and the C64
%   anything good form BBS: The Documentary?



Hunt the Wumpus?


Oregon Trail - original game in HP Timeshared BASIC.  The game was popular among
North American elementary school students in the mid 1980s to late
1990s.  Written  in 1971 by a college senior who was a student teacher for an
American history course, Don Rawitsch, who recruited fellow student
teachers Bill Heinemann and Paul Dillenberger to help.

As implementation of commercial software - strategy games
  - Invasion Orion - from Autoamted Simulations - all their games were
  in BASIC and could be listed, modified, etc.
  - SAGA - first one Pirate's ADventure was in BASIC, but using an
  ``Adventure intepreter'' implemented in TRS-80 Level II BASIC!  SOurce
  code is in Dec 1980 BYTE.
  - Broderbund Galactic Empire (TRS-80?)

% milestone: credible graphics on PCs breathed new life into games,
% since game logic could go far beyond what was possible on game console
% at the time.  VIC-20 and later C-64 bridged that gap.
