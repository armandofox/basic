


\section{INFLUENCE AND LEGACY}

\makequotation{Whether we're still programming in it or not, the spirit
  of BASIC lives on in all of us.}{Jeff Atwood, founder of
  StackExchange and author of Coding Horror blog~\cite{codinghorror_basic}}

\smallpicfigure{figs/startrek.pdf}{fig:startrek}{%
  Mike Mayfield's original \emph{Star Trek} could be played on any
  computer that could display text.
  The USS Enterprise (represented by ``E'') visits various sectors in
  the grid looking for possible Klingons (``*'') without running out
  of fuel or torpedoes or getting blown up by the Klingons.  The
  \href{https://en.wikipedia.org/Star_Trek_(1971_video_game}{Wikipedia
    article} discusses the game's hundreds of subsequent variations,
  especially adding graphics and sound, and links to the game's source
code and even online versions that behave like the original.} 

BASIC may have started as an academic exercise to introduce
nontechnical students to programming, but BASIC's dominance in the
early days of microcomputers empowered hobbyists to kickstart two
computer-related industries that would largely define the subsequent
decades: computer games and networking.

Although cassettes were widely used by hobbyists as a way to store
programs inexpensively, and floppy disk drives were slowly beginning
to appear (they wouldn't become sufficiently reliable or inexpensive
for hobbyists untile the mid-1980s), cassette and disk formats were incompatible across
different computer systems.
Cassettes or disks created for a TRS-80 wouldn't work on an Apple or
Commodore, just as today's PC programs don't work on Macs.
So hobbyists had to laboriously type in programs from printed
listings---usually ones they found
in hobbyist magazines.

The mother of all ``type-in'' BASIC games
traces its origins to 1971,
when Mike Mayfield and some high school friends ``borrowed'' time on
an SDS Sigma~7 computer at UC~Irvine to write a game inspired by the
original \emph{Star Trek} TV series, which had just ended its 3-season
run and was very popular among computer hobbyists.
As Figure~\ref{fig:startrek} shows, the game grid was displayed as a
set of text characters that was redrawn 
after each turn, so the game could be played on both printing
terminals like the ASR-33 and the video displays that were around
the corner.

  
\smallpicfigure{figs/creative_computing_1_1.jpg}{fig:creative_computing}{%
  An autographed inaugural issue of \emph{Creative Computing}
  (1974--1985), launched by  David  Ahl in 1974
during  a leave of absence from DEC.  The magazine
  heavily featured ``type-in programs'' in each issue---games, personal finance,
astrology, business, education.  \emph{Super Star Trek} was featured in
the \href{https://archive.org/details/CreativeComputingv01n04MayJune1975/page/n41}{May/June
  1975} issue, one of many available from The Internet Archive.}

A version of Mayfield's game
was discovered by David Ahl, a programmer
then working at DEC (makers of the popular PDP series we've already
encountered a few times) and editing their education newsletter.
In the share-alike mentality of the day, Ahl created a modified
version he called Super Star Trek that worked with DEC's dialect of
BASIC and published it in the
newsletter, encouraging readers to submit their own games.
This version, and many other ``greatest hits'' of games written in
BASIC, found their way into a book of type-in programs edited by Ahl called \emph{BASIC
Computer Games}~\cite{basic_computer_games}.

    \smallpicfigure{figs/basicgames.png}{fig:basicgames}{%
\emph{BASIC Computer Games,} edited by David Ahl, was the first computer book to sell over a
million copies and was a staple in high school computer labs and hobbyists'
homes.
Most of the games in it will still work with
GWBASIC, the version of BASIC included with MS-DOS and subsequently
replaced by a version called QBASIC included with Windows.
}

Younger readers may be surprised to learn that all of those games
relied on line-at-a-time text output, the only
output medium guaranteed to be available on every type of
microcomputer.
One very popular genre was the interactive text adventure,
a turn-based game in which you, the
player, are exploring a fictional world trying to collect treasure,
battle evil pirates, or whatever.
The player types something like ``Go North''; the computer types
back something like ``You are next to a running stream''.  While
exploring, you can pick up objects, fight evil creatures, and so on.
Scott Adams
started the first software company devoted exclusively to game
publishing, Adventure International, selling games he authored in BASIC.
(Adams even published a complete game called \emph{Pirate's Adventure}
as a type-in program in a special issue of BYTE devoted to adventure
games~\cite{byte80:adventure}, to show how it was done.)

\begin{tangent}
  Adventure games were not new---the first one (called ADVENT, because
  the computer on which it was written allowed file names to be only six
  letters long) had been
  \href{https://armandofox.blogspot.com/2007/08/the-original-original-adventure.html}{created
by Will Crowther} several years before---but it was written in
FORTRAN, the earlier language created for engineers and
mathematicians, so tinkering with ADVENT required familiarity with
that more difficult language.
Now, with the wide availability of BASIC and a huge audience of
consumers, such games briefly enjoyed major commercial success, as
related in the fascinating documentary \emph{Get
Lamp}~\cite{get_lamp}.
\end{tangent}

As microcomputer hardware rapidly improved,
games were among the first programs to exploit new features.
Star Trek was updated to display ships as graphic icons rather than
letters and symbols on a text grid.
Scott Adams's adventures gained colorful (albeit static) images to
show each scene.
\emph{Oregon Trail}, an adventure-like game based on the US westward
migration and written in 1971 by US~history teacher Don Rawitsch and colleagues,
received graphic enhancements and became
a bestseller on the Apple~II and other computers;
the millions of
North American elementary school students who played the game in school
in the 1980s--1990s
may be unaware that it was originally text-only.
Strategy game publisher
Automated Simulations published two turn-based simulation games about
space colonization,
% high-quality ``space opera'' games,
\emph{Starfleet Orion} and \emph{Invasion Orion}, that were not only
written in BASIC but could be modified by sophisticated users to
customize the gameplay: since BASIC programs were interpreted, the
companies had no choice but to provide the source code.
\w{Odyssey: The Compleat Apventure} [sic], a 1980 turn-based adventure game
written in Apple BASIC,
made extensive use of the Apple's different graphic display ``modes''
in different phases of game play.

\picfigure{figs/odyssey.png}{fig:odyssey}{Four screenshots from \emph{Odyssey: The Compleat
    Apventure}, written in Wozniak's original Integer BASIC.  Each ``phase'' of the game was
  a completely separate program, loaded \w[Overlay (programming)]{overlay}-style.}

BASIC also allowed the creation of a new profession---the independent
programmer.
Game designer Richard Garriott, whose \emph{nom de plume} is Lord
British, started writing games in BASIC at the age of thirteen.
At eighteen he created
\emph{Akalabeth,} the first game to provide a (crude) 3D first-person
perspective view of the dungeon maze (``first person shooter'').
The first version of his very
successful game \emph{Ultima} was also written in BASIC.
Garriott became one of the most successful (and wealthiest)
computer game designers in the world.
In principle, any parent could now buy a computer for their child and
sign a distribution agreement to allow the young prodigy's work to
find its market.

\begin{tangent}
Commodore chairman Tramiel may be the last person to have outsmarted
Bill Gates.
The wording of the original contract under which Microsoft created
BASIC for the Commodore PET allowed Commodore
to package BASIC with every \emph{computer} sold, not just every
\emph{PET} sold, as long as no substantive changes (``derivative
works'') were made to Microsoft's product.
Of course, this meant Commodore couldn't legally modify Microsoft
BASIC to allow access to the advanced graphics and sound capabilities
for the computers they'd produce later---programmers of those
computers would have to learn the much
more difficult 6502 machine language to make use of their
advanced capabilities.
But the license did allow Commodore to sell over 1~million VIC-20s and over
10~million Commodore~64s (the single best-selling microcomputer
model of all time) with no further payments to Microsoft.
This infuriated Bill Gates, but it did mean that VIC-20s and
Commodore~64s could generally run the many BASIC programs that had
been created for the PET, especially educational software, making
Commodore's later
computers an easy sell into homes and
schools~\cite[p. 414]{commodore}.
\end{tangent}

The second major contribution of BASIC hobbyists was to pave the
earliest lanes of the information superhighway.
After Commodore's success with the PET, chairman Tramiel was obsessed
with creating a radically inexpensive ``computer for the masses'' that
would have color graphics and sound to rival video-game systems, but
still be a genuine computer.
%% one of the ``big three'' early microcomputer makers, had been working on
%% a custom chip for color graphics and sound, which they hoped to sell
%% to home game console manufacturers. 
%% When they failed to find any customers for the chip, two young
%% engineers,
Young engineers Bob Yannes and Al Charpentier rose to the challenge
and designed the VIC-20, which paired the inexpensive and successful 6502
microprocessor with a new custom-designed chip that offered graphics and sound
capabilities far beyond those of the VIC-20's competitors, at the astonishing
price of \$300---a quarter of the price of the cheapest
competitor featuring color graphics and sound, and about the same price
as a home video game console.
%% Bob Albrecht brought his informal and folksy writing style to the
%% introductory programming manual that came with the breakthrough
%% computer~\cite{commodore}.
But that wasn't enough for Tramiel: besides outselling all other computers when introduced in
1982 (over 1 million units), the VIC-20 also stimulated the sales of the first
under-\$100 modem, allowing VIC-20 owners to connect their computers
to the landline telephone system.  But who would they call?
Commodore chairman Jack Tramiel negotiated a deal with 
CompuServe, a paid service that was a forerunner of
today's World Wide Web and provided online chat, message boards,
libraries of freely downloadable software, business-related
content such as the text of articles from major newspapers, and even
images and comics.  Buyers of the VICmodem received nearly \$200 of credits
on CompuServe, its competitor The Source, and the Dow Jones financial
information service.  In 1982, VICmodem users accounted for the largest
traffic on CompuServe's system.
As modems proliferated, individual programmers would create their own
bulletin board systems (BBSes)  in BASIC, providing a free
community-based alternative to these paid services.
Today's topic-based, membership-oriented message exchange boards, such
as Reddit and StackOverflow, are the distant descendants of these
efforts.

\begin{tangent}
  CompuServe invented the now-ubiquitous GIF (Graphics Interchange Format) as a way to
  compress comic-book-like images for transmission over the slow
  modems of the day---early consumer modems could transmit just over
  1~kilobyte per second, and it would be years before they hit the
  blazing speed of 64~kilobytes per second, about the fastest that
  could be managed using analog phone lines.
\end{tangent}

At the time of this writing (2018), anyone can point their Web browser
at a cloud-based programming site such
as Codenvy and immediately start programming, without first performing
the potentiall confusing steps of installing and configuration special
software to allow programming to happen.
For all the criticism of ``street BASIC,'' its user experience was
exactly this in the early days of microcomputers: as BASIC's creators
originally intended, when PC users turned on their computers, the
first thing they'd see was a BASIC interpreter prompt---eliminating
virtually every confusing step that might come between a beginning programmer
turning on the computer and starting to type their first line
of code.
From that humble launching pad, newbies
created video games before we had Atari,
and communications programs before we had the Web.
One of the world's largest and most
successful software companies (Microsoft) got its start by creating
BASIC for the PC, and three of the most successful computer companies
of the time (Commodore, Radio Shack, Apple) owed their success to
that BASIC's availability and ease of use.
According to Wikipedia, even today there are over 450 extant and extinct
dialects of BASIC.
Some would call that victory.

