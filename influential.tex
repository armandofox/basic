

\section{INFLUENCE AND LEGACY}

\makequotation{Whether we're still programming in it or not, the spirit
  of BASIC lives on in all of us.}{Jeff Atwood, founder of
  StackExchange and author of Coding Horror blog~\cite{codinghorror_basic}}

BASIC's role in the 1980s is analogous to JavaScript's role in the
1990s: its performance and relative isolation from the hardware limited
the kinds of programs you could write, but your programs could have a
huge audience, especially if you didn't rely too heavily on
platform-specific features in your programs.
With no economical mass storage (floppy disks) coming until 1980, and no
consumer-facing Internet until the 1990s, BASIC programs were
distributed by a far more prosaic medium: magazines published source
code listings that users would laboriously type in.

One of the ancestors of ``type-in'' BASIC games traces its origins to
1971, when Mike Mayfield and some high school friends ``borrowed'' time
on an SDS Sigma~7 computer at UC~Irvine to write a game inspired by the
original Star Trek TV series, which had just ended its 3-season run and
was very popular among computer hobbyists.
In the game, the player commands the USS Enterprise and moves it around
a grid seeking and destroying Klingon warships.
Mayfield later ported the game to Hewlett-Packard BASIC in exchange for
some time on an HP-2000C timesharing computer at a local HP sales
office.
The HP version was discovered by David Ahl of Digital Equipment
Corporation's education department; DEC was then the most successful
maker of minicomputers, which had carved out a profitable niche in
universities and small business but were about to be overtaken by
microcomputers.  In the share-alike mentality of the day, Ahl
immediately ported Star Trek to DEC's dialect of BASIC and shared it freely via the DEC
Education Newsletter, encouraging readers to submit their own games.

In 1978 Ahl launched \emph{Creative Computing}, one of the first
magazines focused on emerging microcomputers and featuring lots of
so-called ``type-in programs.''
Ahl ported his original collection of games (including an improved
\emph{Super Star Trek}) to Microsoft BASIC, which had emerged as the
\emph{de~facto} standard microcomputer BASIC, and published the result
as \emph{BASIC Computer Games}~\cite{basic_computer_games}, the first
computer book to sell over a million copies, and became a staple in high
school computer labs and hobbyists' homes.

Younger readers may be surprised to learn how many pioneering games
written in BASIC relied on line-at-a-time text output, but
surprisingly sophisticated games were created this way.
Scott Adams wrote his first interactive text adventure games in BASIC, and started
the first software company devoted exclusively to game publishing,
Adventure International.\footnote{The earliest interactive text adventure (you type
``Go North''; the computer types back something like ``You are next to a
running stream'') had been written by Will Crowther in FORTRAN.}
Adams even published a complete game called \emph{Pirate's Adventure} as
a type-in program in a special issue of BYTE devoted to adventure
games~\cite{byte80:adventure}; the article helpfully notes which parts
of the BASIC program would have to be changed to run on other
Microsoft-derived BASIC dialects.
\emph{Oregon Trail}, a turn-based simulation based on the US westward
migration, was written in 1971 by Don Rawitsch, a college senior who was
a teaching assistant for a US~history course, with the help of
fellow student teachers Bill Heinemann and Paul Dillenberger.

Falling hardware prices soon enabled PCs with crude graphics.
Later versions of Star Trek displayed ships as graphic icons rather than
letters and symbols on a text grid.
Scott Adams enhanced his adventures' visual appeal with colorful---albeit
static---images.
Oregon Trail also received graphic enhancements and became
a bestseller on the Apple~II and other computers;
the millions of
North American elementary school students who played the game in school
in the 1980s--1990s
may be unaware that it was originally text-only.
Strategy game publisher
Automated Simulations published two turn-based simulation games about
space colonization,
% high-quality ``space opera'' games,
\emph{Starfleet Orion} and \emph{Invasion Orion}, that were not only
written in BASIC but could be modified by sophisticated users to
customize the gameplay: since BASIC programs were interpreted, the
companies had no choice but to provide the source code.

\w{Odyssey: The Compleat Apventure} [sic] was a 1980 turn-based adventure game
written in Apple II Integer BASIC
that made extensive (for the time) use of
graphics, using the Apple's ``high resolution'' mode ($280\times 192$, 6 colors)
to display overall maps showing the player's
avatar and major landscape features (Figure~\ref{fig:odyssey_hires}), and its
``low resolution'' mode ($40\times 40$, 15
colors) mode to display ???.

When Commodore was unable to find any customers for a video game chip it
had developed, they instead used it in a new computer called the VIC-20,
While its hardware was entry-level, the graphics were impressive
compared to its competitors, and at just \$300, it dropped the price of
an entry-level PC by more than half.
Bob Albrecht brought his informal and folksy writing style to the
introductory programming manual that came with the breakthrough
computer~\cite{commodore}.
Not only did the VIC-20 outsell all other computers when introduced in
1982 (over 1 million units), it also stimulated the sales of the
first under-\$100 modem, which came bundled with free trials of various
dialup services including CompuServe and in 1982 accounted for the
largest traffic on that network.  The first bulletin board systems
(BBSes)---analogous to dialup newsgroup servers in the Unix world---were
written in BASIC.

\begin{tangent}{The last person to outsmart Bill Gates}
Hard-driving Commodore chairman Tramiel claimed that the wording of the
original ``flat fee'' deal between Commodore and Microsoft allowed
Commodore to make minor tweaks to the original PET BASIC and include it
with every VIC-20 and Commodore 64, without further payments to
Microsoft.
Over 1~million VIC-20s and over 10 million Commodore~64s would be sold.
This infuriated Gates, and it resulted in a BASIC unable to exploit
these computers' advanced graphics, but it did allow VIC-20 and
Commodore~64 buyers to run many BASIC programs that had been written
for the PET, especially educational software, making the computers an
easy sell into both homes and schools~\cite[p. 414]{commodore}.

\end{tangent}

As of this writing (2016), cloud-based IDEs such as Codenvy and Cloud9
have come into their own: anyone can fire up a Web browser and
immediately start programming.
For all the criticism of ``street BASIC,'' its user experience was quite
close to this ``programmer appliance'' view, as BASIC's creators
originally intended: when PC users turned on their computers, the first
thing they'd see was a BASIC interpreter prompt, allowing them to jump
right in and start programming.
And program they did: the first PC games and the first PC
telecommunications programs were written in BASIC, magazines published
program listings in it, and one of the world's largest and most
successful software companies got its start with well-crafted BASIC
interpreters.
According to Wikipedia, there are today over 450 extant and extinct
dialects of BASIC.
Some would call that victory.
