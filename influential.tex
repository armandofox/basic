

\section{INFLUENCE AND LEGACY}

\makequotation{Whether we're still programming in it or not, the spirit
  of BASIC lives on in all of us.}{Jeff Atwood, founder of
  StackExchange and author of Coding Horror blog~\cite{codinghorror_basic}}

With no economical mass storage (floppy disks) coming until 1980, and no
consumer-facing Internet until the 1990s, BASIC programs were
distributed by a far more prosaic medium: you had to type in the source
code from a printed listing.  Students traded source code listings, 
magazines published listings contributed by readers, and in 1972, David
Ahl, a programmer then working in the education department of Digital Equipment
Corporation and who had previously started the magazine \emph{Creative
  Computing}, published a whole book of them---\emph{BASIC Computer
  Games}~\cite{basic_computer_games}. 

\smallpicfigure{figs/creative_computing_1_1.jpg}{fig:creative_computing}{%
  The inaugural issue of \emph{Creative Computing} (1974--1985)
  autographed by publisher David Ahl, courtesy of
  \href{https://archive.org/details/creativecomputing}{The Internet Archive}. The magazine
  heavily featured ``type-in programs'' in each issue---games, personal finance,
astrology, and every other conceivable category that the community of
computer hobbyists might gravitate towards.  Many issues are available
for from the Internet Archive.}

The mother of all ``type-in'' BASIC games traces its origins to
1971, when Mike Mayfield and some high school friends ``borrowed'' time
on an SDS Sigma~7 computer at UC~Irvine to write a game inspired by the
original \emph{Star Trek} TV series.  The show had just ended its 3-season run and
was very popular among computer hobbyists.
In the game, the player commands the USS~Enterprise and moves it around
a grid, seeking and destroying Klingon warships.  The grid was rendered
as a set of text characters, and was redrawn after each turn, so
the game could be played on both printing terminals like the ASR-33 and
the video displays that were around the corner.
Mayfield later ported\footnote{To ``port'' a program is to make the
  necessary changes to allow it to run on a computer other than the
  one for which it was written.  Typically porting implied a full
  rewrite of the program from the original design, much like building another building from the
  same blueprint but using different materials.}
the game to Hewlett-Packard BASIC in exchange for
some time on an HP-2000C timesharing computer at a local HP sales
office.

\smallpicfigure{figs/basicgames.png}{fig:basicgames}{%
Ahl ported his original collection of games (including Super Star
Trek) to Microsoft BASIC, which had emerged as the \emph{de~facto}
standard for microcomputers, and published the resulting collection as
\emph{BASIC Computer Games}.
The book became a staple in high school computer labs and hobbyists'
homes.}

The HP version was discovered by David Ahl, who (in the share-alike mentality of the day)
immediately ported it to DEC's dialect of BASIC  (on the RSTS system
mentioned earlier!) and shared it freely via the DEC
Education Newsletter, encouraging readers to submit their own games.
That version, Super Star Trek, was
\href{https://archive.org/details/CreativeComputingv01n04MayJune1975/page/n41}{featured}
in the May/June 1975 issue of Creative Computing as well as in the
breakthrough book \emph{Basic Computer Games}~\cite{basic_computer_games}, the first computer book
to sell over a million copies.

Younger readers may be surprised to learn how many 
surprisingly sophisticated games relied on line-at-a-time text output,
the only output medium guaranteed to be available in all BASIC dialects.
For example, Scott Adams wrote his first interactive text adventure game
in BASIC, and started
the first software company devoted exclusively to game publishing,
Adventure International.\footnote{The earliest interactive text adventure (you type
``Go North''; the computer types back something like ``You are next to a
running stream'') had been
\href{https://armandofox.blogspot.com/2007/08/the-original-original-adventure.html}{written by Will Crowther in FORTRAN.}}
Adams even published a complete game called \emph{Pirate's Adventure} as
a type-in program in a special issue of BYTE devoted to adventure
games~\cite{byte80:adventure}; the article helpfully notes which parts
of the BASIC program would have to be changed to run on other
Microsoft-derived BASIC dialects.

Games were crucial to the success of BASIC on PCs.
Later versions of Star Trek displayed ships as graphic icons rather than
letters and symbols on a text grid.
Scott Adams enhanced his adventures' visual appeal with colorful---albeit
static---images.
\emph{Oregon Trail}, a turn-based simulation based on the US westward
migration and originally written in 1971 by US~history TA Don Rawitsch and colleagues,
received graphic enhancements and became
a bestseller on the Apple~II and other computers;
the millions of
North American elementary school students who played the game in school
in the 1980s--1990s
may be unaware that it was originally text-only.
Strategy game publisher
Automated Simulations published two turn-based simulation games about
space colonization,
% high-quality ``space opera'' games,
\emph{Starfleet Orion} and \emph{Invasion Orion}, that were not only
written in BASIC but could be modified by sophisticated users to
customize the gameplay: since BASIC programs were interpreted, the
companies had no choice but to provide the source code.

\w{Odyssey: The Compleat Apventure} [sic] was a 1980 turn-based adventure game
written in Apple II Integer BASIC
that made extensive (for its time) use of
graphics, using the Apple's ``high resolution'' mode ($280\times 192$, 6 colors)
to display overall maps showing the player's
avatar and major landscape/seascape features, and its
``low resolution'' mode ($40\times 40$, 15
colors) mode to display cave and dungeon interiors in some phases of game play.
Figure~\ref{fig:odyssey} shows some screen shots from the game.

\picfigure{figs/odyssey.png}{fig:odyssey}{Four screenshots from \emph{Odyssey: The Compleat
    Apventure}, written in Wozniak's original Integer BASIC.  Each ``phase'' of the game was
  a completely separate program, loaded \w[Overlay (programming)]{overlay}-style.}

Meanwhile, Commodore, the company under whose umbrella MOS Technology
had produced the wildly popular 6502 microprocessor, had been working on
an advanced video game chip they hoped to sell to home game console manufacturers
such as Atari.
When Commodore was unable to find any customers for the chip, two young
Commodore engineers, Bob Yannes and Al Charpentier, decided instead to
develop an inexpensive computer around it.
The VIC-20 had entry-level hardware but very impressive graphics
compared to its competitors, and at just \$300, it dropped the price
of an entry-level PC by more than half.
Bob Albrecht brought his informal and folksy writing style to the
introductory programming manual that came with the breakthrough
computer~\cite{commodore}.
Not only did the VIC-20 outsell all other computers when introduced in
1982 (over 1 million units), it also stimulated the sales of the first
under-\$100 modem, which came bundled with free trials of various dialup
services including CompuServe and in 1982 accounted for the largest
traffic on that network.
The first bulletin board systems (BBSes)---analogous to dialup
newsgroup servers in the Unix world---were written in BASIC.
A little over a year later, Commodore engineers outdid themselves with
the Commodore~64, the best-selling personal computer of all time (over
10 million sold), featuring spectacular graphics and sound and 64KiB
of RAM for an unheard-of \$595.

Part of the credit for Commodore's success is doubtless due to
hard-driving Commodore chairman Tramiel, who may be the last person to
have outsmarted Bill Gates.
Tramiel insisted that the wording of the original ``flat fee'' paid to
Microsoft for porting BASIC to the PET allowed Commodore to package
BASIC with every \emph{computer} sold, not just every \emph{PET} sold,
as long as no substantive changes (``derivative works'') were made.
Of course, this meant Commodore couldn't legally modify Microsoft
BASIC to allow access to the advanced graphics and sound capabilities
in the VIC-20 or Commodore~64, so programmers had to learn the much
more difficult 6502 machine language to really make use of those
advanced capabilities.
But it allowed Commodore to sell over 1~million VIC-20s and over
10~million Commodore~64s without further payments to Microsoft.
This infuriated Bill Gates, but it did mean that VIC-20s and
Commodore~64s could generally run the many BASIC programs that had
been created for the PET, especially educational software, making the
computers an easy sell into homes and
schools~\cite[p. 414]{commodore}.

BASIC's role in the 1980s is analogous to JavaScript's role in the
1990s: its performance and relative isolation from the hardware
limited
the kinds of programs you could write, but those programs could reach a
huge audience.
But as falling hardware prices enabled PCs to add unique but
idiosyncratic hardware features such as graphics and sound, BASIC
programs that relied on
platform-specific features weren't portable, so the potential audience
for a given BASIC program soon narrowed to the audience for that
platform.  Programs for popular computers such as the Apple II,
Commodore VIC-20, and Commodore 64 still found a wide audience;
less-popular computers such as the TI-99/4A, Atari 400, and others soon
saw their ``shareware'' base contract.

BASIC also provided a new avenue of empowerment for young programmers.
Game designer Richard Garriott started writing games in BASIC at the age
of thirteen, and created his first game offered for sale,
\emph{Akalabeth,} when he was just eighteen.
While that game was perhaps primitive compared to some commercial
competitors written in faster assembly language, it was the first game
to provide a 3D first-person perspective view of the dungeon maze,
albeit a crude one~\cite{akalabeth}.
Garriott went on to become one of the most successful (and
wealthiest) computer game designers in the world; the first version of
his very successful \emph{Ultima} was also written in BASIC.
In principle, any parent could now buy a computer for their child and
sign a distribution agreement to allow the young prodigy's work to
find its market.

As of this writing (2016), cloud-based IDEs such as Codenvy and Cloud9
have come into their own: anyone can fire up a Web browser and
immediately start programming.
For all the criticism of ``street BASIC,'' its user experience was quite
close to this ``programmer appliance'' view, as BASIC's creators
originally intended: when PC users turned on their computers, the first
thing they'd see was a BASIC interpreter prompt, allowing them to jump
right in and start programming.
And program they did: the first PC games and the first PC
telecommunications programs were written in BASIC, magazines published
program listings in it, and one of the world's largest and most
successful software companies got its start with well-crafted BASIC
interpreters.
According to Wikipedia, there are today over 450 extant and extinct
dialects of BASIC.
Some would call that victory.
