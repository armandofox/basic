

\section{INFLUENCE AND LEGACY}

\makequotation{Whether we're still programming in it or not, the spirit
  of BASIC lives on in all of us.}{Jeff Atwood, founder of
  StackExchange.com, author of Coding Horror blog~\cite{codinghorror_basic}}

Street BASIC may have corrupted the language, but the tradeoff was that
it was used to write influential production software.
While BASIC had never been designed for speed, it was fast enough that
non-expert programmers 
could create entertainment, games, and educational software that were
more interesting because of the use of graphics.

Early BASIC games such as Star Trek and Hunt the Wumpus already had loyal
followers, and because their game displays used only text characters,
they could be fairly easily ported across computers.
Similarly, interactive text-only adventure games had long been
popular (you type ``Go North''; the computer types back something like
``You are next to a running stream'').  Adventure game author Scott
Adams even published the code for Pirate's Adventure 
written in the 
TRS-80's Microsoft-derived BASIC in a special issue of BYTE devoted
largely to adventure games~\cite{byte80:adventure}; the article
helpfully notes which parts of the BASIC program would have to be
changed to run on other Microsoft-derived BASIC dialects.  (In 1980
there was no user-friendly Internet to speak of, so the main method by
which free programs were distributed was magazines that published
listings that 
users would laboriously type in.)

But now, emerging PCs with graphics opened
new possibilities.
Scott Adams's Graphic Adventures
were instant hits because of the way they used colorful
graphics---albeit just static images---to enhance the game's visual
appeal.  
Woz built just enough graphics and sound into the
Apple II to be able to recreate 
\emph{Breakout}, a game he had designed while working at Atari.
Oregon Trail, an educational game about American history originally
written in BASIC by a group of student teachers
in 1971, received graphic enhancements and became
a bestseller on the Apple ][ and other computers;
the game was popular among
North American elementary school students in the mid 1980s to late
1990s.  
%% Written  in 1971 by a college senior who was a student teacher for an
%% American history course, Don Rawitsch, who recruited fellow student
%% teachers Bill Heinemann and Paul Dillenberger to help.
Automated Simulations published two high-quality ``space opera'' games,
\emph{Starfleet Orion} and \emph{Invasion Orion}, that were not only
written in BASIC but could be modified by sophisticated users to
customize the gameplay.

\w{Odyssey: The Compleat Apventure} [sic] was an early (1980) adventure game
for the Apple~][.
Like prior text adventures such as Advent and Zork, the game is
turn-based, with the player using the 
keyboard to select one of a half-dozen or so possible actions
constrained by context: move in a certain direction, pick up an
available object, and so on.
Unlike its predecessors, however, it made extensive (for the time) use of
graphics, using the Apple's ``high resolution'' mode ($280\times 192$, 6 colors)
to display overall maps showing the player's
avatar and major landscape features (Figure~\ref{fig:odyssey_hires}), and its
``low resolution'' mode ($40\times 40$, 15
colors) mode to display ???.
The three main ``phases'' of the game were authored as separate programs
that shared data via disk files, using a primitive overlay-like mechanism
to transition between phases.  
The various programs, totaling 64~KiB of tokenized code, were written in Apple II Integer
BASIC, the 
original interpreter hand-coded in assembly by Steve Wozniak for
the Apple II before Microsoft adapted its floating-point BASIC for that
computer.  Between the game programs (64~KiB total), bitmaps of the ``hi-res''
maps (four at 8~KiB each), a few assembly routines to help with graphics
drawing (6.5~KiB), and the image of Integer BASIC, which had to be
included on any boot floppy that wanted to use it since Integer BASIC
was replaced by Applesoft BASIC in ROM early in the Apple II's lifetime,
the program nearly filled the 130~KiB floppy disk.



BASIC and graphics: a way to do entertainment and games
  - ZX80, TRS-80: graphic character glyphs, plus most BASICs provided "gotoXY"
  - Apple II: lo-res graphics mode repurposes text glyphs
  - Apple II: weirdly memory-mapped hires graphics library; shape tables
  - VIC-20, C64
  - Famous graphics demos in BASIC: Fire Organ, moire pattern, breakout
  (Woz added graphics and PDL() commands to IntBasic just to be able to
  recreate Breakout)

As popular press vehicle for computing
  - "BASIC games" books
  - type-in listings


According to Wikipedia,
there are \w[List of BASIC dialects]{over 450 dialects} of BASIC
(extant and extinct).



Ousterhout predicted interpreted scripting langauges would be ``glue''.
BASIC served this role to glue together bits of assembly.  Graphics,
comms (BBS), serial card, other stuff


Platform constraints as virtues:
 - ZX80 "Hampson's Plane" uses blackout during display generation 
 - ZX80 and other BASICs where "Save" saves BASIC memory image incl
 variable values (save game?)

% milestone: consumer-priced modems => BBSs, and the C64
%   anything good form BBS: The Documentary?



Hunt the Wumpus?


Oregon Trail - original game in HP Timeshared BASIC.  The game was popular among
North American elementary school students in the mid 1980s to late
1990s.  Written  in 1971 by a college senior who was a student teacher for an
American history course, Don Rawitsch, who recruited fellow student
teachers Bill Heinemann and Paul Dillenberger to help.

As implementation of commercial software - strategy games
  - Invasion Orion - from Autoamted Simulations - all their games were
  in BASIC and could be listed, modified, etc.
  - SAGA - first one Pirate's ADventure was in BASIC, but using an
  ``Adventure intepreter'' implemented in TRS-80 Level II BASIC!  SOurce
  code is in Dec 1980 BYTE.
  - Broderbund Galactic Empire (TRS-80?)

% milestone: credible graphics on PCs breathed new life into games,
% since game logic could go far beyond what was possible on game console
% at the time.  VIC-20 and later C-64 bridged that gap.

