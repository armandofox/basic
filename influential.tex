

\section{INFLUENTIAL PRODUCTION SOFTWARE IN BASIC}

% milestone: consumer-priced modems => BBSs, and the C64
%   anything good form BBS: The Documentary?

\makequotation{Whether we're still programming in it or not, the spirit
  of BASIC lives on in all of us.}{Jeff Atwood, founder of
  StackExchange.com, author of Coding Horror blog~\cite{codinghorror_basic}}

Street BASIC may have corrupted the language, but the tradeoff was that
it was used to write influential production software.


Hunt the Wumpus?


Oregon Trail - original game in HP Timeshared BASIC.  The game was popular among
North American elementary school students in the mid 1980s to late
1990s.  Written  in 1971 by a college senior who was a student teacher for an
American history course, Don Rawitsch, who recruited fellow student
teachers Bill Heinemann and Paul Dillenberger to help.

As implementation of commercial software - strategy games
  - Invasion Orion - from Autoamted Simulations - all their games were
  in BASIC and could be listed, modified, etc.
  - SAGA - first one Pirate's ADventure was in BASIC, but using an
  ``Adventure intepreter'' implemented in TRS-80 Level II BASIC!  SOurce
  code is in Dec 1980 BYTE.
  - Broderbund Galactic Empire (TRS-80?)



\w{Odyssey: The Compleat Apventure} was an early (1980) adventure game
for the Apple~II.
Like prior text adventures such as Advent and Zork, the game is
turn-based, with the player using the 
keyboard to select one of a half-dozen or so possible actions
constrained by context: move in a certain direction, pick up an
available object, and so on.
Like its descendants, however, it made extensive (for the time) use of
graphics, using both the ``high resolution'' mode ($280\times 192$, 0 colors
depending on how you count) to display overall maps showing the player's
avatar and major landscape features (Figure~\ref{fig:odyssey_hires}), and the
``low resolution'' ($40\times 40$, 15
colors) modes of the Apple II.
The three main ``phases'' of the game were authored as separate programs
that shared data via disk files; a primitive overlay-like mechanism was
used to transition between phases.  
The various programs totaling 64~KiB of tokenized code were written in Apple II Integer
BASIC, the
original interpreter hand-coded in assembly by Steve Wozniak for
the Apple II before Microsoft adapted its floating-point BASIC for that
computer.  Between the game programs (64~KiB total), bitmaps of the ``hi-res''
maps (four at 8~KiB each), a few assembly routines to help with graphics
drawing (6.5~KiB), and the image of Integer BASIC, which had to be
included on any boot floppy that wanted to use it since Integer BASIC
was replaced by Applesoft BASIC in ROM early in the Apple II's lifetime,
the program nearly filled the 130~KiB floppy disk.
