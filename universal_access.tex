
\section{UNIVERSAL ACCESS AND COMPUTATIONAL THINKING}

%% \makequotation{Thinking like a computer scientist means more than being able to 
%% program a computer. It requires thinking at multiple levels of
%% abstraction.}{Jeannette M. Wing, \emph{Computational
%%     Thinking}~\cite{wing_computational_thinking}} 

\makequotation{I think everyone should learn how to program a computer,
  because it teaches you how to think.}{Steve Jobs~\cite{steve_jobs_interview}}


For a while in the 80s and 90s, ``computer literacy'' meant familiarity
with productivity tools such as email, Web browsers, word processors,
spreadsheets, and so on.
But within the computer science community, discussions about computer
literacy have always been framed quite differently.
In 1996, MIT computer scientist and educator \w{Seymour Papert} coined
the term \w{computational thinking} to describe the mindset and types of
techniques that characterize a computer scientist's view of
problem-solving.
Ten years later, an
\href{http://www.cs.cmu.edu/afs/cs/usr/wing/www/publications/Wing06.pdf}{influential
op-ed} by Carnegie-Mellon University computer scientist Jeannette Wing
argued for the practical importance of computational thinking and
problem solving~\cite{wing_computational_thinking}.
In 2013, two influential computer scientists, remarking on the demand
for computational thinking in virtually every career path \emph{outside}
of computer science,
stated simply
``Computing is the new
literacy''~\cite{ieee_computer_special_issue_computing_education}.
%% The ``digital divide'' is no longer about citizens lacking 
%% access to computers, but about citizens lacking \emph{understanding} of how
%% they perform 
%% their tasks and how they augment human intellect.  As Rushkoff says, in
%% this environment we must either ``program or be
%% programmed''~\cite{rushkoff}.

In 1962, Tom Kurtz, a mathematics professor at Dartmouth College who was
concerned about
all of these same issues, approached his department chair John Kemeny
with a remarkably farsighted vision: Since computers were clearly going
to be important in everyday life, \emph{all} Dartmouth students should
become computer literate.
At the time,
computers were room-filling mainframes, and were so expensive
that they could only be rented, at a monthly cost typically exceeding a
professor's salary.
Universities that had computers charged students for actual
computer time used, incentivizing them to minimize usage.
Kurtz's radical proposal envisioned students
having ``unlimited'' computer access, just as with library privileges:
by simply typing their student ID number into a
computer, they could start working, and they wouldn't be billed for
time.
The proposal was even more radical considering that Dartmouth was 
a liberal arts 
college where 75\% of the students were nontechnical majors~\cite{goto}.

Like many prestigious universities, Dartmouth could mitigate the cost of
the computers themselves by cultivating good relationships with
computer manufacturers and raising funds from donors and government
grants.
But Kurtz would also need to overcome a technology problem.
Just like today, Kurtz believed everyone should be exposed to coding.
But programmers in 1962 had to prepare programs on decks of punched cards,
submit them to a technician,
% TBD picture of punched card
and come back hours or days later to pick up a printout with
the results of their program run.
(It may seem ridiculous today to wait hours or days for the results of a
running a computer program, but given the speeds of 
communication and transportation in 1962, it was normal for
business information to be a few days out of
date~\cite{ceruzzi}.) 
%% If the results were botched because of a simple error in the program,
%% the programmer would re-punch the offending cards, resubmit the job, and
%% again wait hours or days for a result.
%% This method of working, called \w{batch processing}, guaranteed the
%% computer would never be idle as long as there was a long queue of user
%% jobs waiting.
Such \emph{batch processing} favored efficient use of computer time rather
than programmer convenience, an understandable
tradeoff given the high cost of computers.
Kurtz feared that liberal arts students with no
extrinsic career motivation to learn computing would be unwilling
to endure these rituals.

Fortunately, a technical milestone saved the day.
The ability to \w[Computer multitasking]{multitask} is common on today's
PC operating systems: multiple apps can be open
simultaneously, and apps can even do work ``in the background,''
such as when a mail program
retrieves mail even while the user isn't
directly interacting with the mail program's UI.
Professor John McCarthy, then at MIT, had been part of a group working
on an experimental technology called
\w{timesharing}~\cite{corbato62timesharing}, in which a computer
switches its attention many times per second among jobs corresponding to
different users.
Because the computer could switch its attention much more quickly than
users can read or type, each user would have the illusion that the computer
was waiting only on her.  That is, the computer would feel
interactive, ``instantly'' responding to each user's commands.
McCarthy easily persuaded Kurtz and Kemeny that such interactive
use would be perfect for education because it would allow
students to get
immediate feedback, eliminating the annoying rituals of batch
processing that might discourage beginners.

Timesharing would eventually turn out to be a revolutionary
technological innovation: McCarthy even foresaw what we now call cloud
computing as a key application of multitasking.
But timesharing wasn't taken seriously in industry at that time, since batch
processing provided a more efficient use of the expensive computer's
time.  So supporting
timesharing would require building a multitasking OS, which 
was happening at MIT.

  \begin{tangent}
In fact, IBM's landmark System/360 line of computers, on which the
company had staked its future in business computing, was rather visibly
rejected by MIT in 1964 because it did not implement hardware-assisted 
address translation (i.e., virtual memory), which MIT considered
essential for supporting its experimental timesharing
OS~\cite{ibm360history}.  
That OS was called \w{Multics}, for
Multiplexed Information and Computing Service.  While Multics was
criticized for being overly complex, it pioneered ideas that influenced
virtually all operating systems that followed, including a
hierarchically-organized file system, dynamic linking of libraries, the
idea of privileged vs.\ unprivileged code (kernel vs.\ user mode), and
separate memory mapping for each process.
Some of these ideas were carried forward by researchers at Bell Labs
(which pulled out 
of the project in 1969) in its experimental OS called Unix.
Indeed, Unix
was originally spelled Unics, with the U standing humorously for
``Uniplexed''---it was a play on words suggesting the much simpler
system's suitability for only a single user.  Unix carefully selected
some of the best ideas in Multics, and ironically
became the most influential multiuser and multitasking OS.
  \end{tangent}

So Dartmouth decided to
follow MIT's lead, acquiring a General Electric computer
and creating their own timesharing
operating system for it.
Dartmouth's timesharing system was ultimately so successful that it
catapulted GE into a new timesharing business that was profitable for a
few years, and led GE to later offer Dartmouth a brand-new model
computer in exchange for collaboration on re-creating timesharing and
BASIC on it.

\smallpicfigure{figs/ASR33.jpg}{fig:ASR33}{The venerable Teletype ASR-33
printing terminal, which integrated a paper tape punch and reader.
Its distinctive uppercase-only font is used in the section headings of
this article.
You can see one in action at https://www.youtu.be/ObgXrIYKQjc---try to
imagine the noise level in a room full of these.}



%% ASR-33 internals in action: https://www.youtube.com/watch?v=11Bcfr8zBvg
%% ASR-33 in action: https://www.youtube.com/watch?v=izC9rIvVnE0

Another challenge was that timesharing would also increase costs:
if many students could use the computer at the same time, each student
would need their own terminal or ``station'' for typing in programs.  (In batch
processing, only a single terminal was needed, since it was used by the
system operator; separate card-punch machines were used by programmers
to convert their programs into punched-card decks.)
In 1964, video terminals were still a decade away.
%% and popular printing terminals
%% were slow and expensive---the
%% Friden Flexowriter printed only 10 characters per second and
%% \href{http://retrotechnology.com/herbs_stuff/flex_behr.html}{cost
%%   \$2500-4000}, or nearly \$20,000 in 2014 dollars.
Fortunately, the Teletype Corporation had recently designed a rugged and
inexpensive ``teleprinter'' for the US~Navy, the Model~ASR-33 (Automatic
Send/Receive terminal).  A teleprinter combined a keyboard with a
printer mechanism; the keyboard was used for input, and the printer for
output. 
Teletype made the ASR-33
available to the retail market in 1963 for about $1/4$ the price of
existing teleprinter terminals (\$700, or \$5,200
in 2014).
Over
half a million ASR-33s were sold before the product was discontinued in 1981.
Dartmouth acquired several ASR-33s to equip the new interactive
timesharing computer lab.

So BASIC was born just when the rapid shift from batch processing to
timesharing began, and as a result
was the first programming language to include a command that made
the program pause and wait for the user to type
something.  Such a concept made no sense in batch processing and was
absent from the high level languages such as FORTRAN and COBOL that had
been developed up to that time.
 
