
\section{CAMBRIDGE, 1975 - GATES, ALLEN, AND ALTAIR}

\begin{milestone}{The microprocessor}

\end{milestone}

% transition/tech milestone: the microprocessor - 4004, 8008, 8080

\makequotation{%
Interviewer: What do you consider your greatest achievement ever in
programming? \\
Bill Gates: I'd have to say BASIC for the 8080, because of the effect it's
had, and because of how appropriate it was at the time, and because we
managed to get it so small. It was the original program we wrote when we
decided to start Microsoft.}{Bill Gates~\cite{smithsonian_interview}} 

There's no question that Microsoft BASIC was not only the product that
launched the company, it was also Bill Gates's baby.
Paul Allen and Gates had same goal as Bob Albrecht---computing for the
masses---but not the 
same open source philosophy.  Gates realized that with so little RAM
(4KiB)  and no operating system, a BASIC compiler like Dartmouth's was
infeasible, so he opted for a BASIC interpreter with minimal 
dependencies on the underlying hardware.
In fact, Gates had
previously written a BASIC interpreter for  a DEC PDP-10 in high school and
had learned a lot from the experience.
Gates also believed that an interpreter would make the language even
easier for beginners to learn, because unlike a compiler, 
an interpreter gives immediate feedback on the programmer's work.
``\ldots I'm a big believer in interpreted languages,
not only from the beginning of computing, but the future of
computing. It was really the right approach, because you could just type
the thing in and immediately see what was happening. And yet you could
add new capabilities very easily.'' ~\cite{smithsonian_interview}

However, in the absence of any operating system, some kind of gateway
between BASIC and the hardware would be needed.  
In 1971, DEC engineers had developed a timesharing runtime system called
RSTS-11 (``Resource sharing time sharing'') for the wildly popular
PDP-11.
RSTS-11 was implemented entirely in BASIC, 
which DEC had
extended with three new commands to allow interaction with the underlying
hardware: \T{SYS} to call a machine language subroutine at a known
logical address from a BASIC
program (like a system call), and 
\T{PEEK} and \T{POKE} to query and set the contents of individual memory
addresses at the byte level (like C \texttt{unsigned char~*}
pointers)~\cite[pp.~204--205]{ceruzzi}. 
Gates adapted all three for Altair
BASIC~\cite{smithsonian_interview}, with \T{SYS} 
renamed to \T{USR} for User Service Routine, and
all three would turn out to be critical for PC adaptations of 
BASIC. 

In a feat of programming that Bill Gates
still calls his proudest moment, he squeezed a relatively featureful
and high-performance BASIC interpreter into the Altair's 4KB RAM.  
During its development, at a dinner
conversation with fellow Harvard students, Gates remarked that he didn't
know how he would handle floating-point arithmetic in his BASIC, and
fellow student Monte Davidoff chimed in with ``I can do that.''
Gates retained
Davidoff to write the floating-point math package; at the time, there
was no standard for how to handle real arithmetic in 
computer languages, so Davidoff designed his own scheme, probably based
on the techniques that were used by the popular DEC computers of that time.
Squeezing a well-featured BASIC that included real arithmetic into just
4096 bytes of 8080 assembly language was a feat indeed, and Gates later
recounted that the extensive tuning
gave the authors confidence of the superiority of their
work~\cite{programmers_at_work}.  

  \begin{geeknote}
  A floating-point math standard would finally be developed by a
  committee convened by Intel 
  and other vendors and then endorsed by the Institute of Electrical and
  Electronics Engineers (IEEE), a vendor-neutral professional and scholarly
  society.  \w{William Kahan}, a professor at UC~Berkeley, ultimately
  received the 
  Turing Award for his work on this difficult
  problem~\cite{kahan_interview}, which has become codified as IEEE
  standard number 754.
  \end{geeknote}

Gates and Allen faced a cultural obstacle in trying to monetize their
BASIC.  
In early days of computing, hardware was  ``where the action
was'' and software was an afterthought---something you learned to
do in order to use your nifty new hardware.
Programming was seen as glorified grunge work, not as the separate
economic driver it has since become.
Hence, the knowledge of how to do programming was viewed as something to be
shared freely, since the belief was that the hardware was the
competitive/proprietary advantage.
In an environment dominated by this
``share-alike'' mindset, it was
tricky to sell individual copies of software: one person would buy a
copy of Altair BASIC and then simply make copies to give to friends.
Gates
expressed his disdain at the ``software pirates'' (first use of term??)
in an infamous 
1976 ``open letter to computer hobbyists'' in which he exhorted them, in
fairly pointed language, not to steal
software.  Notwithstanding the hue and cry generated by the letter,
which was published in the MITS newsletter (?), 
Gates quickly figured out that a better solution was to license BASIC to
computer manufacturers, who would burn it into a ROM chip in each computer
sold.  This pattern would be repeated later with Windows: computer
manufacturers would license it and  preinstall it on new PCs,  rather
than relying on the purchaser to buy and install Windows separately.

Gates and Allen moved to Albuquerque, New Mexico, home of MITS
computers, and set up shop as Micro-Soft [sic].  BASIC was the product
that launched the company.
