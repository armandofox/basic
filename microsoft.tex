
\section{CAMBRIDGE, 1975 - GATES, ALLEN, AND ALTAIR}

\begin{milestone}{The microprocessor}

\end{milestone}

% transition/tech milestone: the microprocessor - 4004, 8008, 8080

\makequotation{%
Interviewer: What do you consider your greatest achievement ever in
programming? \\
Bill Gates: I'd have to say BASIC for the 8080, because of the effect it's
had, and because of how appropriate it was at the time, and because we
managed to get it so small. It was the original program we wrote when we
decided to start Microsoft.}{Bill Gates~\cite{smithsonian_interview}} 

Allen and Gates had same goal as Bob Albrecht---computing for the
masses---but not the 
same open source mentality.  In a feat of programming that Bill Gates
says is still his proudest moment, he squeezed a relatively featureful
and performant interpreter into the Altair's 4KB RAM.  (Gates had
previously written a BASIC interpreter for  a DEC PDP-10 in high school and
had learned a lot from the experience.)  With 4KB and the nonexistent
OS/runtime, an interpreter would ahve to be the way to go, even though
Dartmouth BASIC had always been load-and-go compiled.

In 1971, DEC engineers had developed a timesharing runtime system called
RSTS-11 (``Resource sharing time sharing'') for the newly-announced
PDP-11.  
RSTS-11 was implemented entirely in BASIC, 
which DEC had
extended with three new commands to allow interaction with the underlying
machine: \T{SYS} to call a machine language routine (system call), and
\T{PEEK} and \T{POKE} to query and set the contents of individual memory
locations (like C pointers)~\cite[pp.~204--205]{ceruzzi}.
All three appeared in Gates's original Altair
BASIC~\cite{smithsonian_interview}, with \T{SYS} 
renamed to \T{USR} for User Service Routine, and
all would turn out to be critical for PC adaptations of 
BASIC. 


``\ldots I'm a big believer in interpreted languages,
not only from the beginning of computing, but the future of
computing. It was really the right approach, because you could just type
the thing in and immediately see what was happening. And yet you could
add new capabilities very easily.'' ~\cite{smithsonian_interview}


Gates retained
Monte Davidoff to write the floating-point math package.
At the time, there was no standard for how to handle real arithmetic in
computer languages, so Davidoff designed his own scheme, probably based
on the techniques used by the popular DEC computers of that time and
which would eventually inform the industry standard.
Squeezing a well-featured BASIC that included real arithmetic into just
4096 bytes of 8080 assembly language was a feat indeed, and Gates later
recounted that the extensive tuning required to squeeze everything in
gave the authors confidence that no one could improve on their
work~\cite{programmers_at_work}. 

  \begin{geeknote}
  A standard would finally be developed by a committee convened by Intel
  and other vendors and then endorsed by the Institute of Electrical and
  Electronics Engineers, a vendor-neutral professional and scholarly
  society, as IEEE~754.  \w{William Kahan} ultimately received the
  Turing Award for his work on this difficult problem~\cite{kahan_interview}.
  \end{geeknote}

In early days of computers, HW was the real thing and SW was an
afterthought.  Programming was seen as glorified clerical/grunge work.
Hence, the knowledge of how to do it was viewed as something to be
shared freely, since the belief was that the hardware was the
competitive/proprietary advantage.
Gates would learn that in an environment dominated by this
``share-alike'' mindset, it was
tricky to sell individual copies of software and expect people not to
copy them.  Gates
expressed his disdain at the ``software pirates'' (first use of term??) in infamous
1976 open letter.
Gates quickly figured out that it was better to license BASIC to
computer manufacturers to place in ROM, similar to how Windows would be
licensed later to be preinstalled, rather than sell copies to
hobbyists.  Radio Shack and TI were the first two, with Apple shortly
thereafter.  

From CBASIC to visual BASIC....


First popular BASIC game: startrek.bas?
  - on non-XY-addressable text displays

Getting off the ground: Microsoft BASIC and the Altair 8800.  Why was
    BASIC chosen?


How did constraints of BASIC affect programs written:
- for business?
- for entertainment?



Platform constraints as virtues:
 - ZX80 "Hampson's Plane" uses blackout during display generation 
 - ZX80 and other BASICs where "Save" saves BASIC memory image incl
 variable values (save game?)
