
\section{CAMBRIDGE 1975 - GATES - ALLEN - ALTAIR}

% transition/tech milestone: the microprocessor - 4004, 8008, 8080

\makequotation{%
Interviewer: What do you consider your greatest achievement ever in
programming? \\
Bill Gates: I'd have to say BASIC for the 8080, because of the effect it's
had, and because of how appropriate it was at the time, and because we
managed to get it so small. It was the original program we wrote when we
decided to start Microsoft.}{Bill Gates~\cite{smithsonian_interview}} 

Gates, then a Harvard undergraduate, and his colleague 
Paul Allen, who
had dropped out of college and was working as a programmer for Honeywell
in Boston, had shared an enthusiasm for computing since becoming
friends in middle school.
Allen persuaded Gates that personal computers were coming quickly, and
they quickly arrived at the same goal as Bob Albrecht---computing for the 
masses---but didn't necessarily share Albrecht's open source philosophy.  

In fact, Gates had
previously written a BASIC interpreter for  a DEC PDP-10 in high school
as a side project, and also believed interpreters were better for
learning because
beginners could get truly instant feedback without going through an
edit-compile-run cycle:
``\ldots I'm a big believer in interpreted languages,
not only from the beginning of computing, but the future of
computing \ldots because you could just type
the thing in and immediately see what was
happening.''~\cite{smithsonian_interview} 
And interpreters can be implemented in less memory than compilers,
an important advantage given the paltry 4KB~RAM in the entry-level  Altair.
(PCs with more than 4KB RAM would not arrive until 1977.)

Gates began creating a full-featured BASIC interpreter for the
newly-announced Altair~8800 and the other personal computers that he
knew would soon follow, implemented in Intel 8080 assembly language, as
a part-time project while at Harvard.
When Gates remarked during a dinner conversation at Harvard
that he didn't know how to handle floating-point arithmetic in his
BASIC, fellow student Monte Davidoff chimed in with ``I can do that.''
At the time, there was no standard for implementing floating-point in
computer languages, so Davidoff designed his own scheme, probably based
on the techniques used by the popular DEC computers of that time.
Squeezing a well-featured BASIC that included floating-point arithmetic
into just 
4096 bytes of 8080 assembly language was a feat indeed, and Gates later
recounted that the extensive tuning
gave the authors confidence of the superiority of their
work~\cite{programmers_at_work}.  

BASIC's original creators had added \T{NEW} and \T{OLD} commands to the
language to
provide some minimal access to underlying operating system features, in
this case access to the filesystem,
without requiring beginners to learn an entire operating system.
How would Gates's BASIC interact with the underlying hardware of 
the Altair computer, which \emph{had} no operating
system?
Fortunately inspiration was near at hand:
In 1971, DEC had had to solve the same problem with a
timesharing OS called RSTS-11 (``Resource sharing time
sharing'') for the wildly popular PDP-11 computer.
RSTS-11 was implemented entirely in BASIC, so DEC had added
three new commands to allow interaction between BASIC programs and the
hardware: \T{SYS} to make a system call to a
known logical address from a BASIC program, and
\T{PEEK} and \T{POKE} to query and set the contents of individual memory
bytes (like C \texttt{unsigned char~*}
pointers)~\cite[pp.~204--205]{ceruzzi}.
Gates put all three into Altair BASIC~\cite{smithsonian_interview},
with \T{SYS} renamed to \T{USR} for User Service Routine; all three
would also turn out to be critical for PC adaptations of BASIC.

In a feat of programming that Bill Gates still calls his proudest
moment, he squeezed a relatively featureful and high-performance BASIC
interpreter into the Altair's paltry 4K~RAM.
The story of how Allen got on a plane to Albuquerque, New Mexico, where
he gave a successful demo to MITS computers and quickly set up shop as
Micro-Soft [sic], has been well told: Microsoft BASIC was not only the
product that launched a juggernaut company, it was also Bill Gates's
baby.


  \begin{tangent}
  The problem of doing principled
  floating-point math was so tough that Intel 
  and other chip vendors eventually convened a technical committee to
  research the problem 
  and suggest a solution. UC~Berkeley professor \w{William Kahan}
  ultimately received the 
  Turing Award for
  leading the effort~\cite{kahan_interview}, which was codified as
  standard number 754 by the  Institute of Electrical and 
  Electronics Engineers (IEEE), a vendor-neutral professional and scholarly
  society. 
  \end{tangent}

Gates and Allen now faced a cultural obstacle in trying to sell their
BASIC.
In early days of computing, hardware was ``where the action was'' and
software was an afterthought---something you learned to do in order to
use your nifty new hardware.
Programming was seen as glorified grunge work, not as the separate
economic driver it was soon to become.
Hence, the knowledge of how to do software was viewed as something to be
shared freely, since the belief was that the hardware was where the
competitive/proprietary advantage lay.
The ``share-alike'' mindset was further strengthened by the influential
role of the California counterculture in the early days of PCs.  In the
environment that birthed the People's Computer Company and the Homebrew
Computer Club, where a young Steve Wozniak would
eventually demonstrate the computer that would become the Apple I, it
was considered socially acceptable for one person to
acquire a piece of software and simply make copies to give to friends.
Gates expressed his disdain at the ``software pirates'' 
%% (first use of term??)
in an infamous 1976 ``open letter to computer hobbyists'' that ran in
the MITS newsletter, which all Altair owners received.
In fairly pointed language, Gates scolded  Altair owners that by freely
circulating paper tapes of his BASIC, they were effectively stealing his
work and making it financially infeasible to start companies that could
develop great software.
Notwithstanding the hue and cry generated by the letter, which was
published in the MITS newsletter, Gates quickly figured out that a
better solution was to license BASIC to computer manufacturers, who
would burn it into a ROM chip in each computer sold.
This pattern would be repeated later with Windows, which would be
licensed by computer manufacturers to preinstall on new PCs, rather than
purchased separately by the end user.




