
\section{CAMBRIDGE 1975 - GATES - ALLEN - ALTAIR}

% transition/tech milestone: the microprocessor - 4004, 8008, 8080

\makequotation{%
\emph{Interviewer:}\/ What do you consider your greatest achievement ever in
programming? \\
\emph{Bill Gates:}\/ I'd have to say BASIC for the 8080, because of the effect it's
had, and because of how appropriate it was at the time, and because we
managed to get it so small. It was the original program we wrote when we
decided to start Microsoft.}{1994 interview with Bill Gates~\cite{smithsonian_interview}} 

Gates, a Harvard undergraduate, and his colleague 
Paul Allen, a college dropout working as a programmer for Honeywell
in Boston, had shared an enthusiasm for computing since they were
teenagers growing up together in Seattle.
When the Altair 8800 was announced,
Allen persuaded Gates that personal computers were coming quickly.
Like Bob Albrecht, the pair immediately realized that all those
computer owners would want to write programs, there being no serious third-party
software industry at that time.
But where Albrecht saw a way to extend the ``share alike''
counterculture via open source, Allen and Gates saw a business opportunity.

Gates had
previously written a BASIC interpreter for  a DEC \w{PDP-10}
minicomputer 
as a  high school project.  Gates believed interpreters were better for
learning because of the instant feedback they gave:
``[y]ou could just type
the thing in and immediately see what was
happening.''~\cite{smithsonian_interview} 
And interpreters can be implemented in less memory than compilers,
an important advantage given the paltry 4KB~RAM in the entry-level  Altair.

So Gates and Allen began writing a BASIC interpreter in Intel 8080
assembly language for the
Altair~8800 and the other personal computers they
knew would soon follow.
When Gates remarked during a dinner conversation at Harvard
that he didn't know how to handle floating-point arithmetic in his
BASIC, fellow student Monte Davidoff chimed in with ``I can do that.''
At the time, there was no standard for implementing floating-point in
computer languages, so Davidoff designed his own scheme, probably based
on the techniques used by the DEC~PDP-10 and other DEC minicomputers of
that time.

  \begin{tangent}
  Implementing
  floating-point arithmetic in a rigorous manner was so tough that Intel 
  and other chip vendors eventually convened a technical committee to
  research the problem, under the aegis of the Institute of Electrical and 
  Electronics Engineers (IEEE), a vendor-neutral professional and scholarly
  society. 
  UC~Berkeley professor \w{William Kahan}
  ultimately received the 
  Turing Award for
  leading the effort~\cite{kahan_interview}, whose results were codified as
  the \w[IEEE floating point]{IEEE-754} standard in 1985.
  \end{tangent}

Gates faced new problems in adapting BASIC to PCs.
How could his BASIC interact with the underlying hardware of 
the Altair computer, which had no operating
system?
Fortunately inspiration was near at hand:
In 1971, DEC had had to solve a similar problem 
on the wildly popular PDP-11 computer, successor to the PDP-10 Gates had
programmed in high school.
The PDP-11 had a 
timesharing OS called \w[RSTS/E]{RSTS} (``Resource sharing, time
sharing'') much of which was implemented in BASIC\footnote{This is not as
  ridiculous as it sounds given that BASIC was compiled.  After all,
   Unix is implemented largely in C.}, so DEC had added
three new commands to allow interaction between BASIC programs and the
hardware: \T{SYS} to make a system call to a
known logical address from a BASIC program, and
\T{PEEK} and \T{POKE} to query and set the contents of individual memory
bytes (like C \texttt{unsigned char~*}
pointers)~\cite[pp.~204--205]{ceruzzi}.
Gates put all three into Altair BASIC~\cite{smithsonian_interview},
with \T{SYS} renamed to \T{USR} for User Service Routine; all three
survived in virtually every PC adaptation of BASIC.

In an impressive feat of programming, Gates squeezed a 
feature-laden and high-performing BASIC interpreter into the Altair's
paltry 4K~RAM.
Gates later recounted that the extensive tuning gave the authors
confidence of the superiority of their work~\cite{programmers_at_work}.
The feat is even more impressive considering that none of the team had
access to an actual 8080 or Altair computer: the whole project was done
using an 8080 emulator Gates had written to run on Harvard's student
computer system, and it wasn't until they loaded the code into actual
Altair hardware during their first demo at the MITS offices  in Albuquerque
that they knew conclusively that it worked.
The high-stakes demo was a success, and Allen quickly set up shop in
Albuquerque as Micro-Soft [sic].
Microsoft BASIC was not only the product that launched a juggernaut
company, it was also Bill Gates's baby.

\begin{tangent}
Not only had the BASIC interpreter never been tested on real hardware:
during the flight to Albuquerque, Allen realized they hadn't thought
about how to load the interpreter code into the Altair to begin with.
The Altair had no BIOS or boot ROM, so while on the plane, Allen wrote a
simple ``boot loader'' program that, when entered manually into the
Altair by toggling switches to enter one assembly-language instruction
at a time, would be able to read the rest of the interpreter code from
paper tape.  Just-in-time programming, indeed!
\end{tangent}

But Gates and Allen now faced a cultural obstacle in trying to sell their
BASIC.
In early days of computing, the excitement and innovation was around hardware.
Software was an afterthought---something you learned to do in order to
use your nifty new hardware, or if you were a computer company,
something you gave away for free to stimulate hardware sales.
The ``share-alike'' mindset promulgated by Albrecht and his California
colleagues wasn't helping:
in Palo Alto's Homebrew
Computer Club, where a young Steve Wozniak would
eventually demonstrate the Apple~I, it
was considered socially acceptable to
make copies of software  to give your friends---it even enhanced
your standing as a club member if you provided something useful.
The idea of a separate software industry, in which
people paying for software could be a major economic driver of the PC era,
was simply not in most people's sights.  

But Bill Gates was not most people: in 1976 he laid out just this vision,
perhaps somewhat abrasively, in a now-infamous ``open letter to computer
hobbyists'' published in the 
MITS newsletter that all Altair owners received.
In pointed language, Gates railed against the ``software pirates'' who
circulated paper tape copies of Micro-Soft BASIC, complaining that they
were stealing 
his work and making a third-party software industry financially
infeasible.
The culture clash could not have been greater.
Notwithstanding the hue and cry generated by the letter, Gates
quickly figured out that a better solution was to license BASIC directly
to computer manufacturers, who would burn it into the ROM of each
computer sold, much as a BIOS is today.
(This pattern would be repeated later with Windows, which would be
licensed by computer manufacturers to preinstall on new PC hard drives,
rather than purchased separately by the end user.)

So Bill Gates successfully got BASIC running on the first personal computer, but that computer was only
accessible to hobbyists who could solder and who would tolerate using
something that looked more like a piece of lab equipment.  The Altair had no way
to accept or display text, let alone graphics, without adding additional
hardware. 
Gates knew that a ``ready-to-run'' computer was inevitable, but he was a
software designer.  Fortunately, a hardware designer who had also been
bitten by the BASIC bug was taking up the challenge---not in northern
California, nor even in Boston, but in the unlikely town of Valley Forge,
Pennsylvania, where General George Washington had pulled his army
through the bitter winter of the American Revolution nearly 200 years before.


