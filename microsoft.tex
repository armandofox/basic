
\section{CAMBRIDGE 1975 - GATES AND ALLEN AND ALTAIR}

% transition/tech milestone: the microprocessor - 4004, 8008, 8080

\makequotation{%
Interviewer: What do you consider your greatest achievement ever in
programming? \\
Bill Gates: I'd have to say BASIC for the 8080, because of the effect it's
had, and because of how appropriate it was at the time, and because we
managed to get it so small. It was the original program we wrote when we
decided to start Microsoft.}{Bill Gates~\cite{smithsonian_interview}} 

Microsoft BASIC was not only the product that
launched a juggernaut company---it was also Bill Gates's baby.
Paul Allen and Gates had same goal as Bob Albrecht---computing for the
masses---but didn't share Albrecht's open source philosophy.  Gates
wanted to create a full-featured version of BASIC that could run on the
newly-announced Altair and, he assumed, other personal computers that
would soon follow.
Gates realized that with so little RAM 
(4KiB)  and no operating system, a BASIC compiler like Dartmouth's was
infeasible on the Altair, so he opted for a BASIC interpreter with minimal 
dependencies on the underlying hardware.
In fact, Gates had
previously written a BASIC interpreter for  a DEC PDP-10 in high school and
had learned a lot from the experience.
Gates also believed that an interpreter would make the language even
easier for beginners to learn, because unlike a compiler, 
an interpreter gives immediate feedback on the programmer's work.
``\ldots I'm a big believer in interpreted languages,
not only from the beginning of computing, but the future of
computing. It was really the right approach, because you could just type
the thing in and immediately see what was
happening.''~\cite{smithsonian_interview} 

BASIC's original creator had added \T{NEW} and \T{OLD} commands to
provide some minimal access to underlying operating system features
without requiring beginners to learn an entire operating system.
In Gates's case, with the Altair computer, there \emph{was} no operating
system, so some kind of gateway between BASIC and the hardware would be
needed.
Fortunately inspiration was near at hand.
In 1971, DEC (Digital Equipment Corporation) engineers had developed a
timesharing runtime system called RSTS-11 (``Resource sharing time
sharing'') for the wildly popular PDP-11 computer.
RSTS-11 was implemented entirely in BASIC, which DEC had extended with
three new commands to allow interaction between BASIC programs and the
underlying hardware: \T{SYS} to call a machine language subroutine at a
known logical address from a BASIC program (like a system call), and
\T{PEEK} and \T{POKE} to query and set the contents of individual memory
addresses at the byte level (like C \texttt{unsigned char~*}
pointers)~\cite[pp.~204--205]{ceruzzi}.
Gates adapted all three for Altair BASIC~\cite{smithsonian_interview},
with \T{SYS} renamed to \T{USR} for User Service Routine, and all three
would turn out to be critical for PC adaptations of BASIC.

In a feat of programming that Bill Gates still calls his proudest
moment, he squeezed a relatively featureful and high-performance BASIC
interpreter into the Altair's paltry 4K~RAM.
The story of how Allen got on a plane to Albuquerque, New Mexico, where
he gave a successful demo to MITS computers and quickly set up shop as
Micro-Soft [sic], has been well told: BASIC was the product that
launched the company.

When Gates remarked during a dinner
conversation at Harvard that he didn't
know how to handle floating-point arithmetic in his BASIC,
fellow student Monte Davidoff chimed in with ``I can do that.''
Nonspecialists may find it surprising that it's tricky for computers to handle
floating-point
numbers in computers, but it's true; the basic problem is that between
any two real numbers (say 1.0 and 2.0) there is an infinite number of
real numbers, but a finite number of bits in which to represent them.
At the time, there was no standard for how to do this in
computer languages, so Davidoff designed his own scheme, probably based
on the techniques used by the popular DEC computers of that time.
Squeezing a well-featured BASIC that included floating-point arithmetic
into just 
4096 bytes of 8080 assembly language was a feat indeed, and Gates later
recounted that the extensive tuning
gave the authors confidence of the superiority of their
work~\cite{programmers_at_work}.  

  \begin{geeknote}
  The problem of doing principled
  floating-point math was so tough that Intel 
  and other chip vendors convened a committee to research the problem
  and suggest a solution. UC~Berkeley professor \w{William Kahan}
  ultimately received the 
  Turing Award for
  leading the effort~\cite{kahan_interview}, which was codified as
  standard number 754 by the  Institute of Electrical and 
  Electronics Engineers (IEEE), a vendor-neutral professional and scholarly
  society. 
  \end{geeknote}

Gates and Allen faced a cultural obstacle in trying to monetize their
BASIC.
In early days of computing, hardware was ``where the action was'' and
software was an afterthought---something you learned to do in order to
use your nifty new hardware.
Programming was seen as glorified grunge work, not as the separate
economic driver it was soon to become.
Hence, the knowledge of how to do software was viewed as something to be
shared freely, since the belief was that the hardware was where the
competitive/proprietary advantage lay.
In an environment dominated by this ``share-alike'' mindset,
strengthened by the counterculture ecosystem that birthed the People's
Computer Company and the Homebrew Computer Club in northern California,
it was tricky to sell individual copies of software: the first person to
buy a copy of Altair BASIC would simply make copies to give to friends.
Gates expressed his disdain at the ``software pirates'' %% (first use of
term??)
in an infamous 1976 ``open letter to computer hobbyists'' in which he
exhorted them, in fairly pointed language, not to steal software.
Notwithstanding the hue and cry generated by the letter, which was
published in the MITS newsletter, Gates quickly figured out that a
better solution was to license BASIC to computer manufacturers, who
would burn it into a ROM chip in each computer sold.
This pattern would be repeated later with Windows, which would be
licensed by computer manufacturers to preinstall on new PCs, rather
than purchased separately by the end user.

