
\section{CAMBRIDGE 1975 - GATES - ALLEN - ALTAIR}

% transition/tech milestone: the microprocessor - 4004, 8008, 8080

\makequotation{%
Interviewer: What do you consider your greatest achievement ever in
programming? \\
Bill Gates: I'd have to say BASIC for the 8080, because of the effect it's
had, and because of how appropriate it was at the time, and because we
managed to get it so small. It was the original program we wrote when we
decided to start Microsoft.}{Bill Gates~\cite{smithsonian_interview}} 

Gates, a Harvard undergraduate, and his colleague 
Paul Allen, who
had dropped out of college and was working as a programmer for Honeywell
in Boston, had shared an enthusiasm for computing since becoming
friends in middle school in Seattle.
When the Altair 8800 was announced,
Allen persuaded Gates that personal computers were coming quickly, and
like Bob Albrecht, the pair immediately realized that all those
computer owners would want to write programs, there being no third-party
software industry \emph{per~se}\/ at that time.
But Gates and Allen didn't necessarily share Albrecht's open-source
philosophy.   

Gates had
previously written a BASIC interpreter for  a DEC PDP-10 in high school
as a side project, and also believed interpreters were better for
learning because
``you could just type
the thing in and immediately see what was
happening.''~\cite{smithsonian_interview} 
And interpreters can be implemented in less memory than compilers,
an important advantage given the paltry 4KB~RAM in the entry-level  Altair.
(Ready-to-run assembled PCs with more than 4KB RAM would not arrive until 1977.)

Gates and Allen began writing a BASIC interpreter in Intel 8080
assembly language for the
Altair~8800 and the other personal computers they
knew would soon follow.
When Gates remarked during a dinner conversation at Harvard
that he didn't know how to handle floating-point arithmetic in his
BASIC, fellow student Monte Davidoff chimed in with ``I can do that.''
At the time, there was no standard for implementing floating-point in
computer languages, so Davidoff designed his own scheme, probably based
on the techniques used by the popular DEC minicomputers of that time.

  \begin{tangent}
  The problem of doing principled
  floating-point math was so tough that Intel 
  and other chip vendors eventually convened a technical committee to
  research the problem.
  UC~Berkeley professor \w{William Kahan}
  ultimately received the 
  Turing Award for
  leading the effort~\cite{kahan_interview}, which was codified as
  standard number IEEE-754 by the  Institute of Electrical and 
  Electronics Engineers (IEEE), a vendor-neutral professional and scholarly
  society. 
  \end{tangent}

Gates faced new problems in adapting BASIC to PCs.
How could his BASIC interact with the underlying hardware of 
the Altair computer, which had no operating
system?
Fortunately inspiration was near at hand:
In 1971, DEC had had to solve the same problem with a
timesharing OS called RSTS-11 (``Resource sharing time
sharing'') for the wildly popular PDP-11 computer.
RSTS-11 was implemented entirely in BASIC\footnote{This is not as
  ridiculous as it sounds given that BASIC was compiled.  After all,
   Unix is implemented largely in C.}, so DEC had added
three new commands to allow interaction between BASIC programs and the
hardware: \T{SYS} to make a system call to a
known logical address from a BASIC program, and
\T{PEEK} and \T{POKE} to query and set the contents of individual memory
bytes (like C \texttt{unsigned char~*}
pointers)~\cite[pp.~204--205]{ceruzzi}.
Gates put all three into Altair BASIC~\cite{smithsonian_interview},
with \T{SYS} renamed to \T{USR} for User Service Routine; all three
survived in virtually every PC adaptation of BASIC.

In an impressive feat of programming, Gates squeezed a 
feature-laden and high-performing BASIC interpreter into the Altair's
paltry 4K~RAM.
Gates later recounted that the extensive tuning gave the authors
confidence of the superiority of their work~\cite{programmers_at_work}.
The feat is even more impressive considering that none of the team had
access to an actual 8080 or Altair computer: the whole project was done
using an 8080 emulator Gates had written to run on Harvard's student
computer system, and it wasn't until they loaded the code into actual
Altair hardware during their first demo at the MITS offices 
that they knew conclusively that it worked.
The high-stakes demo was a success, and Allen quickly set up shop in
Albuquerque, where MITS was located, as Micro-Soft [sic].
Microsoft BASIC was not only the product that launched a juggernaut
company, it was also Bill Gates's baby.

But Gates and Allen now faced a cultural obstacle in trying to sell their
BASIC.
In early days of computing, hardware was ``where the action was'' and
software was an afterthought---something you learned to do in order to
use your nifty new hardware, or if you were a computer company,
something you gave away for free to stimulate the sales of hardware.
The ``share-alike'' mindset promulgated by Albrecht and his California
colleagues wasn't helping:
in Palo Alto's Homebrew
Computer Club, where a young Steve Wozniak would
eventually demonstrate the Apple~I, it
was considered socially acceptable to
make copies of software  to give your friends.
The idea of a separate software industry, where
people paying for software could be a major economic driver of the PC era,
was simply not in most people's sights.  

But Bill Gates was not most people: in 1976 he laid out just this vision
in a now-infamous ``open letter to computer hobbyists'' published in the
MITS newsletter, which all Altair owners received.
In fairly pointed language, Gates railed at the ``software pirates'' who
circulated paper tape copies of Micro-Soft BASIC that they were stealing
his work and making it financially infeasible to start companies that
could pay people to develop great software.
The culture clash could not have been greater.
Yet notwithstanding the hue and cry generated by the letter, Gates
quickly figured out that a better solution was to license BASIC directly
to computer manufacturers, who would burn it into the ROM of each
computer sold, much as a BIOS is today.
(This pattern would be repeated later with Windows, which would be
licensed by computer manufacturers to preinstall on new PC hard drives,
rather than purchased separately by the end user.)

Bill Gates got BASIC onto the first personal computer, but it was only
accessible to hobbyists who could solder and who would tolerate using
something that looked more like a piece of lab equipment and had no way
to accept or display text, let alone graphics.
Gates knew that a ``ready-to-run'' computer was inevitable, but he was a
software writer.  Fortunately, a hardware designer who had also been
bitten by the BASIC bug took up the challenge.



