\begin{tabular}{|l|l|}

\hline
\B{Keyword} & \B{meaning}   \\ \hline

\T{LET} \emph{var}=\emph{expr} &  assign expression to variable  \\
\T{PRINT} \ldots & output results \\
\T{IF} \emph{cond} \T{THEN} \emph{linenum} & conditional (no \T{ELSE}) \\
\T{GOTO} \emph{linenum} & jump to another statement \\
\T{GOSUB} \emph{linenum} & call subroutine \\
\T{RETURN} & return from subroutine    \\
\T{FOR} \emph{var}=\emph{min} \T{TO} \emph{max}  & loop using \emph{var} as index \\
\T{NEXT} \emph{var} & mark the end of a \T{FOR} loop \\
 \T{DEF} \T{FN}\emph{var}(\emph{arg}) & define single line functions               \\      
 \T{DIM} \emph{var}(\emph{size}) & declare dimension (size) of array          \\           
 \T{DATA}  \ldots & store static data within the program \\                                
 \T{READ}  \emph{var[,var\ldots]} & consume data in \T{DATA} statements \\                 
 \T{REM} \emph{anything} \ldots & comment \\                                               
 \T{STOP} & stop execution before textual end                             \\               
 \T{END} & end of program  \\                
\hline
\end{tabular}
