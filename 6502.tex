
\section{VALLEY FORGE, 1975 - THE 6502}

Chuck Peddle had been working at General Electric when BASIC was
born at Dartmouth.  He had since moved on to chip maker Motorola, whose 6800
microprocessor chip competed with the Intel 4004 and
8008 (the first microprocessors, originally developed for
electronic calculators).  The 6800 sold successfully for \$300, but
Peddle  believed that with a streamlined design they could
produce a competitive product that could sell for just \$20,
allowing Motorola to ``put
microprocessors everywhere''~\cite[p. 31]{commodore}---microwave ovens,
cars, children's toys---creating whole new categories of Motorola customers.

But Motorola was skeptical and uninterested, so Peddle and a few of his close
colleagues started their own company, MOS Technology\footnote{MOS stands
  for metal oxide 
semiconductor, a phrase that describes the breakthrough technology that
made integrated circuits (computer chips) possible.}.
Through Peddle's years
of experience with how these chips were used in ``real world'' settings,
he streamlined the 6800's design 
down to just 4300 transistors, resulting in a smaller chip that was
cheaper to manufacture (since manufacturing costs were dominated by the
ceramic or plastic 
package (picture) the chip was mounted in).
The simple design was created and tested essentially by hand,
%%  laid out and hand-drawn by hand by Bill Mensch,
%% hand-simulated by the other engineers, and hand-tested on a jury-rigged
%% chip testing apparatus, 
since the young company could not afford the
fancy automation used by their competitors.
Peddle's colleague John Paivinen, an expert in the chemical
and physical manufacturing processes used to produce the chips, had
invented a technology\footnote{Contactless mask liners.} that allowed
the ``master'' from which chips are ``stamped'' to last indefinitely,
rather than wearing out gradually as more chips are produced.
And by carefully tuning the manufacturing process,
typically 70\% of the 6502 chips coming off MOS Technology's assembly
line worked correctly, compared to around 30\% for competitors.

The result was that MOS Technology could manufacture the chip for \$12
and sell it for \$20 in single quantities.  The 6502 was priced like
the simple Intel 4004 and 8008 but was faster than the \$360 Motorola
6800 and comparable to the
\$360 Intel 8080 (\$75
wholesale~\cite[p. 228]{ceruzzi})  used by the Altair computer.
The industry magazine \emph{Electronic
  Engineering Times} was skeptical that the company could provide good
technical support at such a low price, but a review in the influential
hobbyist magazine \emph{Byte}~\cite{byte75:6502} caught the attention of
Steve Wozniak, who waited in line  to buy
a 6502 at the Wescon trade show where it was introduced,
hoping to use it in the computer he was
designing for the Homebrew Computer Club meeting in Palo Alto.  

\begin{geeknote}
  Peddle cannily made the 6502 pin-compatible and largely
  assembly-code-compatible (but 
  binary-incompatible) with the 6800 to take advantage of 
  6800 hobbyist kits and existing motherboards, but a lawsuit from
  Motorola soon put a stop to that.
\end{geeknote}


%% In 1974, HP introduced the HP-65 programmable calculator, Intel
%% announced the 8080, and (in December) \emph{Popular Electronics} had the
%% Altair 8800 on the cover.  


\begin{geeknote}
    The 6502 reflected good engineering taste in what to leave out.
    Compared to the 6800, the 6502 lacks the second accumulator, the
    tristate drivers for the address bus (so you can't do DMA, which was seen as
    unnecessary for small systems~\cite{byte75:6502}), 
    and the 6800's 16-bit index register.  In exchange, the 6502 provides a pair
    of 8-bit index registers that enables an additional kind of
    indirect addressing with a true index register offset.
    This design decision, based on designer Chuck Peddle's experience with real
    code and his knowledge of the PDP-11 instruction set, 
    reduces instruction count and execution time by up to a
    factor of 2 over the 6800 for referencing arrays using random
    offsets and for operations that
    maintain pointers into two arrays.
    As a result, the 6502
    beat almost all its competitors on  the AH Systems benchmarks of
    the day~\cite{edn75:6502}, despite selling for a fraction of the price.

    Since RAM could be accessed in one
    clock cycle (the 6502 ran at
    1~MHz) , RAM was used as overflow registers instead of devoting valuable die
    space to registers.  The ``zero
    page'' (bytes 0x0000 to 0x00ff in the 6502's 16-bit address space) had
    special load and store instructions that were one byte shorter to fetch and
    decode, since the high-order address byte was zero.
    The 6502 was the first pipelined
    microprocessor, giving it a higher instructions-per-clock 
    rate than comparable processors and making
    common instructions such as relative
    branch and subroutine calls 1 to 3 cycles faster than
    on comparable microprocessors.  

    Later, the derivative 6507 would
    remove some other features and limit
    physical memory to a 14-bit address space.  These changes allowed it to
    use a 28-pin 
    plastic package vs. 40-pin ceramic package and to be sold for \$5,
    making it the chip that was chosen for the Atari 2600.
\end{geeknote}

With its hobbyist-friendly price and solid performance, the 6502
really kicked the PC revolution into full swing.
By the time the 6502 was on the scene, Microsoft BASIC already had
``design wins'' on 8080 devices, including the MITS Altair and
(curiously) smart terminals from NCR and Data Terminal
Corporation~\cite[p. 96]{gates}, and had committed to do a version for
the Motorola 6800.  Gates disdained the 6502, which he considered
underpowered compared to the 8080 and other microprocessors, but was
soon convinced that there was a sales opportunity in modifying his BASIC
to work with 6502-based systems.

\begin{figure}
  \begin{tabular}{cccccc}
 \B{Licensor \& machine} & \B{Format} & \B{Date} & \B{Terms} & \B{Moniker} & \B{Units sold}  \\ \hline
Altair MITS (Altair BASIC) & Paper tape &  &   &    &    \\
Commodore PET (Commodore BASIC)  & ROM  &  1977 &   &    \\
Apple II (Applesoft BASIC) & ROM (8K)  & 1977 & \$22,000 & 1M+  \\
Texas Instruments TI-99/4 & & \$100,000 &    \\
Radio Shack TRS-80 (Level II BASIC) & ROM (12K), tape, disk  & \$50,000 &   \\
General Electric & 1976 & \$50,000 & Internal use \\
IBM PC 5150 (IBM BASIC) & ROM () & 1981 &  \\
Apple Macintosh (Microsoft BASIC for Mac) & disk & 1984 &   \\

\hline
  \end{tabular}
\caption{\label{fig:basic_licenses}
Table of Microsoft BASIC licenses~\cite[Chapter 8ff]{gates}}
\end{figure}
