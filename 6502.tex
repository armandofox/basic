
\section{VALLEY FORGE, 1975 - THE 6502}

\smallpicfigure{figs/6502.jpg}{fig:6502}{Chip manufacturing costs in the
  1970s were dominated by the size of the die, which influenced both
  yield and packaging costs.  The 6502 was targeted to fit in an
  inexpensive 40-pin ceramic package.}

Chuck Peddle had been working at General Electric when BASIC was
born at Dartmouth.  He had since moved on to chip maker Motorola, whose 6800
microcontroller chip competed with the Intel 4004 and
8008 microprocessors.  The 6800 sold reasonably well at \$300, but
Peddle  believed that with a streamlined design they could
produce a competitive product that could sell for just \$25,
allowing Motorola to ``put
microprocessors everywhere''~\cite[p. 31]{commodore}---microwave ovens,
cars, children's toys---thereby creating whole new categories of Motorola customers.

But Motorola was skeptical and uninterested, so Peddle and a few of
his close colleagues left to join MOS~Technology\footnote{MOS stands
for metal oxide semiconductor, a phrase that describes the
breakthrough technology that made inexpensive integrated circuits
possible.}, a relatively low-profile company whose main business was
licensing designs from companies such as Texas Instruments and then
manufacturing the chips inexpensively.
In 1975, one of their mainstays was a chip containing all the logic
needed to play the hugely successful Atari Pong arcade game (``table
tennis''), which they sold to manufacturers of low-priced home video
game consoles.
Through Peddle's years of experience using the 6800 in ``real world''
applications and his skill as a chip designer, he eliminated
rarely-used features and streamlined the design to just 4300
transistors, resulting in the smaller and cheaper-to-manufacture chip
he had envisioned.
Figure~\ref{fig:6502} shows an early production model of the \w{MOS
  Technology 6502}.
%% \footnote{The 6501 was a cheaper but
%% software-compatible chip designed to be an upsell vehicle for the
%% 6502, but it was withdrawn after Motorola successfully sued MOS
%% for making the 6501 pin-compatible with the Motorola 6800.}.

\begin{tangent}
    Unlike the 6800, the 6502 lacks
    tristate drivers for the address bus, so it can't do DMA.  This
    omission was seen as a reasonable compromise
    for small systems~\cite{byte75:6502}, but would become
    problematic a few years later when hard drives became affordable, as
    it made it awkward to connect high-speed peripherals to 6502-based systems.
    The 6502 also lacked the 6800's 16-bit index register, but in exchange
    it provided two
     8-bit index registers, so that code that references arrays using
     random offsets or maintains pointers into two arrays at once would
     require about half the instruction count and execution time as on
     the 6800.
    %% Since RAM could be accessed in one
    %% clock cycle (the 6502 ran at
    %% 1~MHz) , RAM was used as overflow registers instead of devoting valuable die
    %% space to registers.  The ``zero
    %% page'' (bytes 0x0000 to 0x00ff in the 6502's 16-bit address space) had
    %% special load and store instructions that were one byte shorter to fetch and
    %% decode, since the high-order address byte was zero.
    The 6502 was also the first pipelined
    microprocessor, giving it a higher instructions-per-clock 
    rate than comparable processors and speeding up
    common instructions such as short-distance branches
     and subroutine calls.

    Later, the derivative 6507 would
    remove some other features and limit
    physical memory to a 14-bit address space.  These changes allowed
    the 6507 to
    use a smaller and cheaper 28-pin 
    plastic package and to be sold for \$5,
    making it the chip that was chosen for the Atari 2600 game console.
\end{tangent}

The 6502's simple design was created and tested essentially by hand,
since the young company could not afford the
fancy automation used by their competitors.
Peddle's colleague John Paivinen, an expert in the chemical
and physical manufacturing processes used to produce the chips, had
%% invented a technology\footnote{Contactless mask liners.} that allowed
%% the ``master'' from which chips are ``stamped'' to last indefinitely,
%% rather than wearing out gradually as more chips are produced.
figured out a way to fix defects in integrated-circuit masks even after the masks were
manufactured, something no other chip manufacturer had been able to do.
Because of this innovation, and through careful tuning of the manufacturing process,
MOS Technology's yields (percentage of chips coming off the assembly
line that worked well enough to sell) approached 70\%,
compared to around 30\% for competitors~\cite{commodore}.
The result was that MOS Technology could manufacture the 6502 for \$12
and sell it for \$25 in  quantities of one, as
Figure~\ref{fig:price_performance} shows.

\begin{figure}
  \begin{tabular}{|l|l|l|l|l|}
\hline
\textbf{Processor} &
\textbf{MIPS \& Clock} &
\textbf{Price for 1} &
\textbf{\$/MIPS}  &
\textbf{\$/MIPS today} \\
\hline

Motorola 6800 &
0.500 MIPS @ 1 MHz &
\$300 &
\$600 &
 \\

MOS Technology 6502 &
0.430 MIPS @ 1 MHz &
\$25 &
\$59 &
\\

Intel 4004 &
0.092 MIPS @ 0.75 MHz &
 &
 \\

Intel 8080 &
0.290 MIPS @ 2 MHz &
\$360 &
\$1,242 &
\\

\B{IBM System/370} &
0.640 MIPS @ 2.5 MHz &
\$475,000 &
\$742,187 &
\\

\hline
\end{tabular}

  \caption{\label{fig:price_performance} 
    The first crop of microprocessors was steadily creeping up on the
    performance of minicomputers; for reference, the IBM System/370,
    which started at \$475,000 for a complete system, ran 0.640~MIPS at
    2.5MHz. 
    The 6502 beat almost all competing microprocessors on the AH~Systems benchmarks of
    the day~\cite{edn75:6502}, as well as in absolute price and price/performance.
    Intel and Motorola offered steep (up to 80\%) wholesale discounts,
    but you might have to order hundreds of units to
    get that price, placing them beyond the reach of hobbyists~\cite[p. 228]{ceruzzi}.}
\end{figure}


%% In 1974, HP introduced the HP-65 programmable calculator, Intel
%% announced the 8080, and (in December) \emph{Popular Electronics} had the
%% Altair 8800 on the cover.  

\smallpicfigure{figs/pet2001.jpg}{fig:pet}{%
Reflecting Commodore chairman Jack Tramiel's frugal streak, the
Commodore PET's original keyboard was roundly criticized for being
made from calculator keyboard buttons made by Commodore's failing
calculator division, and the metal frame was stamped by a
Commodore-owned company that made cheap office furniture.
(The revised PET had a real full-size typewriter keyboard.)
Tramiel learned his lesson, and when Commodore introduced the 
groundbreaking VIC-20 several years later, it was widely praised for
featuring such a high-quality keyboard on a remarkably low-priced computer.}

Around that time, Commodore Business Machines took an interest in MOS
Technology.
Jack Tramiel, Commodore's ruthless chairman and a Holocaust survivor,
was obsessed with achieving low prices for consumer products through
vertical integration.
Commodore originally bought MOS Technology in order to have a
guaranteed in-house source for manufacturing chips for its
calculators~\cite{commodore}.
But Texas Instruments, which designed those chips and licensed
manufacturing rights to companies such as MOS~Technology, had just
decided to enter the consumer calculator business, in effect competing
with its own customers.
Peddle saw a chance to move Commodore into a new market: he convinced
Tramiel that the generaly public was ready for an inexpensive computer based on
the 6502.
With his MOS Technology team and a few superstar new hires, Peddle led
the design of the Commodore PET (Figure~\ref{fig:pet}), the first
ready-to-run personal computer offered for sale to consumers.


Peddle knew that since people would have to create their own software,
a personal computer would have a competitive advantage if it came with
a programming language.
Peddle had discovered BASIC while working at GE during that company's
relationship with Dartmouth, and had used BASIC literally the day
after it was invented.
He thought it would be an ideal language for the kind of hobbyists
who would buy the PET computer, and he knew that Micro-Soft had
created a version of BASIC for the 8080.
Peddle convinced Tramiel to approach Micro-Soft to create a version of
BASIC that would work on the 6502.


But how would BASIC be distributed with the PET?
Gates and Allen had already been burned by hobbyists giving away
copies of their Altair BASIC for free, thwarting their pay-per-use
business model.
The solution that Microsoft and Commodore arrived at was ingenious,
but also driven by the technical constraints of the time.
Floppy disk drives were far too expensive for PC buyers, portable hard
disks were more than a decade away, and cassette tapes were too slow
and unreliable for distributing a program the size of a BASIC
interpreter.
On the other hand, MOS Technology could inexpensively produce ROM
chips (for ``read-only memory''), which could hold copious amounts of
non-modifiable data.
Peddle proposed that Gates's BASIC be burned into a ROM chip that
would be part of every PET, establishing a precedent that would be
followed not only by all first-generation PCs, but much later with
Microsoft Windows, which would be preinstalled on new PC hard drives
rather than purchased separately by the end user.
Ruthless Commodore chairman Jack Tramiel proposed a
take-it-or-leave-it offer: Commodore would pay Micro-Soft a one-time
fee of \$25,000 (about \$100,000 in 2018) for the right to distribute
a copy of Micro-Soft BASIC with every computer they sold, forever.
\$25,000 seemed far too low for the amount of work involved, but
Micro-Soft was struggling to sell enough copies of BASIC for the
Altair 8800, so Gates seized the opportunity to acquire a new
customer.

%% had committed to port BASIC to other 8080-based devices such as smart
%% terminals from NCR and Data Terminal Corporation~\cite[p. 96]{gates} and
%% even the Motorola 6800.

Gates thought the 6502 didn't have much of a future because it was
significantly less powerful than the 8080, but he was wrong.
The 6502 turned out to be powerful enough to build other PCs around,
and although its low price eventually forced Intel and Motorola to
slash their own prices, by then a generation of hobbyists had started
using the affordable 6502 to do just that.
A positive review of the 6502 in the influential hobbyist magazine
\emph{Byte}~\cite{byte75:6502} caught the attention of a shy
electronics whiz who had been attending meetings of the Homebrew
Computer Club in Palo Alto.
The young Steve Wozniak dreamed of designing his own computer,
but Intel and Motorola only sold microprocessors in large quantities
to computer manufacturers.
Here was a microprocessor he could buy directly for just \$25.
Woz waited in line to buy his 6502 at the Wescon trade show where it was
introduced.  The computer he designed around it, the Apple~I,
would launch another legendary company.

By the mid 1980s, the 6502 powered the
Apple~II, BBC Micro, Commodore VIC-20, Commodore 64, and Atari 400 and
800 computers, as well as the Atari 2600 and Nintendo NES game
consoles, plus many lesser-known computers and game consoles---an
ironic success for a chip originally designed to power cash registers
and microwave ovens.
The PC revolution had suddenly begun\ldots{}and Microsoft was just in
time to catch the wave.


