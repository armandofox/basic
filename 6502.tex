
\section{VALLEY FORGE, 1975 - THE 6502}

\begin{milestone}{The 6502}
% milestone: the 6502.  made truly inexpensive home computers possible
% (with all due respect to sinclair)

\end{milestone}

Chuck Peddle,  who had been working at General Electric when BASIC was
born at Dartmouth, had since moved on to chip maker Motorola, whose 6800
processor chip had been introduced to compete with the Intel 4004 and
8008, the earliest microprocessors that were originally developed for
electronic calculators.  The 6800 sold successfully for \$300, but
Peddle  believed that with a few tweaks they could
produce a competitive product that could sell for a fraction of the
price: a \$20 price
point would ``put
microprocessors everywhere''~\cite[p. 31]{commodore} and make it possible for
manufacturers of all kinds of appliances and equipment to become
Motorola customers.

Unable to get Motorola interested, Peddle and a few of his close
colleagues left to start their own company, MOS Technology\footnote{MOS stands for metal oxide
semiconductor, a phrase that describes the breakthrough technology that
made integrated circuits (computer chips) possible.}.

Peddle's strength was in the logical design of the chip and in his years
of experience with how these chips would be used in ``real world''
programs and systems.
By selectively omitting some rarely-used 6800 features and enhancing
others, Peddle 
created a simple and flexible design that required only 4300
transistors.
(By comparison, complex Intel microprocessors in 2012 use over 1
billion transistors!)
The simple design made the chip smaller, an important advantage since
chip manufacturing costs were dominated by the ceramic or plastic
package (picture) the chip was mounted in.
The simple design was laid out and hand-drawn by hand by Bill Mensch,
hand-simulated by the other engineers, and hand-tested on a jury-rigged
chip testing apparatus, since the young company could not afford the
fancy automation used by their competitors.
Meanwhile, Peddle's colleague John Paivinen, an expert in the chemical
and physical manufacturing processes used to produce the chips, had
invented a technology\footnote{Contactless mask liners.} that allowed
the ``master'' from which chips are ``stamped'' to last indefinitely,
rather than wearing out bit by bit as more chips are produced.
With these advantages and careful tuning of the manufacturing process,
typically 70\% of the 6502 chips coming off MOS Technology's assembly
line worked correctly, compared to 30\% yields for competitors.

The result is that MOS Technology could manufacture the chip for \$12
and sell it for \$20 in quantities of one.  The 6502 had been designed
to compete against the simple Intel 4004 and 8008, but as it turned out,
its design and performance made it competitive with the Intel 8080,
which sold for \$360 in small quantities and \$75 to Altair 
in large quantities~\cite[p. 228]{ceruzzi},
and the \$360 Motorola 6800.  The trade paper \emph{Electronic
  Engineering Times} was skeptical that the company could provide good
technical support at such a low price, but a review in the influential
new magazine \emph{Byte}~\cite{byte75:6502} put the 6502 on the radars
of many hobbyists, including Steve Wozniak, who waited in line  to buy
one at the Wescon trade show where it was introduced,
thinking he might be able to use it in the hobbyist computer he was
designing for the Homebrew Computer Club meeting in Palo Alto.  



\begin{geeknote}
Peddle cannily made the 6502 pin-compatible and largely
assembly-code-compatible (but 
binary-incompatible) with the 6800 to take advantage of 
6800 hobbyist kits and existing motherboards, but a lawsuit from
Motorola soon put a stop to that.
\end{geeknote}


(In a similar vein, Steve Wozniak would hand-assemble the Integer BASIC that
shipped with the original Apple ][ (and which bears a striking resemblance to
 HP BASIC) and Bill Gates hand-optimized his 8080 BASIC, which ran on
 native hardware for the first time during its critical demo at MITS.)

%% In 1974, HP introduced the HP-65 programmable calculator, Intel
%% announced the 8080, and (in December) \emph{Popular Electronics} had the
%% Altair 8800 on the cover.  


    \begin{geeknote}
The 6502 reflected good engineering taste in what to leave out.
Compared to the 6800, the 6502 lacks the second accumulator, the
tristate drivers for the address bus (so you can't do DMA, which was seen as
unnecessary for small systems~\cite{byte75:6502}), 
and the 6800's 16-bit index register.  In exchange, the 6502 provides a pair
of 8-bit index registers that enables an additional kind of
indirect addressing with a true index register offset.
This design decision, based on designer Chuck Peddle's experience with real
code and his knowledge of the PDP-11 instruction set, 
reduces instruction count and execution time by up to a
factor of 2 over the 6800 for referencing arrays using random
offsets and for operations that
maintain pointers into two arrays.
As a result, the 6502
beat almost all its competitors on  the AH Systems benchmarks of
the day~\cite{edn75:6502}, despite selling for a fraction of the price.

Since RAM could be accessed in one
clock cycle (the 6502 ran at
1~MHz) , RAM was used as overflow registers instead of devoting valuable die
space to registers.  The ``zero
page'' (bytes 0x0000 to 0x00ff in the 6502's 16-bit address space) had
special load and store instructions that were one byte shorter to fetch and
decode, since the high-order address byte was zero.
The 6502 was the first pipelined
microprocessor, giving it a higher instructions-per-clock 
rate than comparable processors and making
common instructions such as relative
branch and subroutine calls 1 to 3 cycles faster than
on comparable microprocessors.  

Later, the derivative 6507 would
remove some other features and limit
physical memory to a 14-bit address space.  These changes allowed it to
use a 28-pin 
plastic package vs. 40-pin ceramic package and to be sold for \$5,
making it the chip that was chosen for the Atari 2600.
    \end{geeknote}

By the time the 6502 was on the scene, Microsoft BASIC already had
``design wins'' on 8080 devices, including the MITS Altair and
(curiously) smart terminals from NCR and Data Terminal
Corporation~\cite[p. 96]{gates}, and had committed to do a version for
the Motorola 6800.  

\begin{figure}
  \begin{tabular}{cccccc}
 \B{Licensor \& machine} & \B{Format} & \B{Date} & \B{Terms} & \B{Moniker} & \B{Units sold}  \\ \hline
Altair MITS (Altair BASIC) & Paper tape &  &   &    &    \\
Apple II (Applesoft BASIC) & ROM (8K)  & 1977 & \$22,000 & 1M+  \\
Texas Instruments TI-99/4 & & \$100,000 &    \\
Radio Shack TRS-80 (Level II BASIC) & ROM (12K), tape, disk  & \$50,000 &   \\
General Electric & 1976 & \$50,000 & Internal use \\
IBM PC 5150 (IBM BASIC) & ROM () & 1981 &  \\
Apple Macintosh (Microsoft BASIC for Mac) & disk & 1984 &   \\

\hline
  \end{tabular}
\caption{\label{fig:basic_licenses}
Table of Microsoft BASIC licenses~\cite[Chapter 8ff]{gates}}
\end{figure}
