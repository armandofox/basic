
\section{VALLEY FORGE, 1975 - THE 6502}

\smallpicfigure{figs/6502.jpg}{fig:6502}{Chip manufacturing costs in the
  1970s were dominated by the size of the die, which influenced both
  yield and packaging costs.  The 6502 was targeted to fit in an
  inexpensive 40-pin ceramic package.}

Chuck Peddle had been working at General Electric when BASIC was
born at Dartmouth.  He had since moved on to chip maker Motorola, whose 6800
microcontroller chip competed with the Intel 4004 and
8008 (the first microprocessors, originally developed for
electronic calculators).  The 6800 sold reasonably well at \$300, but
Peddle  believed that with a streamlined design they could
produce a competitive product that could sell for just \$25,
allowing Motorola to ``put
microprocessors everywhere''~\cite[p. 31]{commodore}---microwave ovens,
cars, children's toys---thereby creating whole new categories of Motorola customers.

But Motorola was skeptical and uninterested, so Peddle and a few of his close
colleagues left to join MOS~Technology\footnote{MOS stands
  for metal oxide 
semiconductor, a phrase that describes the breakthrough technology that
made inexpensive integrated circuits possible.}, a low-profile chip
fabrication company.  MOS~Technology licensed designs from other
companies such as Texas Instruments and then 
manufactured the chips inexpensively; one of their mainstays was 
selling an all-in-one chip with all the logic to play the Pong (``table
tennis'') video game, enabling the manufacture of low-priced home video game
consoles that mimicked the hugely successful Atari Pong arcade game.
Through Peddle's years
of experience using the 6800 in ``real world'' applications and his
skill as a chip designer,
he eliminated rarely-used features and streamlined the design
to just 4300 transistors, resulting in the smaller and
cheaper-to-manufacture chip he had envisioned.  Figure~\ref{fig:6502}
shows an early production model of the \w{MOS Technology 6502}\footnote{The 6501 was a 
  cheaper but software-compatible chip designed to be an upsell
  vehicle for the 6502, but it was withdrawn after Motorola successfully
  sued MOS for making the 6501 
  pin-compatible with the Motorola 6800.}.


\begin{tangent}
    Unlike the 6800, the 6502 lacks
    tristate drivers for the address bus, so it can't do DMA.  This
    omission was seen as a reasonable compromise
    for small systems~\cite{byte75:6502}, but would become
    problematic a few years later when hard drives became affordable, as
    it made it awkward to connect high-speed peripherals to 6502-based systems.
    The 6502 also lacked the 6800's 16-bit index register, but in exchange
    it provided two
     8-bit index registers, so that code that references arrays using
     random offsets or maintains pointers into two arrays at once would
     require about half the instruction count and execution time as on
     the 6800.
    %% Since RAM could be accessed in one
    %% clock cycle (the 6502 ran at
    %% 1~MHz) , RAM was used as overflow registers instead of devoting valuable die
    %% space to registers.  The ``zero
    %% page'' (bytes 0x0000 to 0x00ff in the 6502's 16-bit address space) had
    %% special load and store instructions that were one byte shorter to fetch and
    %% decode, since the high-order address byte was zero.
    The 6502 was also the first pipelined
    microprocessor, giving it a higher instructions-per-clock 
    rate than comparable processors and speeding up
    common instructions such as short-distance branches
     and subroutine calls.

    Later, the derivative 6507 would
    remove some other features and limit
    physical memory to a 14-bit address space.  These changes allowed
    the 6507 to
    use a smaller and cheaper 28-pin 
    plastic package and to be sold for \$5,
    making it the chip that was chosen for the Atari 2600 game console.
\end{tangent}

The 6502's simple design was created and tested essentially by hand,
since the young company could not afford the
fancy automation used by their competitors.
Peddle's colleague John Paivinen, an expert in the chemical
and physical manufacturing processes used to produce the chips, had
%% invented a technology\footnote{Contactless mask liners.} that allowed
%% the ``master'' from which chips are ``stamped'' to last indefinitely,
%% rather than wearing out gradually as more chips are produced.
figured out a way to fix defects in integrated-circuit masks even after the masks were
manufactured, something no other chip manufacturer had been able to do.
Because of this innovation, and through careful tuning of the manufacturing process,
MOS Technology's yields approached 70\%,
compared to around 30\% for competitors~\cite{commodore}.
The result was that MOS Technology could manufacture the 6502 for \$12
and sell it for \$25 in  quantities of one, as
Figure~\ref{fig:price_performance} shows.

\begin{figure}
  \begin{tabular}{|l|l|l|l|l|}
\hline
\textbf{Processor} &
\textbf{MIPS \& Clock} &
\textbf{Price for 1} &
\textbf{\$/MIPS}  &
\textbf{\$/MIPS today} \\
\hline

Motorola 6800 &
0.500 MIPS @ 1 MHz &
\$300 &
\$600 &
 \\

MOS Technology 6502 &
0.430 MIPS @ 1 MHz &
\$25 &
\$59 &
\\

Intel 4004 &
0.092 MIPS @ 0.75 MHz &
 &
 \\

Intel 8080 &
0.290 MIPS @ 2 MHz &
\$360 &
\$1,242 &
\\

\B{IBM System/370} &
0.640 MIPS @ 2.5 MHz &
\$475,000 &
\$742,187 &
\\

\hline
\end{tabular}

  \caption{\label{fig:price_performance} 
    The first crop of microprocessors was steadily creeping up on the
    performance of minicomputers (the comparison is not entirely fair,
    since for the IBM System/370 we show the total price, whereas for
    the microprocessors we show only the price of the chip and not a
    complete computer).
    The 6502 offered far better
    price-performance than its contemporaries,
    beating almost all of them on the AH~Systems benchmarks of
    the day~\cite{edn75:6502}.  
    The single-unit price is shown; wholesale discounts could be
    as much as 80\%, but you might have to order hundreds of units to
    get that price~\cite[p. 228]{ceruzzi}.}
\end{figure}


%% In 1974, HP introduced the HP-65 programmable calculator, Intel
%% announced the 8080, and (in December) \emph{Popular Electronics} had the
%% Altair 8800 on the cover.  

Around that time, Commodore Business Machines took an interest in MOS
Technology.  Commodore's ruthless chairman,
Jack Tramiel, was 
obsessed with achieving low prices through vertical integration.
Commodore bought
MOS Technology outright in order to have a guaranteed in-house source of
chips for its calculators~\cite{commodore}.
But Texas Instruments, which had been licensing chip designs for MOS
Technology to manufacture, had just decided
to enter the consumer calculator business, in effect competing with its own
customers. 
Peddle saw a chance to move Commodore into a new market: he convinced
Tramiel that the market was ready for an inexpensive computer
based on the 6502.
With his MOS Technology team and a few superstar
new hires, Peddle led the design of the Commodore PET (Figure~\ref{fig:pet}), 
the first ready-to-run personal computer offered for sale to consumers.

Peddle knew that since people would have to create their own software, a
personal computer would have a competitive advantage if it shipped with
a programming language built right in.
But floppy disk drives were far too expensive for PC buyers, portable
hard disks weren't even on the horizon, and cassette tapes were slow and
unreliable.  On the other hand, MOS Technology could produce low-cost
(though low-capacity) ROM 
chips.  So Peddle decided to  put the computer language interpreter in ROM, establishing
a precedent that nearly all first-generation PCs would follow.

\smallpicfigure{figs/pet2001.jpg}{fig:pet}{%
Reflecting Commodore chairman Jack Tramiel's frugal streak,
the Commodore PET's original keyboard was made from calculator-keyboard
parts, and the metal frame was stamped by another Commodore-held company
that made cheap office furniture.  The revised PET had a real full-size
typewriter keyboard.}


%% had committed to port BASIC to other 8080-based devices such as smart
%% terminals from NCR and Data Terminal Corporation~\cite[p. 96]{gates} and
%% even the Motorola 6800.
Peddle had discovered BASIC while working at GE during that company's
relationship with Dartmouth, and had used BASIC literally the day after
it was invented.  He thought it would be an ideal language for the kind
of hobbyists who would buy the PET computer, and he had heard about Micro-Soft
BASIC for the 8080.
Peddle convinced Commodore chairman Tramiel to approach Micro-Soft with
an offer: for a flat fee of \$25,000,
Micro-Soft would port their BASIC to work with the 6502-based Commodore
PET, and would 
allow Commodore to package it with every computer they sold.
Gates had initially disdained the 6502 as underpowered compared to the
8080, and \$25,000 seemed far too low.  But Micro-Soft was struggling to
sell enough copies of BASIC for the Altair 8800, so Gates
seized the opportunity to acquire a new customer.

Gates thought the 6502 didn't have much of a future because it was less
powerful than the 8080, but he was wrong.
Although the 6502's low price eventually forced Intel and Motorola to slash
their own prices, by then a generation of
hobbyists had started using the affordable 6502 to build their own computers.
While Commodore was developing the PET, a positive review of the 6502 in
the influential hobbyist magazine \emph{Byte}~\cite{byte75:6502} caught
the attention of electronics whiz Steve Wozniak, who waited in line to
buy a 6502 at the 
Wescon trade show where it was introduced.
The shy Woz had been attending meetings of the Homebrew Computer Club in
Palo Alto and was hoping to design a computer he could demonstrate
there; that computer would eventually be called the Apple~I and would
launch another legendary company.

By the mid 1980s, the 6502 or one of its derivatives powered the Apple
II, BBC Micro, Atari 2600 game console, Commodore VIC-20, Commodore 64,
Atari 400 and 800 computers, the Nintendo NES, and many lesser-known
computers and game consoles---an ironic success for a chip
originally designed to power cash registers and microwave ovens.
The PC revolution had suddenly begun\ldots{}and Microsoft was just in time
to catch the wave.

