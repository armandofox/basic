
\section{VALLEY FORGE, 1975 - THE 6502}

Chuck Peddle had been working at General Electric when BASIC was
born at Dartmouth.  He had since moved on to chip maker Motorola, whose 6800
microprocessor chip competed with the Intel 4004 and
8008 (the first microprocessors, originally developed for
electronic calculators).  The 6800 sold well at \$300, but
Peddle  believed that with a streamlined design they could
produce a competitive product that could sell for just \$20,
allowing Motorola to ``put
microprocessors everywhere''~\cite[p. 31]{commodore}---microwave ovens,
cars, children's toys---creating whole new categories of Motorola customers.

But Motorola was skeptical and uninterested, so Peddle and a few of his close
colleagues started their own company, MOS Technology\footnote{MOS stands
  for metal oxide 
semiconductor, a phrase that describes the breakthrough technology that
made integrated circuits (computer chips) possible.}.
Through Peddle's years
of experience with how these chips were used in ``real world'' settings,
he streamlined the 6800's design 
down to just 4300 transistors, resulting in the smaller and
cheaper-to-manufacture chip that Figure~\ref{fig:6502} shows.  Peddle
christened it the \w[MOS_Technology_6502]{6502}\footnote{The 6501 was a
  cheaper but software-compatible chip designed to be an upsell
  vehicle for the 6502. However, Motorola sued MOS because the 6501 was
  pin-compatible with the 6800, so the 6501 was soon quietly withdrawn.}

\smallpicfigure{figs/6502.jpg}{fig:6502}{Chip manufacturing costs in the
  1970s were dominated by the size of the die, which influenced both
  yield and packaging costs.  The 6502 was targeted to fit in an
  inexpensive 40-pin ceramic package.}

The simple design was created and tested essentially by hand,
since the young company could not afford the
fancy automation used by their competitors.
Peddle's colleague John Paivinen, an expert in the chemical
and physical manufacturing processes used to produce the chips, had
invented a technology\footnote{Contactless mask liners.} that allowed
the ``master'' from which chips are ``stamped'' to last indefinitely,
rather than wearing out gradually as more chips are produced.
And by carefully tuning the manufacturing process,
typically 70\% of the 6502 chips coming off MOS Technology's assembly
line worked correctly, compared to around 30\% for competitors~\cite{commodore}.

The result was that MOS Technology could manufacture the chip for \$12
and sell it for \$25 in  quantities of one.  The 6502 was priced to
compete against
the simple Intel 4004 and 8008, but its performance was actually
comparable to  the \$360
Intel 8080 (which even at wholesale cost \$75~\cite[p. 228]{ceruzzi}),
and faster than the Motorola 6800 that Peddle had originally set out to beat.
The industry magazine \emph{Electronic
  Engineering Times} was skeptical that such a small company could provide good
technical support at such a low price. 
But a positive review of the 6502 in the influential
hobbyist magazine \emph{Byte}~\cite{byte75:6502} caught the attention of
Steve Wozniak, who waited in line  to buy
a 6502 at the Wescon trade show where it was introduced. 
Woz had been attending meetings of the 
Homebrew Computer Club in Palo Alto and was hoping to design a computer
he could demonstrate there; that computer would eventually be called
``the Apple I'' and would launch another legendary company.


%% In 1974, HP introduced the HP-65 programmable calculator, Intel
%% announced the 8080, and (in December) \emph{Popular Electronics} had the
%% Altair 8800 on the cover.  


\begin{tangent}
    The 6502 reflected good engineering taste in what to leave out.
    Compared to the 6800, the 6502 lacks the second accumulator, the
    tristate drivers for the address bus (so it can't do DMA, which was seen as
    unnecessary for small systems~\cite{byte75:6502}), 
    and the 6800's 16-bit index register.  In exchange, the 6502 provides two
     8-bit index registers that enable
    indirect addressing with a true index register offset.
    This design decision, based on Peddle's experience with real
    code and his knowledge of the PDP-11 instruction set, 
    reduces instruction count and execution time by up to a
    factor of 2 over the 6800 for referencing arrays using random
    offsets and for operations that
    maintain pointers into two arrays.
    %% Since RAM could be accessed in one
    %% clock cycle (the 6502 ran at
    %% 1~MHz) , RAM was used as overflow registers instead of devoting valuable die
    %% space to registers.  The ``zero
    %% page'' (bytes 0x0000 to 0x00ff in the 6502's 16-bit address space) had
    %% special load and store instructions that were one byte shorter to fetch and
    %% decode, since the high-order address byte was zero.
    The 6502 was also the first pipelined
    microprocessor, giving it a higher instructions-per-clock 
    rate than comparable processors and making
    common instructions such as relative
    branch and subroutine calls 1 to 3 cycles faster than
    on comparable microprocessors.  
    As a result, the 6502
    beat almost all its competitors on  the AH~Systems benchmarks of
    the day~\cite{edn75:6502}, despite selling for a fraction of the price.

    Later, the derivative 6507 would
    remove some other features and limit
    physical memory to a 14-bit address space.  These changes allowed it to
    use a smaller and cheaper 28-pin 
    plastic package and to be sold for \$5,
    making it the chip that was chosen for the Atari 2600 game console.
\end{tangent}

Commodore, a calculator company whose ruthless chairman Jack Tramiel was
obsessed with achieving low prices through vertical integration, bought
MOS Technology outright in order to have a guaranteed in-house source of
6502 chips for its calculators~\cite{commodore}.
Commodore had been buying its chips from Texas Instruments,
but now TI was selling calculators as well, competing with its own
customers.
Chuck Peddle had long been musing about a possible inexpensive personal computer based on
the 6502, and now he had his chance.  With his MOS Technology team and a
few superstar new hires, Peddle led the 
design of the Commodore PET, which would be the first ready-to-run
personal computer offered for sale to consumers.

Peddle knew that since people would have to create their own software, a
personal computer would have a competitive advantage if it shipped with
a programming language built right in.
At the time, floppy disk drives were far too expensive for PC buyers,
portable hard disks weren't even on the horizon, and cassette tapes were
too slow and unreliable for loading large programs such as a language interpreter.
So the only practical way to include a programming
language along with the computer would be to put the
interpreter in ROM alongside the computer's BIOS (basic input/output system).
Happily, MOS Technology could produce low-cost (but low-capacity) ROM chips.

Meanwhile, Microsoft was struggling to sell enough
copies of BASIC for the Altair 8800 to make the business worthwhile, and
had committed to port BASIC to other 8080-based devices such as smart
terminals from NCR and Data Terminal Corporation~\cite[p. 96]{gates} and
even the Motorola 6800.
So chairman Tramiel negotiated a deal with Bill Gates: for a
one-time flat fee of \$25,000, Micro-Soft would port their BASIC to
the 6502 and fit it into MOS Technology's limited-capacity ROMs,
and allow Commodore to package it with every computer.
Gates initially disdained the 6502 as underpowered compared to the 8080,
but was quick to seize the new sales opportunity.

The Commodore PET thereby established the pattern that all PC makers
would follow: BASIC is built right into the hardware, and when you turn
on the machine out of the box, even with no disk or cassette storage
attached, you're immediately dropped into BASIC and you can start
programming.

The 6502's hobbyist-friendly price and solid performance suddenly
kicked the PC revolution into full swing---somewhat ironically, since
the 6502 had targeted cash registers and microwave ovens rather than
microcomputers---and Microsoft was in a great position to catch the
wave.

