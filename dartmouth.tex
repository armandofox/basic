
\section{DARTMOUTH COLLEGE, 1964 - FOCUS ON SIMPLICITY}

\makequotation{In cases where there is a choice between simplicity and efficiency,
simplicity is chosen\ldots{}no attempt will be made at making the use of the Time Sharing 
equipment be compatible with \emph{standard use of a computer.}}{Time Sharing Project Memorandum \#1,
Dartmouth College, November 6, 1963~\cite{hopl}} 

\picfigure{figs/ASR33.jpg}{fig:ASR33}{The Teletype ASR-33 printing
  terminal.}

Dartmouth had obtained a GE-225 computer system from General Electric
via two National Science Foundation research grants and
an educational discount from GE.  It was a bold decision that the
``environment'' in which beginners would learn programming need not have
anything in common with the environment used by scientists and engineers
to do their work (the ``standard use of a computer'' in the quote above).


Kemeny tried to convince Kurtz to use a subset of FORTRAN~I, which had
debuted with great success in 1957 as a language for scientists and
engineers to express mathematical problems.  
But Kurtz felt strongly that even existing high-level
languages, which had been designed for domain experts such as scientists
(FORTRAN) or business people (COBOL) 
had enough idiosyncrasies, quirks, and arbitrary rules of syntax and
semantics to interfere with lay people's ability to focus on the
concepts.  
And whereas FORTRAN had to be designed to be performance-competitive with
handcrafted assembly language programs, 
only \emph{reasonable} performance was necessary for BASIC, so the
design focused on 
ease of use.

For example, FORTRAN variable names beginning with letters I
through N always represented integers, whereas other variables represented
real numbers; in BASIC, a number is a number, and automatic type
promotion takes care of converting between integers and reals.  In
fact, BASIC didn't even require you to declare your variables in
advance of assigning them, as FORTRAN did.  FORTRAN's
conditional statement had the form 
\T{IF}~\emph{condition,label1,label2,label3}, leaving the reader to
understand the meaning of the punctuation and how the three labels
relate to the evaluation of the condition; BASIC uses English
words and a two-way
conditional---\T{IF}~\emph{condition}~\T{THEN}~\emph{label}---to
clarify intent.  FORTRAN's \T{DO} loop body always executes at least once,
even if the condition at the top of the loop is initially false; not
so for BASIC's \T{FOR\ldots{}NEXT} loop.

Since industrial systems were primarily batch mode, the Dartmouth
professors and students had to create  their own timesharing operating
system for the Ge computer, based on the ideas prototyped at MIT.  Their
system was ultimately so successful that it catapulted GE into a new
and profitable business selling timesharing.
Although that business failed after a few
years because customers preferred cheaper but less-friendly batch
processing, it gave GE engineer Chuck Peddle his
first exposure to BASIC, literally the day after it was
invented~\cite[p.~5]{commodore}.  Years later, Peddle would design the
6502 microprocessor, which launched the PC revolution that would make
BASIC the most widely-used computer language in the world.

BASIC borrowed good ideas from other languages.  Like COBOL, every BASIC
statement starts with an English verb.  (This was true in the original
BASIC, though not always in Street BASICs.)  The language's simple
design made it possible for the compiler to be implemented by a team of
undergraduates: the first BASIC had only 14 commands, and used notation
for arithmetic 
equations that would be familiar to anyone with high school algebra knowledge.


  \begin{geeknote}
   This characteristic was exploited by Woz's Apple Integer BASIC and by
   Sinclair ZX80 BASIC to 
    provide a context-sensitive syntax checker that verified the syntax
    of your BASIC program lines as you typed them, rather than at
    runtime~\cite{zx80_basic_techreport}.  Woz's BASIC also stored the
    program in a parsed form that distinguished different uses of the
    same BASIC keyword, saving time during execution.
  \end{geeknote}

The original 1964 BASIC manual~\cite[p. 14]{dartmouth_basic_manual}
makes clear that the
creators' intent was a beginner-friendly ``BASIC programming appliance'' 
After a short (10-page double-spaced) exposition of BASIC itself, the
manual explains the steps students must take to begin writing programs: 
type \T{HELLO} at an ASR-33
Teletype, enter your student ID number when asked, 
and then type the name of the system you want to use (\T{BASIC}).
That's the entire contents of the section on ``how to use the computer
system'': three pages, one of which is a full-page diagram of the ASR-33
keyboard 
explaining weird keys such as \T{CTRL} to typists in 1964.
Given that most batch-processing systems required programmers to
learn and use a 
separate \w{job control} language to add 
job-management cards to the card deck, shielding novices from this
distinction was a bold move, and one that would remain a characteristic of
PC BASICs, in part necessitated by the low cost and
modest resources of those machines.

% (film: montage of power-on sequences of various home computers)


BASIC was also influenced by the technology of the time.  
FORTRAN was used in batch mode, so the order of program statements was
determined by the order of the punched cards in the program deck, since
each statement required its own card.
%% With CRT terminals years away, how to specify the order of statements in
%% a BASIC program?  Indeed, how to shield beginners from having to learn
%% to work with other elements of the computer system, such as files and editors?
Kemeny and Kurtz came up with an ingenious solution to the problem of
how to specify the order of BASIC statements: each line in a BASIC
program would begin with a line number, and the program would be
executed in order of ascending line numbers.
Typing a line of text consisting of a number followed by a BASIC statement
would create (or replace) that line in the current program.
Typing a line of text consisting of only a line number would delete that line.
Typing \T{LIST} would show
you all or part of your program in line-number order.
\T{SAVE} stored a copy of the current program with a name
chosen by the programmer, \T{OLD} (later \T{LOAD} in Street BASIC)
retrieved a previously-stored program, and \T{NEW} erased the current
program in order to start new work.  All of these commands were part
of the language.

  \begin{geeknote}
  The convention of line numbers was retained in virtually all ``street
  BASICs'' long after cursor-addressable CRT terminals were ubiquitous,
  and was finally dropped in Microsoft Visual Basic (although many modern
  BASICs, including VB, still allow them).
  \end{geeknote}

A more pronounced example that's often overlooked is BASIC's \T{INPUT}
statement, which means ``At this point in the program, stop and wait for
the user to type something, then record what was typed and proceed.''
Before timesharing, this concept was absent from programming, since the
idea of the computer stopping and waiting for a human to type was
ludicrous.  But since timesharing used that ``wasted'' waiting time to
serve other users, the idea of programs that would interact with the
user became a reality, and BASIC was the first high-level language to
support this concept directly.  (FORTRAN programs had to be accompanied
by a stack of ``data cards'' specifying the particular values on which
the program would operate, and a \T{READ} statement in the language
would consume the values from the data cards.)

These examples show how Kurtz and Kemeny worked around the technology of their
time without cluttering the concepts that beginners would learn.



