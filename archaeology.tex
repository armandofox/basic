
\section{ARCHAEOLOGY}

\B{BASIC Interpreters, Compilers, and Emulators.}

QuiteBASIC is my personal favorite for meeting the goal of a simple
BASIC that is beginner-friendly.  That said,
Microsoft's DevLabs 
released \href{http://smallbasic.com}{Microsoft Small Basic} in 2008 as a
Technology Preview plug-in to Visual Studio.
The idea is well-intentioned, although hardly small---Small Basic
requires Windows XP or later, meaning a 500
MHz x86 processor, several gigabytes of disk space, and at least 256
megabytes of RAM.
Also, because of Small Basic's explicitly
object-oriented syntax,  a novice must either understand why she
must say \C{TextWindow.WriteLine("Hello world")} to display text on the
screen or accept the arbitrariness of that syntax---one of the very problems
BASIC's creators set out to eliminate.  

A more authentic route for the historically-minded is to use emulators.
You can try a lot of the software yourself, and explore
its source code, thanks to the many great emulators available and the
collection of emulator-compatible disk images floating around the Internet.

\begin{itemize}

\item \href{http://quitebasic.com}{Quite BASIC} is a free in-browser BASIC
  interpreter that is probably closest in spirit to what David Brin
  lamented missing in his Salon article.  It has a graphics canvas
  allowing programs to do simple graphics.

\item \href{http://virtualii.com}{Virtual-][} is an extraordinarily
  faithful Apple II emulator, right down to the sound effects of
  whirring disk drives and dot matrix printing.

\item \href{http://web.archive.org/web/20011211231432/http://www.rjh.org.uk/altair/4k/em/altem.htm}{Altair emulator with BASIC} (Java applet emulator; doesn't run in Safari,
maybe needs earlier JRE?)

\end{itemize}


% computer history museum

\B{Old source code.}
There is a long saga about trying to get access to the original
Microsoft BASIC source code---the 8080 Altair BASIC authored by Bill
Gates, Paul Allen and Monte Davidoff that launched the PC revolution.
The true urtext is still elusive, believed to be locked in Bill Gates's
personal safe.
The Pusey Library at Harvard University has a copy (which you may
inspect but not photograph\ldots{}WTF?) annotated with version number
1.1.
You can read more about this archaeological quest at
\href{http://www.theregister.co.uk/2001/05/13/raiders_of_the_lost_altair/}{Raiders
of the Lost Altair BASIC Source Code}, The Register, May 13, 2001,
which tells the overall story.
\href{http://www.interact-sw.co.uk/altair/other\%20versions/ian.htm}{Quest
for the Holy Source: Ian's Trip to Harvard} tells about Ian
Griffiths's visit to the Pusey Library at Harvard to view version
1.1.
He reports that the library's copy includes a transmittal letter from
Harry Lewis, Dean of Harvard College and a professor of Computer
Science there, who reportedly found this copy of the source code behind
a cabinet in an old CS office.
The letter reports that other computer scientists, including Donald
Knuth, have made ``pilgrimages'' to Harvard to see this copy.

\href{http://www.drdobbs.com/back-to-the-future/184404733}{Bill's
  Lost Code} (a section of the Back to the Future column in this online
issue of Dr. Dobbs' Journal) observes that based on reading this source
code, Gates was indeed a craftsman and hacker, pulling nifty tricks such
as jumping into the middle of an instruction (because a multi-byte
opcode had a different interpretation if you started reading at the 2nd
or 3rd byte) to squeeze a BASIC interpreter into just 4KiB of ROM.

Reuben Harris, a London programmer, has created an
\href{http://web.archive.org/web/20011211233332/www.rjh.org.uk/altair/4k/index2.html}{annotated
disassembly of the actual Altair 4K BASIC} that you can visit thanks to the
Wayback Machine.
