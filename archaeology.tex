
\section{ARCHAEOLOGY}

\B{Retrocomputing}

It's been decades since BASIC was in ROM, but widely-available emulators
and reimplementations make it easy to experience BASIC.
QuiteBASIC is my personal favorite for meeting the goal of a simple
BASIC that is beginner-friendly, with a nice GUI and basic graphics.
Microsoft's DevLabs released \href{http://smallbasic.com}{Microsoft
Small Basic} in 2008 as a plug-in to Visual Studio, but while
well-intentioned, it's hardly small---Small Basic requires Windows XP or
later, and several gigabytes of disk space and at least 256 megabytes of
RAM to run Visual Studio,
Also, because of Small Basic's explicitly object-oriented syntax, a
novice must either understand why she must say
\C{TextWindow.WriteLine("Hello world")} to display text on the screen or
accept the arbitrariness of that syntax---one of the very problems
BASIC's creators set out to eliminate.

A more authentic route for the historically-minded is to use emulators.
You can try a lot of the software yourself, and explore
its source code, thanks to the many great emulators available and the
collection of emulator-compatible disk images floating around the Internet.

\begin{itemize}

\item \href{http://virtualii.com}{Virtual-][} is an extraordinarily
  faithful Apple II emulator, right down to the sound effects of
  whirring disk drives and dot matrix printing.

\item \href{http://web.archive.org/web/20011211231432/http://www.rjh.org.uk/altair/4k/em/altem.htm}{Altair emulator with BASIC} (Java applet emulator; doesn't run in Safari,
maybe needs earlier JRE?)

\end{itemize}

% computer history museum

\B{Old source code for influential BASIC programs.}
The
\href{http://web.archive.org/web/20081209035052/http://www.dunnington.u-net.com/public/startrek/}{Star
  Trek Page} archives source code for over 30 variants of the Star Trek
game in different BASICs, plus ports to other languages including LISP, C, and
JavaScript, a testament to the game's longevity and influence.

\B{Old source code for BASIC itself.}

TBD: visiting library at Dartmouth

There is a long saga~\cite{raiders} about trying to get access to the original
Micro-Soft BASIC interpreter source code.
The true \emph{urtext\/} is still elusive, believed to be locked in Bill Gates's
personal safe.
The Pusey Library at Harvard University has a copy (which, confusingly, you may
inspect but not photograph, annotated with version number 1.1;
Ian Griffiths reports~\cite{ians_trip} that the library's copy includes
a transmittal letter from 
Harry Lewis, Dean of Harvard College and a professor of Computer
Science there, who reportedly found this copy of the source code behind
a cabinet in an old CS office.
The letter reports that other computer scientists, including Donald
Knuth, have made ``pilgrimages'' to Harvard to see this copy.

\href{http://www.drdobbs.com/back-to-the-future/184404733}{Bill's
  Lost Code} (a section of the Back to the Future column in this online
issue of Dr. Dobbs' Journal) observes that based on reading this source
code, Gates was indeed a craftsman and hacker, pulling nifty tricks such
as jumping into the middle of an instruction (because a multi-byte
opcode had a different interpretation if you started reading at the 2nd
or 3rd byte) to squeeze a BASIC interpreter into just 4KiB of ROM.


Reuben Harris, a London programmer, has created an
annotated
disassembly of the actual Altair 4K BASIC 
that you can visit (\url{http://web.archive.org/web/20011211233332/www.rjh.org.uk/altair/4k/index2.html}) thanks to the
Wayback Machine.
